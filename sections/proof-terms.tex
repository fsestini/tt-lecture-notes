\section{Proof terms and the Curry-Howard isomorphism}

\subsection{Historical notes (WIP)}

The term ``Curry-Howard isomorphism'', also stated as the slogans ``propositions
as types'' or ``proofs as programs'', originally refers to a strong
correspondence between formal systems of constructive logic and typed lambda
calculi, and in a more general and modern view, between proof theory and
theoretical computer science.

The BHK interpretation of intuitionistic logic was already a hint towards this
idea (although the formulas-as-types paradigm is only one of the possible
incarnations of BHK, among others such as Kleene's realizability semantics).
The first explicit statement was made by Curry in the 40s, and the
correspondence was made precise for typed combinatory logic by Curry and Feys in
their milestone monograph \cite{curry1958combinatory}, in the form of theorems
giving a connection between provability and type inhabitation.

The works of Curry, however, accounts only for one aspect of the isomorphism,
namely the \emph{static} correspondence between propositions and types, as well
as proof terms and $\lambda$-terms (or combinators). The other important aspect
is \emph{dynamic}, and relates proof normalization in natural deduction and term
reduction (computation) in typed lambda calculi. W. W. Tait is often credited
for the discovery of this dynamic aspect of the correspondence, but an explicit
statement of it did not occur until 1969, when the famous paper by Howard began
to be privately circulated (it was actually published only in 1980). In
\cite{howard:tfatnoc}, explicit emphasis on the relationship between reduction
and normalization is made.

Despite the isomorphism being associated with Curry and Howard, the work by
N. G. De Bruijn on its tool Automath \cite{aut-001-1} represented a huge
contribution, and was the first example of making an actual practical use of the
ideas underlying the isomorphism. Automath was developed as a proof assistant
for writing mathematical proofs so that they could be verified by a
computer. Many fundamental concepts were for the first time used here, such as
dependent types, with implication defined as a special case of the universal
quantifier.

The ultimate development of the isomorphism came with the work of Per
Martin-Loef and its Intuitionistic Type Theory: the correspondence between
propositions and types is brought to an extreme, to the point where propositions
are \emph{identified} with types. In the following years, type theories, proof
assistants and programming languages have been created following the idea of the
Curry-Howard isomorphism, and inspired especially by Martin-Loef’s work. Notable
examples are Coquand and Huet’s Calculus of Constructions \cite{COQUAND198895},
the proof assistants Coq and Lego, and the dependently typed functional
programming languages Agda \cite{Norell:2009:DTP:1481861.1481862} and Idris.

From the point of view of categorical semantics, we discover yet another side of
the same correspondence. Objects and morphisms correspond to types and terms,
and also to formulas and proofs. This analogy was first discovered by Lawvere
and Lambek, to that the correspondence is sometimes stated as the
Curry-Howard-Lambek isomorphism. TODO: computational trinitarianism.


\subsection{Multi-sorted natural deduction with explicit context}

We now introduce explicit contexts, and see how they can be defined. In
particular, we are going to introduce a new judgement

\[
  \gammactxt
\]

and explain how it can be derived:

\[
  \ctxtj{[]}
  \qquad
  \begin{prooftree}
    \Gamma \vdash \propj{\varphi}
    \justifies
    \ctxtj{\Gamma, \tyj{u}{\varphi}}
    \using{u \text{ fresh}}
  \end{prooftree}
  \qquad
  \begin{prooftree}
    \Gamma \vdash \setj{A}
    \justifies
    \ctxtj{\Gamma, \inj{x}{A}}
    \using{x \text{ fresh}}
  \end{prooftree}
\]

that is, our contexts may contain hypotheses on the existence of elements of
sets and proofs of propositions, must the \emph{name} or \emph{tag} of each
hypothesis must be unique in the context. This restriction ensures that when an
assumption is removed from the context, it gets actually \emph{discharged} in
the sense that it cannot be used further in the derivation.  The most basic use
of an assumption is to assert it: if we assume $x$ is a proof of a proposition
$A$, then we are entitled to assert that $A$ is true, hence

\[
  \begin{prooftree}
    \ctxtj{\Gamma, \tyj{x}{\varphi}, \Gamma'}
    \justifies
    \Gamma, \tyj{x}{\varphi}, \Gamma' \vdash \truej{\varphi}
  \end{prooftree}
\]

In the same way, if we assume that $\inj{x}{A}$, we may as well assert it:

\[
  \begin{prooftree}
    \ctxtj{\Gamma, \inj{x}{A}, \Gamma'}
    \justifies
    \Gamma, \inj{x}{A}, \Gamma' \vdash \inj{x}{A}
  \end{prooftree}
\]

Notice that we implicitly assume that the context formation jusgements are
extended with their hypothetical forms, for example

\[
  \begin{prooftree}
    \Gamma, \inj{x}{A}, \Gamma' \vdash \propj{\varphi(x)}
    \justifies
    \ctxtj{\Gamma, \inj{x}{A}, \Gamma', \tyj{u}{\varphi(x)}}
  \end{prooftree}
\]

is a perfectly valid derivation, provided the variable $u$ did not already occur
in the context. Notice, also, that from our definition it follows
that in a context $\Gamma, \inj{x}{A}, \Gamma'$, $\Gamma$ neither contains
$\inj{x}{A}$, nor it contains $x$ free. This property will come in handy when we
introduce substitution in contexts.
The complete natural deduction system for multi-sorted first-order
intuitionistic logic with explicit contexts is shown in
Figure~\ref{natded-explicit-context}.

\begin{figure}[ht]
\begin{mdframed}

  \[
    \begin{prooftree}
      \Gamma \vdash \varphi \qquad \Gamma \vdash \psi
      \justifies
      \Gamma \vdash \andconn{\varphi}{\psi}
    \end{prooftree}
    \qquad
    \begin{prooftree}
      \Gamma \vdash \andconn{\varphi}{\psi}
      \justifies
      \Gamma \vdash \varphi
    \end{prooftree}
    \qquad
    \begin{prooftree}
      \Gamma \vdash \andconn{\varphi}{\psi}
      \justifies
      \Gamma \vdash \psi
    \end{prooftree}
  \]

  \[
    \begin{prooftree}
      \Gamma \vdash \varphi
      \justifies
      \Gamma \vdash \orconn{\varphi}{\psi}
    \end{prooftree}
    \qquad
    \begin{prooftree}
      \Gamma \vdash \psi
      \justifies
      \Gamma \vdash \orconn{\varphi}{\psi}
    \end{prooftree}
    \qquad
    \begin{prooftree}
      \Gamma \vdash \orconn{\varphi}{\psi}
      \quad
      \Gamma, \tyj{u}{\varphi} \vdash C
      \quad
      \Gamma, \tyj{v}{\psi} \vdash C
      \justifies
      \Gamma \vdash C
    \end{prooftree}
  \]

  \[
    \begin{prooftree}
      \Gamma, \tyj{u}{\varphi} \vdash \psi
      \justifies
      \Gamma \vdash \implconn{\varphi}{\psi}
    \end{prooftree}
    \qquad
    \begin{prooftree}
      \Gamma \vdash \implconn{\varphi}{\psi} \qquad \Gamma \vdash \varphi
      \justifies
      \Gamma \vdash \psi
    \end{prooftree}
  \]

  \[
    \begin{prooftree}
      \Gamma, \inj{x}{A} \vdash \propj{\varphi(x)}
      \quad
      \Gamma, \inj{x}{A} \vdash \varphi(x)
      \justifies
      \Gamma \vdash \forallconn{x}{A}{\varphi(x)}
    \end{prooftree}
    \qquad
    \begin{prooftree}
      \Gamma \vdash \forallconn{x}{A}{\varphi(x)}
      \quad
      \Gamma \vdash \inj{t}{A}
      \justifies
      \Gamma \vdash \varphi(t)
    \end{prooftree}
  \]

  \[
    \begin{prooftree}
      \begin{array}{@{}c}
        \Gamma, \inj{x}{A} \vdash \propj{\varphi(x)} \\
        \Gamma \vdash \varphi(t)
        \qquad
        \Gamma \vdash \inj{t}{A}  
      \end{array}
      \justifies
      \Gamma \vdash \existsconn{x}{A}{\varphi(x)}
    \end{prooftree}
    \qquad
    \begin{prooftree}
      \begin{array}{@{}c}
        \Gamma \vdash \propj{\psi}
        \qquad
        \Gamma \vdash \exists{x}{A}{\varphi(x)} \\
        \Gamma, \inj{x}{A}, \tyj{y}{\varphi(x)} \vdash \psi
      \end{array}
      \justifies
      \Gamma \vdash \psi
    \end{prooftree}
  \]
  \[
    \begin{prooftree}
      \justifies
      \Gamma \vdash \trueconn
    \end{prooftree}
    \qquad\qquad\qquad
    \begin{prooftree}
      \Gamma \vdash \falseconn
      \justifies
      \Gamma \vdash \varphi
    \end{prooftree}
  \]
\end{mdframed}
\caption{\label{natded-explicit-context} Natural deduction with explicit context.}
\end{figure}

\begin{example}
  We show how the intuitionistically valid formula

  \[
    \varphi \vee (\forall x \in A)\psi(x) \rightarrow
    (\forall x \in A)(\varphi \vee \psi(x))
  \]

  can be derived in our system, where $x$ is intended to be \emph{not} free in
  $\varphi$.

  \[
    \begin{prooftree}
      \[
        \[
          \Gamma \vdash \varphi \vee (\forall x \in A)\psi (x) \quad 
          \[
            \Gamma, \varphi \vdash \varphi
            \justifies
            \Gamma, \varphi \vdash \varphi \vee \psi (x)
          \]
          \[
            \[
              \Delta \vdash (\forall x \in A)\psi (x) \quad 
              \Delta \vdash x \in A
              \justifies
              \Delta \vdash \psi (x)
            \]
            \justifies
            \Delta \equiv \Gamma, (\forall x \in A)\psi (x) \vdash \varphi \vee \psi (x)
          \]
          \justifies
          \Gamma \equiv \varphi \vee (\forall x \in A)\psi (x), x \in A \vdash \varphi \vee \psi (x)
        \]
        \justifies
        \varphi \vee (\forall x \in A)\psi (x) \vdash (\forall x \in A)(\varphi \vee \psi (x))
      \]
      \justifies
      \varphi \vee (\forall x \in A)\psi (x) \rightarrow (\forall x \in A)(\varphi \vee \psi (x))
    \end{prooftree}
  \]
\end{example}

\begin{example}
  We have seen that, in the system of natural deduction with implicit
  hypotheses, we must pay attention to the parameters involved in the rules of
  $\forall$ introduction and $\exists$ elimination in order to avoid logical
  absurdities to be derived. The alert reader may have noticed that there are no
  side conditions of this sort in the rules with explicit contexts. This is
  because the definition of context itself, which disallows the presence of
  variables with the same name, takes care of such situations. As an example,
  notice how the derivation of the invalid $\forall xy (P(x) \rightarrow P(y))$
  gets blocked:

  \[
    \begin{prooftree}
      \[
        \[
          \[ \justifies x \in A, y \in A, u : P(x) \vdash y \in A \] \quad 
          \[
            x \in A, y \in A, u : P(x), z \in A \vdash P(z)
            \justifies
            x \in A, y \in A, u : P(x) \vdash (\forall x \in A)P(x)
          \]
          \justifies
          x \in A, y \in A, u : P(x) \vdash P(y)
        \]
        \justifies
        x \in A, y \in A \vdash P(x) \rightarrow P(y)
      \]
      \justifies
      \vdash (\forall x \in A)(\forall y \in A)(P(x) \rightarrow P(y))
    \end{prooftree}
  \]

  Notice how there is no way to complete the right branch: we would be able to
  derive it only in the case $x = z$, but this is not possible by the definition
  of context, which requires the $\forall$ introduction rule to refer to a fresh
  variable (in this case, $z$).

  Consider another logical fallacy that can be derived if we ignore the variable
  conditions of natural deduction:

  \[
    \begin{prooftree}
      \[
        \[
          \[
            \justifies
            \exists x A
            \using{u}
          \] \quad 
          \[
            \justifies
            A[w/x]
            \using{v}
          \]
          \justifies
          A[w/x]
          \using{v}
        \]
        \justifies
        \forall x A
      \]
      \justifies
      \exists x A \rightarrow \forall x A
    \end{prooftree}
  \]

  This derivation is \emph{not} valid, since the parameter $w$ used in the
  discharged hypothesis of the existential elimination is free in the conclusion
  of the rule. Notice how this incorrect development of the proof gets blocked
  by the definition of context itself:

  \[
    \begin{prooftree}
      \[
        \[
          \[
            \justifies
            (\exists x \in A) \varphi(x), x \in A \vdash (\exists x \in A)
            \varphi(x)
          \]
          \quad 
          (\exists x \in A) \varphi(x), x \in A, y \in A, \varphi(y) \vdash \varphi(x)
          \justifies
          (\exists x \in A) \varphi(x), x \in A \vdash \varphi(x)
        \]
        \justifies
        (\exists x \in A) \varphi(x) \vdash (\forall x \in A)\varphi(x)
      \]
      \justifies
      \vdash (\exists x \in A) \varphi(x) \rightarrow (\forall x \in A)\varphi(x)
    \end{prooftree}
  \]

  The variable $y$ introduced by the existential elimination rule is forced to
  be different from $x$, which is already in context. Hence, there is no way to
  unify the hypothesis, which must be $\varphi(y)$, with the conclusion
  $\varphi(x)$.

\end{example}

\begin{example}
  A common misconception about intuitionistic logic in that it is in explicit
  opposition to classical logic, to the point that typically classical
  principles like LEM are provably false in it. This view is, of course, wrong,
  since intuitionistic logic and in general constructive mathematics in the
  Bishop sense are compatible with classical mathematics. In particular, observe
  that $\neg \neg (A \vee \neg A)$ is an intuitionistic tautology:

  \[
    \begin{prooftree}
      \[
        \[\justifies (A \vee \neg A) \rightarrow \bot \using{u}\] \quad 
        \[
          \[
            \[
              \[\justifies (A \vee \neg A) \rightarrow \bot \using{u}\] \quad 
              \[
                \[ \justifies A \using{w} \]
                \justifies
                A \vee \neg A
              \]
              \justifies
              \bot
            \]
            \justifies
            A \rightarrow \bot
            \using{w}
          \]
          \justifies
          A \vee \neg A
        \]
        \justifies
        \bot
      \]
      \justifies
      \neg \neg (A \vee \neg A)
    \end{prooftree}
  \]
  
\end{example}

\subsection{Proof terms}\label{proof-terms}

We now see the \emph{static} nature of the Curry-Howard isomorphism, that is the
correspondence between propositions (expressed as formulas) and types for
$\lambda$-terms. This correspondence is just another way to reveal the
connections (and differences) between language and meta-language: we
\emph{construct proofs} of a proposition $\varphi$ in the same way as we
\emph{derive} the \emph{judgement} $\truej{\varphi}$. The first is an
internalization in the object language of the second, which is a feature of the
meta-language.

According to the BHK interpretation of intuitionistic logic, a proof of a
proposition is an effective construction. Proof terms formalize this aspect,
showing not only that a proposition is true, but also giving a construction for
it. Writing $\tyj{p}{\varphi}$ for the judgement ``$p$ is a proof term of the
proposition $\varphi$'', we have thus the following equivlence

\[
  \truej{\varphi} \quad \text{iff} \quad
  \text{there exists } p \text{ such that } \tyj{p}{\varphi}
\]

Our proof terms need to be effective. Despite not formally binding ourselves to
recursive functions (CT\footnote{Formal Church's Thesis. For an overview on
  constructive axioms and their validity/consistency within constructive
  theories, see Troelstra, Metamathematical Investigation of Intuitionistic
  Arithmetic and Analysis, or Troelstra and van Dalen, Constructivism in
  Mathematics vol. 1.} is not axiomatized in most constructive theories,
although valid in some of their interpretations), we will construct them as
terms of a typed $\lambda$-calculus.

We extend our definition of contexts to allow assumptions about proofs of
propositions:

\[
  \begin{prooftree}
    \Gamma \vdash \propj{A}
    \justifies
    \ctxtj{\Gamma, \tyj{x}{A}}
    \using{x \text{ fresh}}
  \end{prooftree}
\]

We now extend our system of natural deduction with proof terms. We review the
meaning of logical connectives according to the BHK interpretation, and see how
to reflect it as proof terms.

\paragraph{Conjunction}

We know what counts as a proof of the proposition $\andconn{\varphi}{\psi}$,
namely a proof of $\varphi$ and a proof of $\psi$. It makes sense for a proof
term for the conjunction, therefore, to be a pair of proofs, one of $\varphi$
and one of $\psi$:

\[
  \begin{prooftree}
    \Gamma \vdash \tyj{M}{\varphi} \qquad \Gamma \vdash \tyj{N}{\psi}
    \justifies
    \Gamma \vdash \tyj{\pair{M}{N}}{\andconn{\varphi}{\psi}}
  \end{prooftree}
\]

Then, it seems reasonable to extract the first or second of this pair of proofs
by projection:

\[
  \begin{prooftree}
    \Gamma \vdash \tyj{M}{\andconn{\varphi}{\psi}}
    \justifies
    \Gamma \vdash \tyj{\snd{M}}{\psi}
  \end{prooftree}
  \qquad
  \begin{prooftree}
    \Gamma \vdash \tyj{M}{\andconn{\varphi}{\psi}}
    \justifies
    \Gamma \vdash \tyj{\fst{M}}{\varphi}
  \end{prooftree}
\]

\paragraph{Disjunction}

A proof of a disjunction $\orconn{\varphi}{\psi}$ is a proof of $\varphi$ or a
proof of $\psi$, together with the information telling us \emph{which one} of
the two. The proof term then is the proof itself, with \emph{tags} telling us
whether the proof refers to the first or the second disjunct, and what is the
disjunct about which we are not saying anything:

\[
  \begin{prooftree}
    \Gamma \vdash \tyj{M}{\varphi}
    \justifies
    \Gamma \vdash \tyj{\inleft{\psi}{M}}{\orconn{\varphi}{\psi}}
  \end{prooftree}
  \qquad
  \begin{prooftree}
    \Gamma \vdash \tyj{M}{\psi}
    \justifies
    \Gamma \vdash \tyj{\inright{\varphi}{M}}{\orconn{\varphi}{\psi}}
  \end{prooftree}
\]

A proof of a disjunction is used by \emph{reading} the information on the term
itself, and proving the conclusion by case analysis.

\[
  \begin{prooftree}
    \begin{array}{@{}c}
      \Gamma \vdash \propj{\chi}
      \qquad
      \Gamma \vdash \tyj{M}{\orconn{\varphi}{\psi}}
      \\
      \Gamma, \tyj{u}{\varphi} \vdash \tyj{d_1(u)}{\chi}
      \qquad
      \Gamma, \tyj{w}{\psi} \vdash \tyj{d_2(w)}{\chi}
    \end{array}
    \justifies
    \Gamma \vdash \tyj{\orelimop{M}{(u)d_1}{(w)d_2}}{\chi}
  \end{prooftree}
\]

Notice how the elimination specifically happens towards a single
\emph{proposition} $\chi$, rather than a set or a family of sets. This
happens for reasons that will be made clear when we introduce the
existential quantification, which can be seen as a potentially
infinite disjunction and thus a generalization of the binary case.

\paragraph{Implication}

A proof of an implication $\implconn{\varphi}{\psi}$ is understood to be a
method, or operation, that yields a proof of $\psi$ given a proof of $\varphi$.

\[
  \begin{prooftree}
    \Gamma, \tyj{x}{\varphi} \vdash \tyj{b(x)}{\psi}
    \justifies
    \Gamma \vdash \tyj{\implctor{x}{b(x)}}{\implconn{\varphi}{\psi}}
  \end{prooftree}
  \qquad
  \begin{prooftree}
    \Gamma \vdash \tyj{M}{\implconn{\varphi}{\psi}}
    \qquad
    \Gamma \vdash \tyj{N}{\varphi}
    \justifies
    \Gamma \vdash \tyj{\implelimop{M}{N}}{\psi}
  \end{prooftree}
\]

\paragraph{Universal quantification}

A similar situation is with universal quantification: a proof of
$\forallconn{x}{A}{\varphi(x)}$ is an operation that yields a proof of
$\varphi(x)$ given an element $x$ of the set $A$.

\[
  \begin{prooftree}
    \Gamma, x \in A \vdash \propj{\varphi(x)}
    \qquad
    \Gamma, x \in A \vdash \tyj{b(x)}{\varphi (x)}
    % \Gamma \vdash \tyj{M}{A[w/x]}
    \justifies
    \Gamma \vdash \tyj{\forallctor{w}{b(x)}}{\forallconn{x}{A}{\varphi(x)}}
  \end{prooftree}
\]

To eliminate a universal quantification, it sufficies to provide an
element of the set of quantification:

\[
  \begin{prooftree}
    \Gamma \vdash \inj{t}{A}
    \qquad
    \Gamma \vdash \tyj{f}{\forallconn{x}{A}{\varphi(x)}}
    \justifies
    \Gamma \vdash \tyj{\forallelimop{f}{t}}{\varphi(t)}
  \end{prooftree}
\]

\paragraph{Existential quantification}

The rules of $\exists$ decorated with proof terms validate the usual
interpretation of the intuitionistic existential quantifier as a ``package''
containing an element of the set of quantification (a ``witness'') together with
a proof that the predicate holds for the witness.

\[
  \begin{prooftree}
    \Gamma, \inj{x}{A} \vdash \propj{\varphi(x)}
    \quad
    \Gamma \vdash \inj{t}{A}
    \quad
    \Gamma \vdash \tyj{p}{\varphi(t)}
    \justifies
    \Gamma \vdash \tyj{\pair{t}{p}}{\existsconn{x}{A}{\varphi(x)}}
  \end{prooftree}
\]

Notice that, as for disjunction, the elimination rule of the existential
quantifier eliminates only towards plain propositions without any form of
dependency.

\[
  \begin{prooftree}
    \Gamma \vdash \propj{\psi}
    \qquad
    \Gamma \vdash \tyj{M}{\existsconn{x}{A}{\varphi(x)}}
    \qquad
    \Gamma, x \in A, \tyj{y}{\varphi(x)} \vdash \tyj{N}{\psi}
    \justifies
    \Gamma \vdash \tyj{\existselimop{M}{(y)N}}{\psi}
  \end{prooftree}
\]

This is the main technical difference between the existential quantifier and the
$\Sigma$ set constructor (which coincide in their introduction rules). Indeed,
one of the reasons to keep propositions and sets formally distinct is to be able
to give a weaker notion of existential quantification than the one that comes
from the identification of propositions and sets/types as in M-L, and in
particular of the existential quantifier and the strong indexed sum set.

Notice in fact that, interpreting $\forall$ and $\exists$ as the
set-constructors $\Pi$ and $\Sigma$, one validates the axiom
of choice, identifying functions (total set-valued relations) and operations
(effective total procedures).

\[
  \implconn{
    \forallconn{x}{A}{\existsconn{y}{B}{R(x,y)}}
  }{
    \existsconn{f}{A \rightarrow B}{\forallconn{x}{A}{R(x, f(x))}}
  }
\]

Distinguishing between propositions and sets allows us to define a weaker notion
of existential elimination, which blocks the validity of the axiom of choice
while still modelling the usual intuitionistic logic. The details of all this
will be made precise when we talk about the stratification of the theory in two
levels of abstraction, and the need for compatibility of these levels with
existing classical and constructive theories.

On a final note, observe how the premise $\Gamma \vdash \propj{\psi}$ forces
$\psi$ to \emph{not} depend on the parameter $x$ used in the second
premise. This is to say that the rule automatically satisfies the side-condition
of the usual natural deduction system that the parameter be not free in the
conclusion.

\paragraph{True and false}

The proposition $\top$ is trivially true, therefore we fix a constant symbol
$\star$ that denotes its unique, canonical proof.

\[
  \begin{prooftree}
    \justifies
    \Gamma \vdash \tyj{\star}{\trueconn}
  \end{prooftree}
\]

The false proposition $\bot$ has no proof, hence, by \emph{ex falso quodlibet},
if we are presented with a proof of it we might as well deduce anything.

\[
  \begin{prooftree}
    \Gamma \vdash \tyj{M}{\falseconn}
    \justifies
    \Gamma \vdash \tyj{\falseelimop{A}{M}}{A}
  \end{prooftree}
\]

Figure~\ref{natded-proofterms} show the complete formal system, where the
obvious ``prop'' judgements are left implicit.

\begin{figure}[ht]
  \begin{mdframed}
    \[
      \begin{prooftree}
        \Gamma \vdash \tyj{p}{\varphi} \qquad \Gamma \vdash \tyj{q}{\psi}
        \justifies
        \Gamma \vdash \tyj{\pair{p}{q}}{\andconn{\varphi}{\psi}}
      \end{prooftree}
      \qquad
      \begin{prooftree}
        \Gamma \vdash \tyj{p}{\andconn{\varphi}{\psi}}
        \justifies
        \Gamma \vdash \tyj{\fst{p}}{\varphi}
      \end{prooftree}
      \qquad
      \begin{prooftree}
        \Gamma \vdash \tyj{p}{\andconn{\varphi}{\psi}}
        \justifies
        \Gamma \vdash \tyj{\snd{p}}{\psi}
      \end{prooftree}
    \]
    \[
      \begin{prooftree}
        \Gamma \vdash \tyj{p}{\varphi}
        \justifies
        \Gamma \vdash \tyj{\inleft{\psi}{p}}{\orconn{\varphi}{\psi}}
      \end{prooftree}
      \qquad
      \begin{prooftree}
        \Gamma \vdash \tyj{p}{\psi}
        \justifies
        \Gamma \vdash \tyj{\inright{\varphi}{p}}{\orconn{\varphi}{\psi}}
      \end{prooftree}
    \]
    \[
      \begin{prooftree}
        \begin{array}{@{}c}
          \Gamma \vdash \propj{\phi}
          \qquad
          \Gamma \vdash \tyj{p}{\orconn{\varphi}{\psi}}
          \\
          \Gamma, \tyj{x}{\varphi} \vdash \tyj{d_1(x)}{\phi}
          \qquad
          \Gamma, \tyj{x}{\psi} \vdash \tyj{d_2(x)}{\phi}
        \end{array}
        \justifies
        \Gamma \vdash \tyj{\orelimop{p}{(x)d_1}{(x)d_2}}{\phi}
        \using{\orelimrule}
      \end{prooftree}
    \]
    \[
      \begin{prooftree}
        \Gamma, \tyj{x}{\varphi} \vdash \tyj{b(x)}{\psi}
        \justifies
        \Gamma \vdash \tyj{\implctor{x}{b(x)}}{\implconn{\varphi}{\psi}}
      \end{prooftree}
      \qquad
      \begin{prooftree}
        \Gamma \vdash \tyj{p}{\implconn{\varphi}{\psi}}
        \qquad
        \Gamma \vdash \tyj{q}{\varphi}
        \justifies
        \Gamma \vdash \tyj{\implelimop{p}{q}}{\psi}
      \end{prooftree}
    \]
    \[
      \begin{prooftree}
        \begin{array}{@{}c}
          \Gamma, x \in A \vdash \propj{\varphi(x)} \\
          \Gamma, x \in A \vdash \tyj{b(x)}{\varphi (x)}  
        \end{array}
        \justifies
        \Gamma \vdash \tyj{\forallctor{w}{b(x)}}{\forallconn{x}{A}{\varphi(x)}}
      \end{prooftree}
      \qquad
      \begin{prooftree}
        \begin{array}{@{}c}
          \Gamma, \inj{x}{A} \vdash \propj{\varphi(x)} \\
          \Gamma \vdash \inj{t}{A}
          \qquad
          \Gamma \vdash \tyj{p}{\forallconn{x}{A}{\varphi(x)}}
        \end{array}
        \justifies
        \Gamma \vdash \tyj{\forallelimop{p}{t}}{\varphi(t)}
      \end{prooftree}
    \]
    \[
      \begin{prooftree}
        \begin{array}{@{}c}
          \Gamma, \inj{x}{A} \vdash \propj{\varphi(x)} \\  
          \Gamma \vdash t \in A
          \quad
          \Gamma \vdash \tyj{p}{\varphi(t)}
          \end{array}
        \justifies
        \Gamma \vdash \tyj{\pair{t}{p}}{\existsconn{x}{A}{\varphi(x)}}
      \end{prooftree}
      \qquad
      \begin{prooftree}
        \begin{array}{@{}c}
          \Gamma \vdash \propj{\psi} \\
          \Gamma \vdash \tyj{p}{\existsconn{x}{A}{\varphi(x)}}\\
          \Gamma, x \in A, \tyj{y}{\varphi(x)} \vdash \tyj{q}{\psi}
        \end{array}
        \justifies
        \Gamma \vdash \tyj{\existselimop{p}{(y)q}}{\psi}
      \end{prooftree}
    \]
  \end{mdframed}
  \caption{\label{natded-proofterms} Natural deduction with proof terms.}
\end{figure}

\begin{example}

  Below is a derivation of the intuitionistically valid proposition

  \[
    (\psi \rightarrow \chi) \rightarrow (\varphi \rightarrow \psi)
    \rightarrow \varphi \rightarrow \chi
  \]

  \[
    \begin{prooftree}
      \[
        \[
          \[
            f \in ..., ... \vdash f \in ... \quad 
            \[
              ... \vdash g \in \varphi \rightarrow \psi \quad 
              ... \vdash x \in \varphi
              \justifies
              ... \vdash Ap(g,x) \in \psi
            \]
            \justifies
            f \in ..., g \in ..., x \in \varphi \vdash Ap(f, Ap(g,x)) \in \chi
          \]
          \justifies
          f \in ..., g \in \varphi \rightarrow \psi \vdash \lambda x . Ap(f, Ap(g,x)) \in \varphi \rightarrow \chi
        \]
        \justifies
        f \in \psi \rightarrow \chi \vdash \lambda g x . Ap(f, Ap(g,x)) \in (\varphi \rightarrow \psi) \rightarrow \varphi \rightarrow \chi
      \]
      \justifies
      \vdash \lambda f g x . Ap(f, Ap(g,x)) \in (\psi \rightarrow \chi) \rightarrow (\varphi \rightarrow \psi)
    \rightarrow \varphi \rightarrow \chi
    \end{prooftree}
  \]

  The proof term of this proposition corresponds to the $B$ combinator in
  combinatory logic, which performs function composition. In fact, notice how

  \[
    B f g x = f(g(x))
  \]

\end{example}

\begin{example}

  Below is a derivation with proof terms of the intuitionistically valid
  proposition

  \[
    (\exists x \in A)\varphi(x) \rightarrow
    \neg (\forall x \in A)\neg \varphi(x)
  \]
  
  \[
    \begin{prooftree}
      \[
        \[
          e \in ..., ... \vdash e \in ... \quad 
          \[
            \[
              e \in ..., f \in ... \vdash f \in ... \quad 
              x \in A, ... \vdash x \in A
              \justifies
              ... \vdash Ap(f,x) \in \varphi(x) \rightarrow \bot
            \] \quad 
            ..., y \in \varphi(x) \vdash y \in \varphi(x)
            \justifies
            e \in ..., f \in ..., x \in A, y \in \varphi(x) \vdash Ap(Ap(f,x),y) \in \bot
          \]
          \justifies
          e \in ..., f \in (\forall x \in A)(\varphi(x) \rightarrow \bot) \vdash E(e, (x,y)Ap(Ap(f,x),y)) \in \bot
        \]
        \justifies
        e \in (\exists x \in A)\varphi(x) \vdash \lambda f . E(e, (x,y)Ap(Ap(f,x),y)) \in [(\forall x \in A)(\varphi(x) \rightarrow \bot)] \rightarrow \bot
      \]
      \justifies
      \vdash \lambda e f . E(e, (x,y)Ap(Ap(f,x),y)) \in (\exists x \in A)\varphi(x) \rightarrow
    \neg (\forall x \in A)\neg \varphi(x)
    \end{prooftree}
  \]
\end{example}

\begin{example}

  Below is a derivation with proof terms of the intuitionistically valid
  proposition

  \[
    (\forall x \in A)(\varphi(x) \rightarrow \psi(x))
    \rightarrow
    ((\exists x \in A)\varphi(x) \rightarrow (\exists x \in A)\psi(x))
  \]

  \[
    \begin{prooftree}
      \[
    \[
      ..., e \in ... \vdash e \in ...
      \[
        \[
          \[
            ..., f \in ..., ... \vdash f \in ...
            \quad
            ..., x \in A, ... \vdash x \in A
          \justifies
            ..., f \in ..., ... \vdash Ap(f,x) \in \varphi(x) \rightarrow \psi(x)
          \]
          \quad ..., y \in \varphi(x) \vdash y \in \varphi(x)
        \justifies
          ..., x \in A, y \in \varphi(x) \vdash Ap(Ap(f,x),y) \in \psi(x)
        \]
        ..., x \in A, ... \vdash x \in A
        \justifies
        ..., x \in A, y \in \varphi(x) \vdash <x, Ap(Ap(f,x),y)> \in (\exists x \in A)\psi(x)
      \]
    \justifies
      f \in ..., e \in (\exists x \in A)\varphi(x) \vdash E(e, (x,y)<x, Ap(Ap(f,x),y)>) \in (\exists x \in A)\psi(x)
    \]
  \justifies
    f \in (\forall x \in A)(\varphi(x) \rightarrow \psi(x)) \vdash \lambda e . E(e, (x,y)<..>) \in (\exists \rightarrow \exists)
  \]
\justifies
  \vdash \lambda f e . E(...) \in (\forall x \in A)(\varphi(x) \rightarrow \psi(x))
    \rightarrow
    ((\exists x \in A)\varphi(x) \rightarrow (\exists x \in A)\psi(x))
    \end{prooftree}
  \]
\end{example}

\subsection{Proof normalization}


Out treatise of proof normalization starts with the observation that an
elimination rule in our system is, in a sense, the inverse of the corresponding
introduction rule: by an application of an elimination rule one essentially only
restores what had already been established if the major premise of the
application was inferred by an application of an introduction rule. The essence
is that in such situations, the pair of applications of the introduction rule
followed by the elimination rule can be eliminated, because the deduction of the
conclusion can be obtained from their premises. Prawitz calls this the
\emph{inversion principle} in \cite{prawitz1965}. The same idea is expressed by
Pfenning in \cite{pfenninglecturescl}, by saying that the elimination rules of
natural deduction are \emph{not too strong}.

\subsubsection{Digression on substitution rules}

To illustrate proof transformations, we are going to need two essential
structural rules to deal with substitutions. Consider a derivation $\mathcal{D}$
depending on some variable $x$ of type $A$. In our system, we depict this as

\[
  \begin{array}{@{}c}
    \mathcal{D} \\
    \Gamma, \inj{x}{A}, \Gamma'(x) \vdash \varphi(x)
  \end{array}
\]

It seems reasonable to consider a derivation $\mathcal{D}'$ which is just
$\mathcal{D}$ with some term $\inj{t}{A}$ in place of $x$, and expect it to be
derivable as well.

\[
  \begin{array}{@{}c}
    \mathcal{D}' \\
    \Gamma, \Gamma'(t) \vdash \varphi(t)
  \end{array}
\]

\begin{lemma}
  The \emph{substitution rule}

  \[
    \begin{prooftree}
      \Gamma, \inj{x}{A}, \Gamma'(x) \vdash \varphi(x)
      \qquad
      \inj{t}{A}
      \justifies
      \Gamma, \Gamma'(t) \vdash \varphi(t)
      \using{sost}
    \end{prooftree}
  \]

  is admissible.
\end{lemma}
\begin{proof}
  Left as exercise.
\end{proof}

The same goes for derivations. Suppose we have a derivation of $\psi$ on the
basis of some assumptions $\Gamma$ and an assumption for $\varphi$, and suppose
we were also able to show $\varphi$ from the same assumptions $\Gamma$.

\[
  \Gamma, \varphi \vdash \psi \qquad \qquad \Gamma \vdash \varphi
\]

Then, we should be able to derive $\psi$ from $\Gamma$ alone, since the other
assumption $\varphi$ is already a consequence of $\Gamma$.

\begin{lemma}
  The \emph{cut rule}

  \[
    \begin{prooftree}
      \Gamma, \varphi \vdash \psi
      \qquad
      \Gamma \vdash \varphi
      \justifies
      \Gamma \vdash \psi
    \end{prooftree}
  \]

  is admissible.
\end{lemma}
\begin{proof}
  Direct consequence of cut elimination for sequent calculus.
\end{proof}

\subsubsection{Detour conversions}

Consider the following derivation:

\[
  \begin{prooftree}
    \[
      \begin{array}{@{}c}
        \mathcal{D} \\
        \Gamma, A \vdash B  
      \end{array}
      \justifies
      \Gamma \vdash \implconn{A}{B}
    \] \quad
    \begin{array}{@{}c}
      \mathcal{D'}
      \Gamma \vdash A
    \end{array}
    \justifies
    \Gamma \vdash B
  \end{prooftree}
\]

It is quite clear that introducing $\implconn{A}{B}$ and then immediately
eliminating it is a redundancy. Such a detour could have been avoided by simply
replacing the assumption for $A$ in $\mathcal{D}$ with the actual derivation in
$\mathcal{D'}$. More formally:

\[
  \begin{prooftree}
    \begin{array}{@{}c}
      \mathcal{D'} \\
      \Gamma \vdash A
    \end{array}
    \quad
    \begin{array}{@{}c}
      \mathcal{D} \\
      \Gamma, A \vdash B  
    \end{array}
    \justifies
    \Gamma \vdash B
  \end{prooftree}
\]

It turns out that such redundant applications of introduction rule followed by
elimination rule with the formula just introduced as major premise, also called
\emph{detours}, can be removed from derivations in a systematic way. We show how
to perform such \emph{local reductions} for each connective:

\paragraph{Conjunction}

A \emph{detour} on a conjunction simply reduces to either one of the premises of
the introduction rule, depending on the formula that is the conclusion of the
elimination rule.

\[
  \begin{prooftree}
    \[
      \ndderives{\mathcal{D}}{\Gamma \vdash A}
      \qquad
      \ndderives{\mathcal{E}}{\Gamma \vdash B}
      \justifies
      \Gamma \vdash A \wedge B
    \]
    \justifies
    \Gamma \vdash A
  \end{prooftree}
  \qquad \leadsto \qquad
  \ndderives{\mathcal{D}}{\Gamma \vdash A}
\]

and similarly

\[
  \begin{prooftree}
    \[
      \ndderives{\mathcal{D}}{\Gamma \vdash A}
      \qquad
      \ndderives{\mathcal{E}}{\Gamma \vdash B}
      \justifies
      \Gamma \vdash A \wedge B
    \]
    \justifies
    \Gamma \vdash B
  \end{prooftree}
  \qquad \leadsto \qquad
  \ndderives{\mathcal{E}}{\Gamma \vdash B}
\]

\paragraph{Disjunction}

An elimination of a disjunction contains proofs of the conclusion for the two
possible cases, since it is not known \emph{a priori} which of the two disjoints
holds. If we have a derivation $\mathcal{D}$ of the formula that introduced the
disjunction, all information is already in our hands, so we can just take the
corresponding proof of the conclusion and compose it with $\mathcal{D}$.

\[
  \begin{prooftree}
    \[
      \ndderives{\mathcal{D}}{\Gamma \vdash A}
      \justifies
      \Gamma \vdash A \vee B
    \]
    \qquad
    \derives{\mathcal{E}}{\Gamma, A \vdash C}
    \qquad \Gamma, B \vdash C
    \justifies
    \Gamma \vdash C
  \end{prooftree} \qquad \leadsto \qquad
  \begin{prooftree}
    \ndderives{\mathcal{D}}{\Gamma \vdash A}
    \qquad
    \ndderives{\mathcal{E}}{\Gamma, A \vdash C}
    \justifies
    \Gamma \vdash C
  \end{prooftree}
\]

and similarly

\[
  \begin{prooftree}
    \[
      \ndderives{\mathcal{D}}{\Gamma \vdash B}
      \justifies
      \Gamma \vdash A \vee B
    \]
    \qquad
    \Gamma, A \vdash C
    \qquad
    \derives{\mathcal{E}}{\Gamma, B \vdash C}
    \justifies
    \Gamma \vdash C
  \end{prooftree} \qquad \leadsto \qquad
  \begin{prooftree}
    \ndderives{\mathcal{D}}{\Gamma \vdash B}
    \qquad
    \ndderives{\mathcal{E}}{\Gamma, B \vdash C}
    \justifies
    \Gamma \vdash C
  \end{prooftree}
\]

\paragraph{Implication}

In order to eliminate an implication $\implconn{\varphi}{\psi}$ to derive
$\psi$, we must possess a proof of the antecedent $\varphi$. If the implication
has been derived by introduction, that means we were able to show a derivation
$\mathcal{D}$ of the conclusion $\psi$ from the assumption $\varphi$. But then,
we might just compose $\mathcal{D}$ with the derivation of $\varphi$ that we
already have:

\[
  \begin{prooftree}
    \[
      \ndderives{\mathcal{D}}{\Gamma, A \vdash B}
      \justifies
      \Gamma \vdash A \supset B
    \]
    \qquad
    \derives{\mathcal{E}}{\Gamma \vdash A}
    \justifies
    \Gamma \vdash B
  \end{prooftree}\qquad \leadsto \qquad
  \begin{prooftree}
    \ndderives{\mathcal{D}}{\Gamma, A \vdash B}
    \qquad
    \ndderives{\mathcal{E}}{\Gamma \vdash A}
    \justifies
    \Gamma \vdash B
  \end{prooftree}
\]

\paragraph{Universal quantification}

A detour involving a universal quantification is a proof of $\varphi(t)$ for
some term $\inj{t}{A}$ and predicate $\varphi$ on $A$, where the eliminated
universal quantification has been established by a previous derivation of
$\varphi(w)$ for a parameter $w$ that is free in all undischarged
assumptions. It follows that we can just substitute $t$ for $w$ in the
derivation of $\varphi(w)$, and get to the same conclusion.

\[
  \begin{prooftree}
    \[
      \ndderives{\mathcal{D}}{\Gamma, \inj{w}{A} \vdash \varphi(w)}
      \justifies
      \Gamma \vdash \forallconn{x}{A}{\varphi(x)}
    \]
    \qquad
    \derives{\mathcal{E}}{\Gamma \vdash \inj{t}{A}}
    \justifies
    \Gamma \vdash \varphi(t)
  \end{prooftree} \qquad \leadsto \qquad
  \begin{prooftree}
    \ndderives{\mathcal{D}}{\Gamma, \inj{w}{A} \vdash \varphi(w)}
    \quad
    \ndderives{\mathcal{E}}{\Gamma \vdash \inj{t}{A}}
    \justifies
    \Gamma \vdash \varphi(w)[t/w]
    \using{sost}
  \end{prooftree}
\]

where $\varphi(w)[t/w] \equiv \varphi(t)$, which gives us the conclusion. This
result is actually quite obvious: as the universal quantifier is an
internalization of a meta-level universal quantification, it does not really
matter if we instantiate $\varphi$ on the basis on the object- or meta- level
quantification.

\paragraph{Existential quantification}

We prove a proposition $\psi$ from $\existsconn{x}{A}{\varphi(x)}$ if we can
show a derivation $\mathcal{E}$ for $\psi$ from the assumption that $\varphi(x)$
holds for a fresh parameter $x$. That is, $\mathcal{E}$ is parametric in $x$. It
follows that replacing $x$ for another term $t$ in $\mathcal{E}$ does not change
its ability to show $\psi$ true. Now, if the existential has been derived by
introduction, we possess a derivation of $\varphi(t)$ for some term
$\inj{t}{A}$. We can therefore just compose this derivation with $\mathcal{E}$,
to get a derivation of the conclusion without the detour.

\[
  \begin{prooftree}
    \[
      \ndderives{\mathcal{D}}{\Gamma \vdash A(t)}
      \qquad
      \ndderives{\mathcal{D'}}{\Gamma \vdash \inj{t}{A}}
      \justifies
      \Gamma \vdash \existsconn{x}{A}{\varphi(x)}
    \]
    \qquad
    \derives{\mathcal{E}}{\Gamma, \inj{w}{A}, \varphi(w) \vdash \psi}
    \justifies
    \Gamma \vdash B
  \end{prooftree}
\]
\[
  \qquad \leadsto \qquad
  \begin{prooftree}
    \derives{\mathcal{D}}{\Gamma \vdash A(t)}
    \qquad
    \[
      \ndderives{\mathcal{D'}}{\Gamma \vdash \inj{t}{A}}
      \qquad
      \ndderives{\mathcal{E}}{\Gamma, \inj{w}{A}, \varphi(w) \vdash \psi}
      \justifies
      \Gamma, \varphi(t) \vdash \psi[t/w] \equiv \psi
      \using{sost}
    \]
    \justifies
    \Gamma \vdash \psi
  \end{prooftree}
\]

This leads us to a \footnote{we say ``a'' instead of ``the'' because most
  proof-theoretical studies on the normalization of natural deduction consider
  proof reductions that eliminate additional forms of redundancy than that of
  detours. For our purposes, however, it sufficies to consider detour-free
  derivations. For details we refer to the bibliography on the topic, which is
  pointed out at the end of the section.} definition of normal deduction.

\begin{definition}
  A \emph{normal deduction} is a deduction in which no \emph{detours} occur.
\end{definition}

The following important theorem holds.

\begin{theorem}[Normalization theorem]
  If $\Gamma \vdash \varphi$ holds in the system of intuitionistic natural
  deduction, then there is a normal deduction of $\varphi$ from $\Gamma$ in this
  system.
\end{theorem}
\begin{proof}
  By adapting the proof of Theorem 1 in \cite{prawitz1965}, which is a stronger
  statement.
\end{proof}

The above theorem states that every natural deduction proof can be reduced to a
normal proof. This property is also called \emph{weak normalization}. Actually,
the calculus can be proved to have a stronger property, that of \emph{strong
  normalization}: not only there exists a particular way to normalize a proof,
but \emph{every} possible normalization process, or sequence of reduction steps,
terminates to the \emph{same} normal proof.

\subsubsection{Form of normal deductions}

We will now prove a theorem that holds for closed normal deductions, which will
be essential to justify the meaning of proof terms for natural deduction given
in Section~\ref{proof-terms}.

\begin{definition}
  A \emph{main branch} of a derivation is a branch which starts from a top node
  (a leaf of the derivation, if seen as a tree), ends in the conclusion, and in
  which nodes (apart from the last) are either premises of introduction rules or
  major premises of elimination rules.
\end{definition}

\begin{proposition}
  Every deduction has a main branch.
\end{proposition}
\begin{proof}
  By straightforward induction on the height of the derivation. Here are some
  representative cases:

  \begin{enumerate}
  \item The base case in the derivation which assumes $\varphi$ as hypothesis
    and asserts it. In this case, the main branch in trivially the formula
    $\varphi$ itself;
  \item The last rule is $\andintrorule$:

    \[
      \begin{prooftree}
        \begin{array}{@{}c}
          \mathcal{D} \\
          \varphi
        \end{array}
        \quad
        \begin{array}{@{}c}
          \mathcal{E} \\
          \psi
        \end{array}
        \justifies
        \andconn{\varphi}{\psi}
      \end{prooftree}
    \]

    Then, both $\mathcal{D}$ and $\mathcal{E}$ have main branches $\pi_1$ and
    $\pi_2$ respectively. The main branch for the whole derivation is
    constructed by appending $\andconn{\varphi}{\psi}$ to either $\pi_1$ or
    $\pi_2$.
    
  \item The last rule is $\implelimrule$:

    \[
      \begin{prooftree}
        \begin{array}{@{}c}
          \mathcal{D} \\
          \implconn{\varphi}{\psi}
        \end{array}
        \quad
        \begin{array}{@{}c}
          \mathcal{E} \\
          \varphi
        \end{array}
        \justifies
        \psi
      \end{prooftree}
    \]

    Here, again, both $\mathcal{D}$ and $\mathcal{E}$ have main branches $\pi_1$
    and $\pi_2$ respectively, but a main branch for the whole derivation can
    only be constructed by appending $\psi$ to $\pi_1$.
  \end{enumerate}
\end{proof}

\begin{lemma}\label{elimintrolemma}
  Let $\mathcal{D}$ be a detour-free deduction. Then, every main branch is
  formed by a sequence of elimination rules followed by a sequence of
  introduction rules.
\end{lemma}
\begin{proof}
  Suppose there exists a main branch of $\mathcal{D}$ violating the above
  condition. Then, we can find a sequence of introduction rules preceeding an
  elimination rule. The conclusion of the last of such introduction rules
  constitutes the major premise of the following elimination rule, by definition
  of main branch, and therefore it forms a detour. But this is in contradiction
  with the hypothesis that $\mathcal{D}$ is detour-free.
  \footnote{
    The reasoning of this proof may seem rather unconstructive given the usage
    of classical devices such as $\implconn{\neg \neg \varphi}{\varphi}$. We
    recall, however, that the proof is about properties of finite structures
    (derivation trees), which are decidable. Classical reasoning is
    constructively admissible when only decidable predicates are involved.}
\end{proof}

\begin{lemma}\label{mainbranchassumptionlemma}
  Consider a derivation $\mathcal{D}$ of $\varphi$ from assumptions $\Gamma$,
  that is, $\Gamma \vdash \varphi$. Then, if $\pi$ is a main branch of
  $\mathcal{D}$ involving only major premises of elimination rules, it
  starts from a hypothesis in $\Gamma$.
\end{lemma}
\begin{proof}
  It sufficies to observe that no elimination rule discharges assumptions in the
  branch of the major premise. The thesis follows immediately.
\end{proof}

\begin{theorem}\label{canonicalproofthrm}
  Let $\mathcal{D}$ be a closed detour-free derivation of $\varphi$, that is a
  normal derivation of $\vdash \varphi$. Then, the conclusion of $\mathcal{D}$
  has been derived by an introduction rule.
\end{theorem}
\begin{proof}
  Suppose $\mathcal{D}$ begins with an elimination rule. Then, by
  Lemma~\ref{elimintrolemma}, all main branches of the major premise of the
  elimination rule, and by definition of main branch, all main branches of
  $\mathcal{D}$ involve exclusively major premises of elimination rules. From
  Lemma~\ref{mainbranchassumptionlemma}, all such main branches start from an
  undischarged assumption of $\mathcal{D}$. But by hypothesis, there are no
  assumptions.
\end{proof}

Theorem~\ref{canonicalproofthrm}, together with the normalization theorem,
essentially says that every closed proof of a proposition $\varphi$ can be
mechanically reduced in a finite amount of time to a \emph{canonical proof},
that is, one in which $\varphi$ is proved by the introduction rule of its
outermost connective. Notice that a byproduct of this result is the consistency
of the logical calculus:

\begin{theorem}
  Natural deduction for intuitionistic predicate logic is consistent, that is,
  it cannot prove $\bot$.
\end{theorem}
\begin{proof}
  Suppose there is a proof of $\bot$. Then, there is a normal proof, which has
  an introduction rule as last rule. But no rule introduces $\bot$.
\end{proof}

\subsection{Normalization as computation}

%%% Local Variables:
%%% mode: latex
%%% TeX-master: "../notes"
%%% End:
