\section{Proof terms and the Curry-Howard isomorphism}

\subsection{Historical notes (WIP)}

The term ``Curry-Howard isomorphism'', also stated as the slogans ``propositions
as types'' or ``proofs as programs'', originally refers to a strong
correspondence between formal systems of constructive logic and typed lambda
calculi, and in a more general and modern view, between proof theory and
theoretical computer science.

The BHK interpretation of intuitionistic logic was already a hint towards this
idea (although the formulas-as-types paradigm is only one of the possible
incarnations of BHK, among others such as Kleene's realizability semantics).
The first explicit statement was made by Curry in the 40s, and the
correspondence was made precise for typed combinatory logic by Curry and Feys in
their milestone monograph \cite{curry1958combinatory}, in the form of theorems
giving a connection between provability and type inhabitation.

The works of Curry, however, accounts only for one aspect of the isomorphism,
namely the \emph{static} correspondence between propositions and types, as well
as proof terms and $\lambda$-terms (or combinators). The other important aspect
is \emph{dynamic}, and relates proof normalization in natural deduction and term
reduction (computation) in typed lambda calculi. W. W. Tait is often credited
for the discovery of this dynamic aspect of the correspondence, but an explicit
statement of it did not occur until 1969, when the famous paper by Howard began
to be privately circulated (it was actually published only in 1980). In
\cite{howard:tfatnoc}, explicit emphasis on the relationship between reduction
and normalization is made.

Despite the isomorphism being associated with Curry and Howard, the work by
N. G. De Bruijn on its tool Automath \cite{aut-001-1} represented a huge
contribution, and was the first example of making an actual practical use of the
ideas underlying the isomorphism. Automath was developed as a proof assistant
for writing mathematical proofs so that they could be verified by a
computer. Many fundamental concepts were for the first time used here, such as
dependent types, with implication defined as a special case of the universal
quantifier.

The ultimate development of the isomorphism came with the work of Per
Martin-Loef and its Intuitionistic Type Theory: the correspondence between
propositions and types is brought to an extreme, to the point where propositions
are \emph{identified} with types. In the following years, type theories, proof
assistants and programming languages have been created following the idea of the
Curry-Howard isomorphism, and inspired especially by Martin-Loef’s work. Notable
examples are Coquand and Huet’s Calculus of Constructions \cite{COQUAND198895},
the proof assistants Coq and Lego, and the dependently typed functional
programming languages Agda \cite{Norell:2009:DTP:1481861.1481862} and Idris.

From the point of view of categorical semantics, we discover yet another side of
the same correspondence. Objects and morphisms correspond to types and terms,
and also to formulas and proofs. This analogy was first discovered by Lawvere
and Lambek, to that the correspondence is sometimes stated as the
Curry-Howard-Lambek isomorphism. TODO: computational trinitarianism.


\subsection{Multi-sorted natural deduction with explicit context}

We now introduce explicit contexts, and see how they can be defined. In
particular, we are going to introduce a new judgement

\[
  \gammactxt
\]

and explain how it can be derived:

\[
  \ctxtj{[]}
  \qquad
  \begin{prooftree}
    \Gamma \vdash \propj{\varphi}
    \justifies
    \ctxtj{\Gamma, \tyj{u}{\varphi}}
    \using{u \text{ fresh}}
  \end{prooftree}
  \qquad
  \begin{prooftree}
    \Gamma \vdash \setj{A}
    \justifies
    \ctxtj{\Gamma, \inj{x}{A}}
    \using{x \text{ fresh}}
  \end{prooftree}
\]

that is, our contexts may contain hypotheses on the existence of elements of
sets and proofs of propositions, must the \emph{name} or \emph{tag} of each
hypothesis must be unique in the context. This restriction ensures that when an
assumption is removed from the context, it gets actually \emph{discharged} in
the sense that it cannot be used further in the derivation.  The most basic use
of an assumption is to assert it: if we assume $x$ is a proof of a proposition
$A$, then we are entitled to assert that $A$ is true, hence

\[
  \begin{prooftree}
    \ctxtj{\Gamma, \tyj{x}{\varphi}, \Gamma'}
    \justifies
    \Gamma, \tyj{x}{\varphi}, \Gamma' \vdash \truej{\varphi}
  \end{prooftree}
\]

In the same way, if we assume that $\inj{x}{A}$, we may as well assert it:

\[
  \begin{prooftree}
    \ctxtj{\Gamma, \inj{x}{A}, \Gamma'}
    \justifies
    \Gamma, \inj{x}{A}, \Gamma' \vdash \inj{x}{A}
  \end{prooftree}
\]

Notice that we implicitly assume that the context formation jusgements are
extended with their hypothetical forms, for example

\[
  \begin{prooftree}
    \Gamma, \inj{x}{A}, \Gamma' \vdash \propj{\varphi(x)}
    \justifies
    \ctxtj{\Gamma, \inj{x}{A}, \Gamma', \tyj{u}{\varphi(x)}}
  \end{prooftree}
\]

is a perfectly valid derivation, provided the variable $u$ did not already occur
in the context. Notice, also, that from our definition it follows
that in a context $\Gamma, \inj{x}{A}, \Gamma'$, $\Gamma$ neither contains
$\inj{x}{A}$, nor it contains $x$ free. This property will come in handy when we
introduce substitution in contexts.
The complete natural deduction system for multi-sorted first-order
intuitionistic logic with explicit contexts is shown in
Figure~\ref{natded-explicit-context}.

\begin{figure}[ht]
\begin{mdframed}

  \[
    \begin{prooftree}
      \Gamma \vdash \varphi \qquad \Gamma \vdash \psi
      \justifies
      \Gamma \vdash \andconn{\varphi}{\psi}
    \end{prooftree}
    \qquad
    \begin{prooftree}
      \Gamma \vdash \andconn{\varphi}{\psi}
      \justifies
      \Gamma \vdash \varphi
    \end{prooftree}
    \qquad
    \begin{prooftree}
      \Gamma \vdash \andconn{\varphi}{\psi}
      \justifies
      \Gamma \vdash \psi
    \end{prooftree}
  \]

  \[
    \begin{prooftree}
      \Gamma \vdash \varphi
      \justifies
      \Gamma \vdash \orconn{\varphi}{\psi}
    \end{prooftree}
    \qquad
    \begin{prooftree}
      \Gamma \vdash \psi
      \justifies
      \Gamma \vdash \orconn{\varphi}{\psi}
    \end{prooftree}
    \qquad
    \begin{prooftree}
      \Gamma \vdash \orconn{\varphi}{\psi}
      \quad
      \Gamma, \tyj{u}{\varphi} \vdash C
      \quad
      \Gamma, \tyj{v}{\psi} \vdash C
      \justifies
      \Gamma \vdash C
    \end{prooftree}
  \]

  \[
    \begin{prooftree}
      \Gamma, \tyj{u}{\varphi} \vdash \psi
      \justifies
      \Gamma \vdash \implconn{\varphi}{\psi}
    \end{prooftree}
    \qquad
    \begin{prooftree}
      \Gamma \vdash \implconn{\varphi}{\psi} \qquad \Gamma \vdash \varphi
      \justifies
      \Gamma \vdash \psi
    \end{prooftree}
  \]

  \[
    \begin{prooftree}
      \Gamma, \inj{x}{A} \vdash \propj{\varphi(x)}
      \quad
      \Gamma, \inj{x}{A} \vdash \varphi(x)
      \justifies
      \Gamma \vdash \forallconn{x}{A}{\varphi(x)}
    \end{prooftree}
    \qquad
    \begin{prooftree}
      \Gamma \vdash \forallconn{x}{A}{\varphi(x)}
      \quad
      \Gamma \vdash \inj{t}{A}
      \justifies
      \Gamma \vdash \varphi(t)
    \end{prooftree}
  \]

  \[
    \begin{prooftree}
      \begin{array}{@{}c}
        \Gamma, \inj{x}{A} \vdash \propj{\varphi(x)} \\
        \Gamma \vdash \varphi(t)
        \qquad
        \Gamma \vdash \inj{t}{A}  
      \end{array}
      \justifies
      \Gamma \vdash \existsconn{x}{A}{\varphi(x)}
    \end{prooftree}
    \qquad
    \begin{prooftree}
      \Gamma \vdash \exists{x}{A}{\varphi(x)}
      \qquad
      \Gamma, \inj{x}{A}, \tyj{y}{\varphi(x)} \vdash \psi
      \justifies
      \Gamma \vdash \psi
    \end{prooftree}
  \]
  \[
    \begin{prooftree}
      \justifies
      \Gamma \vdash \trueconn
    \end{prooftree}
    \qquad\qquad\qquad
    \begin{prooftree}
      \Gamma \vdash \falseconn
      \justifies
      \Gamma \vdash \varphi
    \end{prooftree}
  \]
\end{mdframed}
\caption{\label{natded-explicit-context} Natural deduction with explicit context.}
\end{figure}

\begin{example}
  We show how the intuitionistically valid formula

  \[
    \varphi \vee (\forall x \in A)\psi(x) \rightarrow
    (\forall x \in A)(\varphi \vee \psi(x))
  \]

  can be derived in our system, where $x$ is intended to be \emph{not} free in
  $\varphi$.

  \[
    \begin{prooftree}
      \[
        \[
          \Gamma \vdash \varphi \vee (\forall x \in A)\psi (x) \quad 
          \[
            \Gamma, \varphi \vdash \varphi
            \justifies
            \Gamma, \varphi \vdash \varphi \vee \psi (x)
          \]
          \[
            \[
              \Delta \vdash (\forall x \in A)\psi (x) \quad 
              \Delta \vdash x \in A
              \justifies
              \Delta \vdash \psi (x)
            \]
            \justifies
            \Delta \equiv \Gamma, (\forall x \in A)\psi (x) \vdash \varphi \vee \psi (x)
          \]
          \justifies
          \Gamma \equiv \varphi \vee (\forall x \in A)\psi (x), x \in A \vdash \varphi \vee \psi (x)
        \]
        \justifies
        \varphi \vee (\forall x \in A)\psi (x) \vdash (\forall x \in A)(\varphi \vee \psi (x))
      \]
      \justifies
      \varphi \vee (\forall x \in A)\psi (x) \rightarrow (\forall x \in A)(\varphi \vee \psi (x))
    \end{prooftree}
  \]
\end{example}

\begin{example}
  We have seen that, in the system of natural deduction with implicit
  hypotheses, we must pay attention to the parameters involved in the rules of
  $\forall$ introduction and $\exists$ elimination in order to avoid logical
  absurdities to be derived. The alert reader may have noticed that there are no
  side conditions of this sort in the rules with explicit contexts. This is
  because the definition of context itself, which disallows the presence of
  variables with the same name, takes care of such situations. As an example,
  notice how the derivation of the invalid $\forall xy (P(x) \rightarrow P(y))$
  gets blocked:

  \[
    \begin{prooftree}
      \[
        \[
          \[ \justifies x \in A, y \in A, u : P(x) \vdash y \in A \] \quad 
          \[
            x \in A, y \in A, u : P(x), z \in A \vdash P(z)
            \justifies
            x \in A, y \in A, u : P(x) \vdash (\forall x \in A)P(x)
          \]
          \justifies
          x \in A, y \in A, u : P(x) \vdash P(y)
        \]
        \justifies
        x \in A, y \in A \vdash P(x) \rightarrow P(y)
      \]
      \justifies
      \vdash (\forall x \in A)(\forall y \in A)(P(x) \rightarrow P(y))
    \end{prooftree}
  \]

  Notice how there is no way to complete the right branch: we would be able to
  derive it only in the case $x = z$, but this is not possible by the definition
  of context, which requires the $\forall$ introduction rule to refer to a fresh
  variable (in this case, $z$).

  Consider another logical fallacy that can be derived if we ignore the variable
  conditions of natural deduction:

  \[
    \begin{prooftree}
      \[
        \[
          \[
            \justifies
            \exists x A
            \using{u}
          \] \quad 
          \[
            \justifies
            A[w/x]
            \using{v}
          \]
          \justifies
          A[w/x]
          \using{v}
        \]
        \justifies
        \forall x A
      \]
      \justifies
      \exists x A \rightarrow \forall x A
    \end{prooftree}
  \]

  This derivation is \emph{not} valid, since the parameter $w$ used in the
  discharged hypothesis of the existential elimination is free in the conclusion
  of the rule. Notice how this incorrect development of the proof gets blocked
  by the definition of context itself:

  \[
    \begin{prooftree}
      \[
        \[
          \[
            \justifies
            (\exists x \in A) \varphi(x), x \in A \vdash (\exists x \in A)
            \varphi(x)
          \]
          \quad 
          (\exists x \in A) \varphi(x), x \in A, y \in A, \varphi(y) \vdash \varphi(x)
          \justifies
          (\exists x \in A) \varphi(x), x \in A \vdash \varphi(x)
        \]
        \justifies
        (\exists x \in A) \varphi(x) \vdash (\forall x \in A)\varphi(x)
      \]
      \justifies
      \vdash (\exists x \in A) \varphi(x) \rightarrow (\forall x \in A)\varphi(x)
    \end{prooftree}
  \]

  The variable $y$ introduced by the existential elimination rule is forced to
  be different from $x$, which is already in context. Hence, there is no way to
  unify the hypothesis, which must be $\varphi(y)$, with the conclusion
  $\varphi(x)$.

\end{example}

\begin{example}
  A common misconception about intuitionistic logic in that it is in explicit
  opposition to classical logic, to the point that typically classical
  principles like LEM are provably false in it. This view is, of course, wrong,
  since intuitionistic logic and in general constructive mathematics in the
  Bishop sense are compatible with classical mathematics. In particular, observe
  that $\neg \neg (A \vee \neg A)$ is an intuitionistic tautology:

  \[
    \begin{prooftree}
      \[
        \[\justifies (A \vee \neg A) \rightarrow \bot \using{u}\] \quad 
        \[
          \[
            \[
              \[\justifies (A \vee \neg A) \rightarrow \bot \using{u}\] \quad 
              \[
                \[ \justifies A \using{w} \]
                \justifies
                A \vee \neg A
              \]
              \justifies
              \bot
            \]
            \justifies
            A \rightarrow \bot
            \using{w}
          \]
          \justifies
          A \vee \neg A
        \]
        \justifies
        \bot
      \]
      \justifies
      \neg \neg (A \vee \neg A)
    \end{prooftree}
  \]
  
\end{example}

\subsection{Proof terms}

We now see the \emph{static} nature of the Curry-Howard isomorphism, that is the
correspondence between propositions (expressed as formulas) and types for
$\lambda$-terms. This correspondence is just another way to reveal the
connections (and differences) between language and meta-language: we
\emph{construct proofs} of a proposition $\varphi$ in the same way as we
\emph{derive} the \emph{judgement} $\truej{\varphi}$. The first is an
internalization in the object language of the second, which is a feature of the
meta-language.

According to the BHK interpretation of intuitionistic logic, a proof of a
proposition is an effective construction. Proof terms formalize this aspect,
showing not only that a proposition is true, but also giving a construction for
it. Writing $\tyj{p}{\varphi}$ for the judgement ``$p$ is a proof term of the
proposition $\varphi$'', we have thus the following equivlence

We now extend our system of natural deduction with proof terms:

\subsubsection{Conjunction}

We know what counts as a proof of the proposition $\andconn{\varphi}{\psi}$,
namely a proof of $\varphi$ and a proof of $\psi$. It makes sense for a proof
term for the conjunction, therefore, to be a pair of proofs, one of $\varphi$
and one of $\psi$:

\[
  \begin{prooftree}
    \Gamma \vdash \tyj{M}{\varphi} \qquad \Gamma \vdash \tyj{N}{\psi}
    \justifies
    \Gamma \vdash \tyj{\pair{M}{N}}{\andconn{\varphi}{\psi}}
  \end{prooftree}
\]

Then, it seems reasonable to extract the first or second of this pair of proofs
by projection:

\[
  \begin{prooftree}
    \Gamma \vdash \tyj{M}{\andconn{\varphi}{\psi}}
    \justifies
    \Gamma \vdash \tyj{\snd{M}}{\psi}
  \end{prooftree}
  \qquad
  \begin{prooftree}
    \Gamma \vdash \tyj{M}{\andconn{\varphi}{\psi}}
    \justifies
    \Gamma \vdash \tyj{\fst{M}}{\varphi}
  \end{prooftree}
\]

\paragraph{Disjunction}

A proof of a disjunction $\orconn{\varphi}{\psi}$ is a proof of $\varphi$ or a
proof of $\psi$, together with the information telling us \emph{which one} of
the two. The proof term then is the proof itself, with \emph{tags} telling us
whether the proof refers to the first or the second disjunct, and what is the
disjunct about which we are not saying anything:

\[
  \begin{prooftree}
    \Gamma \vdash \tyj{M}{\varphi}
    \justifies
    \Gamma \vdash \tyj{\inleft{\psi}{M}}{\orconn{\varphi}{\psi}}
  \end{prooftree}
  \qquad
  \begin{prooftree}
    \Gamma \vdash \tyj{M}{\psi}
    \justifies
    \Gamma \vdash \tyj{\inright{\varphi}{M}}{\orconn{\varphi}{\psi}}
  \end{prooftree}
\]

A proof of a disjunction is used by \emph{reading} the information on the term
itself, and proving the conclusion by case analysis.

\[
  \begin{prooftree}
    \Gamma \vdash \tyj{M}{\orconn{\varphi}{\psi}}
    \qquad
    \Gamma, \tyj{x}{A} \vdash \tyj{N_1}{C}
    \qquad
    \Gamma, \tyj{y}{B} \vdash \tyj{N_2}{C}
    \justifies
    \Gamma \vdash \tyj{\orelimop{M}{(x)N_1}{(y)N_2}}{C}
  \end{prooftree}
\]

\subsubsection{Implication}

A proof of an implication $\implconn{\varphi}{\psi}$ is understood to be a
method, or operation, that yields a proof of $\psi$ given a proof of $\varphi$.

\[
  \begin{prooftree}
    \Gamma, \tyj{x}{\varphi} \vdash \tyj{b(x)}{\psi}
    \justifies
    \Gamma \vdash \tyj{\implctor{x}{b(x)}}{\implconn{\varphi}{\psi}}
  \end{prooftree}
  \qquad
  \begin{prooftree}
    \Gamma \vdash \tyj{M}{\implconn{\varphi}{\psi}}
    \qquad
    \Gamma \vdash \tyj{N}{\varphi}
    \justifies
    \Gamma \vdash \tyj{\implelimop{M}{N}}{\psi}
  \end{prooftree}
\]

\subsubsection{Universal quantification}

A similar situation is with universal quantification: a proof of
$\forallconn{x}{A}{\varphi(x)}$ is an operation that yields a proof of
$\varphi(x)$ given an element $x$ of the set $A$.

\[
  \begin{prooftree}
    \Gamma, x \in A \vdash \propj{\varphi(x)}
    \qquad
    \Gamma, x \in A \vdash \tyj{b(x)}{\varphi (x)}
    % \Gamma \vdash \tyj{M}{A[w/x]}
    \justifies
    \Gamma \vdash \tyj{\forallctor{w}{b(x)}}{\forallconn{x}{A}{\varphi(x)}}
  \end{prooftree}
\]
\[
  \begin{prooftree}
    \Gamma \vdash \inj{t}{A}
    \qquad
    \Gamma \vdash \tyj{M}{\forallconn{x}{A}{\varphi(x)}}
    \justifies
    \Gamma \vdash \tyj{\forallelimop{M}{t}}{\varphi(t)}
  \end{prooftree}
\]

\subsubsection{Existential quantification}

Again, the rules of $\exists$ decorated with proof terms validate the usual
interpretation of the intuitionistic existential quantifier.

\[
  \begin{prooftree}
    \Gamma, \inj{x}{A} \vdash \propj{\varphi(x)}
    \quad
    \Gamma \vdash \inj{t}{A}
    \quad
    \Gamma \vdash \tyj{M}{\varphi(t)}
    \justifies
    \Gamma \vdash \tyj{\pair{t}{M}}{\existsconn{x}{A}{\varphi(x)}}
  \end{prooftree}
\]
\[
  \begin{prooftree}
    \Gamma \vdash \tyj{M}{\existsconn{x}{A}{\varphi(x)}}
    \quad
    \Gamma, x \in A, \tyj{y}{\varphi(x)} \vdash \tyj{N}{\psi}
    \justifies
    \Gamma \vdash \tyj{\existselimop{M}{(y)N}}{\psi}
  \end{prooftree}
\]

\subsubsection{True and false}

The proposition $\top$ is trivially true, therefore we fix a constant symbol
$\star$ that denotes its unique, canonical proof.

\[
  \begin{prooftree}
    \justifies
    \Gamma \vdash \tyj{\star}{\trueconn}
  \end{prooftree}
\]

The false proposition $\bot$ has no proof, hence, by \emph{ex falso quodlibet},
if we are presented with a proof of it we might as well deduce anything.

\[
  \begin{prooftree}
    \Gamma \vdash \tyj{M}{\falseconn}
    \justifies
    \Gamma \vdash \tyj{\falseelimop{A}{M}}{A}
  \end{prooftree}
\]

Figure~\ref{natded-proofterms} show the complete formal system, where the
obvious ``prop'' judgements are left implicit.

\begin{figure}[ht]
  \begin{mdframed}
    \[
      \begin{prooftree}
        \Gamma \vdash \tyj{M}{\varphi} \qquad \Gamma \vdash \tyj{N}{\psi}
        \justifies
        \Gamma \vdash \tyj{\pair{M}{N}}{\andconn{\varphi}{\psi}}
      \end{prooftree}
      \qquad
      \begin{prooftree}
        \Gamma \vdash \tyj{M}{\andconn{\varphi}{\psi}}
        \justifies
        \Gamma \vdash \tyj{\fst{M}}{\varphi}
      \end{prooftree}
      \qquad
      \begin{prooftree}
        \Gamma \vdash \tyj{M}{\andconn{\varphi}{\psi}}
        \justifies
        \Gamma \vdash \tyj{\snd{M}}{\psi}
      \end{prooftree}
    \]
    \[
      \begin{prooftree}
        \Gamma \vdash \tyj{M}{\varphi}
        \justifies
        \Gamma \vdash \tyj{\inleft{\psi}{M}}{\orconn{\varphi}{\psi}}
      \end{prooftree}
      \qquad
      \begin{prooftree}
        \Gamma \vdash \tyj{M}{\psi}
        \justifies
        \Gamma \vdash \tyj{\inright{\varphi}{M}}{\orconn{\varphi}{\psi}}
      \end{prooftree}
    \]
    \[
      \begin{prooftree}
        \Gamma \vdash \tyj{M}{\orconn{\varphi}{\psi}}
        \qquad
        \Gamma, \tyj{x}{\varphi} \vdash \tyj{d_1(x)}{\phi}
        \qquad
        \Gamma, \tyj{x}{\psi} \vdash \tyj{d_2(x)}{\phi}
        \justifies
        \Gamma \vdash \tyj{\orelimop{M}{(x)d_1}{(x)d_2}}{\phi}
        \using{\orelimrule}
      \end{prooftree}
    \]
    \[
      \begin{prooftree}
        \Gamma, \tyj{x}{\varphi} \vdash \tyj{b(x)}{\psi}
        \justifies
        \Gamma \vdash \tyj{\implctor{x}{b(x)}}{\implconn{\varphi}{\psi}}
      \end{prooftree}
      \qquad
      \begin{prooftree}
        \Gamma \vdash \tyj{M}{\implconn{\varphi}{\psi}}
        \qquad
        \Gamma \vdash \tyj{N}{\varphi}
        \justifies
        \Gamma \vdash \tyj{\implelimop{M}{N}}{\psi}
      \end{prooftree}
    \]
    \[
      \begin{prooftree}
        \begin{array}{@{}c}
          \Gamma, x \in A \vdash \propj{\varphi(x)} \\
          \Gamma, x \in A \vdash \tyj{b(x)}{\varphi (x)}  
        \end{array}
        \justifies
        \Gamma \vdash \tyj{\forallctor{w}{b(x)}}{\forallconn{x}{A}{\varphi(x)}}
      \end{prooftree}
      \qquad
      \begin{prooftree}
        \begin{array}{@{}c}
          \Gamma, \inj{x}{A} \vdash \propj{\varphi(x)} \\
          \Gamma \vdash \inj{t}{A}
          \qquad
          \Gamma \vdash \tyj{M}{\forallconn{x}{A}{\varphi(x)}}
        \end{array}
        \justifies
        \Gamma \vdash \tyj{\forallelimop{M}{t}}{\varphi(t)}
      \end{prooftree}
    \]
    \[
      \begin{prooftree}
        \begin{array}{@{}c}
          \Gamma, \inj{x}{A} \vdash \propj{\varphi(x)} \\  
          \Gamma \vdash t \in A
          \quad
          \Gamma \vdash \tyj{M}{\varphi(t)}
          \end{array}
        \justifies
        \Gamma \vdash \tyj{\pair{t}{M}}{\existsconn{x}{A}{\varphi(x)}}
      \end{prooftree}
      \qquad
      \begin{prooftree}
        \begin{array}{@{}c}
          \Gamma \vdash \tyj{M}{\existsconn{x}{A}{\varphi(x)}}\\
          \Gamma, x \in A, \tyj{y}{\varphi(x)} \vdash \tyj{N}{\psi}
        \end{array}
        \justifies
        \Gamma \vdash \tyj{\existselimop{M}{(y)N}}{\psi}
      \end{prooftree}
    \]
  \end{mdframed}
  \caption{\label{natded-proofterms} Natural deduction with proof terms.}
\end{figure}

\begin{example}

  Below is a derivation of the intuitionistically valid proposition

  \[
    (\psi \rightarrow \chi) \rightarrow (\varphi \rightarrow \psi)
    \rightarrow \varphi \rightarrow \chi
  \]

  \[
    \begin{prooftree}
      \[
        \[
          \[
            f \in ..., ... \vdash f \in ... \quad 
            \[
              ... \vdash g \in \varphi \rightarrow \psi \quad 
              ... \vdash x \in \varphi
              \justifies
              ... \vdash Ap(g,x) \in \psi
            \]
            \justifies
            f \in ..., g \in ..., x \in \varphi \vdash Ap(f, Ap(g,x)) \in \chi
          \]
          \justifies
          f \in ..., g \in \varphi \rightarrow \psi \vdash \lambda x . Ap(f, Ap(g,x)) \in \varphi \rightarrow \chi
        \]
        \justifies
        f \in \psi \rightarrow \chi \vdash \lambda g x . Ap(f, Ap(g,x)) \in (\varphi \rightarrow \psi) \rightarrow \varphi \rightarrow \chi
      \]
      \justifies
      \vdash \lambda f g x . Ap(f, Ap(g,x)) \in (\psi \rightarrow \chi) \rightarrow (\varphi \rightarrow \psi)
    \rightarrow \varphi \rightarrow \chi
    \end{prooftree}
  \]

  The proof term of this proposition corresponds to the $B$ combinator in
  combinatory logic, which performs function composition. In fact, notice how

  \[
    B f g x = f(g(x))
  \]

\end{example}

\begin{example}

  Below is a derivation with proof terms of the intuitionistically valid
  proposition

  \[
    (\exists x \in A)\varphi(x) \rightarrow
    \neg (\forall x \in A)\neg \varphi(x)
  \]
  
  \[
    \begin{prooftree}
      \[
        \[
          e \in ..., ... \vdash e \in ... \quad 
          \[
            \[
              e \in ..., f \in ... \vdash f \in ... \quad 
              x \in A, ... \vdash x \in A
              \justifies
              ... \vdash Ap(f,x) \in \varphi(x) \rightarrow \bot
            \] \quad 
            ..., y \in \varphi(x) \vdash y \in \varphi(x)
            \justifies
            e \in ..., f \in ..., x \in A, y \in \varphi(x) \vdash Ap(Ap(f,x),y) \in \bot
          \]
          \justifies
          e \in ..., f \in (\forall x \in A)(\varphi(x) \rightarrow \bot) \vdash E(e, (x,y)Ap(Ap(f,x),y)) \in \bot
        \]
        \justifies
        e \in (\exists x \in A)\varphi(x) \vdash \lambda f . E(e, (x,y)Ap(Ap(f,x),y)) \in [(\forall x \in A)(\varphi(x) \rightarrow \bot)] \rightarrow \bot
      \]
      \justifies
      \vdash \lambda e f . E(e, (x,y)Ap(Ap(f,x),y)) \in (\exists x \in A)\varphi(x) \rightarrow
    \neg (\forall x \in A)\neg \varphi(x)
    \end{prooftree}
  \]
\end{example}

\begin{example}

  Below is a derivation with proof terms of the intuitionistically valid
  proposition

  \[
    (\forall x \in A)(\varphi(x) \rightarrow \psi(x))
    \rightarrow
    ((\exists x \in A)\varphi(x) \rightarrow (\exists x \in A)\psi(x))
  \]

  \[
    \begin{prooftree}
      \[
    \[
      ..., e \in ... \vdash e \in ...
      \[
        \[
          \[
            ..., f \in ..., ... \vdash f \in ...
            \quad
            ..., x \in A, ... \vdash x \in A
          \justifies
            ..., f \in ..., ... \vdash Ap(f,x) \in \varphi(x) \rightarrow \psi(x)
          \]
          \quad ..., y \in \varphi(x) \vdash y \in \varphi(x)
        \justifies
          ..., x \in A, y \in \varphi(x) \vdash Ap(Ap(f,x),y) \in \psi(x)
        \]
        ..., x \in A, ... \vdash x \in A
        \justifies
        ..., x \in A, y \in \varphi(x) \vdash <x, Ap(Ap(f,x),y)> \in (\exists x \in A)\psi(x)
      \]
    \justifies
      f \in ..., e \in (\exists x \in A)\varphi(x) \vdash E(e, (x,y)<x, Ap(Ap(f,x),y)>) \in (\exists x \in A)\psi(x)
    \]
  \justifies
    f \in (\forall x \in A)(\varphi(x) \rightarrow \psi(x)) \vdash \lambda e . E(e, (x,y)<..>) \in (\exists \rightarrow \exists)
  \]
\justifies
  \vdash \lambda f e . E(...) \in (\forall x \in A)(\varphi(x) \rightarrow \psi(x))
    \rightarrow
    ((\exists x \in A)\varphi(x) \rightarrow (\exists x \in A)\psi(x))
    \end{prooftree}
  \]
\end{example}

\subsection{Proof normalization}

\subsubsection{Conjunction}



\[
  \begin{prooftree}
    \[
    \Gamma \vdash A \qquad \Gamma \vdash B
    \justifies
    \Gamma \vdash A \wedge B
    \]
    \justifies
    \Gamma \vdash A
  \end{prooftree}
  \qquad \leadsto \qquad
  \Gamma \vdash A
\]

\[
  \begin{prooftree}
    \[
    \Gamma \vdash A \qquad \Gamma \vdash B
    \justifies
    \Gamma \vdash A \wedge B
    \]
    \justifies
    \Gamma \vdash B
  \end{prooftree}
  \qquad \leadsto \qquad
  \Gamma \vdash B
\]

\subsubsection{Disjunction}

\[
  \begin{prooftree}
    \[
      \Gamma \vdash A
      \justifies
      \Gamma \vdash A \vee B
    \]
    \qquad \Gamma, A \vdash C \qquad \Gamma, B \vdash C
    \justifies
    \Gamma \vdash C
  \end{prooftree} \qquad \leadsto \qquad
  \begin{prooftree}
    \Gamma \vdash A \qquad \Gamma, A \vdash C
    \justifies
    \Gamma \vdash C
  \end{prooftree}
\]

\subsubsection{Implication}

\[
  \begin{prooftree}
    \[
      \Gamma, A \vdash B
      \justifies
      \Gamma \vdash A \supset B
    \]
    \qquad \Gamma \vdash A
    \justifies
    \Gamma \vdash B
  \end{prooftree}\qquad \leadsto \qquad
  \begin{prooftree}
    \Gamma, A \vdash B \qquad \Gamma \vdash A
    \justifies
    \Gamma \vdash B
  \end{prooftree}
\]

\subsubsection{Universal quantification}

\[
  \begin{prooftree}
    \[
      \Gamma \vdash A[w/x]
      \justifies
      \Gamma \vdash \forall x A
    \]
    \justifies
    \Gamma \vdash A[t/x]
  \end{prooftree} \qquad \leadsto \qquad
  \Gamma[t/w] \vdash A[w/x][t/w] \equiv \Gamma \vdash A[t/x]
\]

\subsubsection{Existential quantification}


\[
  \begin{prooftree}
    \[
      \Gamma \vdash A[t/x]
      \justifies
      \Gamma \vdash \exists x A
    \]
    \qquad \Gamma, A[w/x] \vdash B
    \justifies
    \Gamma \vdash B
    \using{w \notin \mathrm{FV}(\Gamma, B)}
  \end{prooftree}
\]
\[
  \qquad \leadsto \qquad
  \begin{prooftree}
    \Gamma \vdash A[t/x]
    \qquad
    \[
      \Gamma, A[w/x] \vdash B
      \justifies
      \Gamma[t/w], A[w/x][t/w] \vdash B[t/w] \equiv \Gamma, A[t/x] \vdash B
    \]
    \justifies
    \Gamma \vdash B
  \end{prooftree}
\]

\subsection{Normalization as computation}

%%% Local Variables:
%%% mode: latex
%%% TeX-master: "../notes"
%%% End:
