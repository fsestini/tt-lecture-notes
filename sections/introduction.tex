
\section{Introduction}

Modern day mathematics is built on top of set theory, where the existence of a
universe of sets closed under standard operations is postulated. This approach
has two main weak points

\begin{itemize}
\item intrinsic difficulty, due to G\"odel's incompleteness theorem, in showing
  that such universe can exist at all;
\item assumption, lacking a reasonable philosophical justification, that all
  mathematical concepts can be reduced to a single notion, that of set, and a
  single relation, that of membership.
\end{itemize}

Under this approach, sets are implicitly understood to exist in a static
supernatural world, whereas mathematics, as all intellectual products, has a
\strong{human nature} with historical and \strong{dynamic} origin.

Here we consider the constructivist point of view, that has the motivation of
providing a \strong{finer grid} to look at reality, \strong{preserving} pieces
of information and conceptual \strong{distinctions} that would be lost of not
considered in the classical approach. Thus constructivism should firstly be seen
as a \strong{widening} of the scope of classical mathematics, and only
consequently as a restriction of its foundational principles.

Historically, there have been many approaches to constructivism. Our intention
is to preserve all notions and conceptual distinctions as much as possible, thus
leading to an arsenal of foundational principles that is reduced to a
\strong{minimum}. In particular, we wish to

\begin{itemize}
\item preserve an effective notion of set all of whose elements are produced by
  a finite number of rules, and to block impredicative definitions. Hence we
  avoid the powerset axiom;
\item preserve a positive notion of existence. Hence the underlying logic must
  be intuitionistic;
\item preserve the distinction between operation (algorithm, explicit procedure)
  and function (total and single-valued relation). Hence we avoid the axiom of
  unique choice AC! (and therefore AC), that would identify the two concepts.
\end{itemize}

An advantage of this approach is that every mathematician can recognize such
theory as one that can be consistently extended to match his own.

\subsection{Basic notions}

A central concept in the philosophy of constructive mathematics is that all
entites consist of constructions. Therefore, we wish to know and control the
concrete content of every entity that we manipulate and statement that we
assert. Hence, the preservation of information is of central importance.

\paragraph{Sets}

Here, sets have an effective nature, as opposed to axiomatic set theory:

\begin{itemize}
\item each set is specified by a finite number of rules to build all of its
  elements;
\item no universe of all sets is assumed to be given a priori. We say that
  ``set'' is an \emph{open} concept.
\end{itemize}

There is a human nature in these constructions: it is we that we call some
aggregate a set when we know how to build its elements. So to know that $X$ is a
set is to know how to build all its elements.  There is no need to specify the
language in which the rules are given, as soon as it can be understood by a
reasonable mathematician. This requirement is fullfilled as soon as the set is
described using

\begin{itemize}
\item a language that is unambiguous;
\item a finite amount of information (such as a finite number of rules, or an
  infinite number of them as described by a finite number of meta-level rules,
  etc...)
\end{itemize}

Every set automatically comes with a notion of \strong{element}, which is an
entity that can be obtained by repeated use of the rules associated with that
set. Every set $X$ must also come equipped with a specific notion of
\strong{equality} $=_X$, which must be an equivalence relation. Every
mathematical entity involving elements of $X$ is expected to respect this
equality. So in order to define a set, we need to give

\begin{itemize}
\item a finite number of rules that allow to construct all elements of the set;
\item a notion of equality between elements of the set.
\end{itemize}

\paragraph{Quotient sets}

It follows that sets are closed under \strong{quotients}: given a set $X$ and an
equivalence relation $\sim$ over $X$, we can construct a set $(X,\sim)$ simply by
changing $=_X$ with $\sim$, leaving the element-forming rules untouched.
Consider for example the set $X^*$ of finite lists of elements of the set
$X$. We can construct the set of finite subsets of $X$ by simply quotienting
$X^*$ with an equivalence relation $\sim$ that identifies lists with the same
elements regardless of the order of appearence or their quantity.

\paragraph{Collections}

We can also consider objects of a certain logical type without having
rules to generate all of them, provided we have grasped what it means
to be an object of that type.
We call these aggregates \strong{collections}, and we write $p :
\mathcal{P}$ for an object $p$ in a collection $\mathcal{P}$. Every collection
comes with a notion of equality preserved by all mathematical entities that
involve objects in that collection. So to know that some aggregate is a
collection is

\begin{itemize}
\item to have grasped what it means to be an object of that collection, and
\item to have given a notion of equality for objects of that collection
\end{itemize}

An example of a collection is the collection $\mathcal{P}(X)$ of subsets of a
set $X$. A simple diagonalization shows that it is impossible to generate all
possible subsets of a set by means of a finite number of rules, so
$\mathcal{P}(X)$ is certainly not a set. However, it is completely clear to us
what constitues a subset of a set, and when two subsets are equal.

The distinction between sets and collections has some analogy with the concepts
of set/class in classical set theory, but different motivations, that have to do
with

\begin{itemize}
\item the quality of information, distinguishing between effective, computable
  domains (sets) and non-computable ones (collections);
\item the flow of time: once a set is defined, it remains the same
  forever. Conversely, a collection may change its contents with the time. For
  example, the collection $\mathcal{P}(X)$ grows as we define new subsets of
  $X$.
\end{itemize}

Hence the need for collections as an entity that is distinct from sets comes
from the need, among others, to recognize and respect the dynamic nature of our
mathematical world.

\paragraph{Family of sets}

Having described sets and collections, we now need a primitve concept for
associating instances of these two concepts. Suppose we know that $T(x)
\text{set} (a \in I)$, that is, that $T(x)$ yields a description of a set
whenever $x$ is an explicit description of an element of the set $I$. We call
this a \strong{family of sets} indexed by $I$.

\paragraph{Operations}

In the same way as a family of sets maps elements of an index set to sets, we
may want to map element of a set to elements of another set. An
\strong{operation} between a set $I$ and a set $X$, written
$p(i) \in X (i \in I)$ is the same as a family of elements of $X$ indexed by
$I$, that is an explicit description $p(i)$ of an element of $X$ for each
element $i$ of $I$. Operations must preserve equality, that is, $p(i) =_X p(i')$
whenever $i =_I i'$.
Equality between operations is extensional, that is, we conclude $p = p'$
whenever $p(i) =_X p'(i)$ for all $i \in I$.
We denote by $\mathrm{Op}(I,X)$ the collection of operations between the sets
$I$ and $X$.

The difference between functions (total single-valued relations) and operations
is clearified by a simple diagonalization argument. We have an operation $p$
from $A$ to $B(x)$ whenever we are able to derive the judgement
$\hinj{p(x)}{B(x)}{x \in A}$, meaning that we are able to effectively enumerate
all operations that can be derived in our system. These are moreover total, and
it is a well known result of computability theory that such enumeration cannot
exhaustively enumerate all total effective functions. We therefore must conclude
that there are strictly more functions than operations.

An operation $\mathcal{O}$ from a collection $\mathcal{P}$ into a collection
$\mathcal{Q}$ is the same as a family of objects of $\mathcal{Q}$ indexed on
$\mathcal{P}$, i.e., an explicit description of an object $\mathcal{O}(p)$ in
$\mathcal{Q}$ for an arbitrary $p : \mathcal{P}$. Note that $\mathcal{O}$ can be
well-defined even if we cannot produce all its arguments. In fact, correct
effective operations may be based only of the evidence that the argument object
actually belongs to the indended collection. An example of this is the identity
operation on collections.

We will also consider families of objects in a collection indexed by a set, that
is, operations $p(i) : \mathcal{P} (i \in I)$ from a set $I$ to a collection
$\mathcal{P}$. An example of this is the operation with logical type $I
\rightarrow \mathcal{P}(X)$ that associates every index to a subset of $X$.

%%% Local Variables:
%%% mode: latex
%%% TeX-master: "../notes"
%%% End:
