\section{Judgements}

In usual mathematical practice, we specify formulas and rules of inference, and
then \emph{a posteriori} give them a semantical interpretation, usually
interpreting them in set theory. Here we will avoid keeping form and meaning
(content) apart, and instead we will at the same time display certain forms of a
judgement and inference and explain them semantically.

Our formal system will consist of a set of rules for deriving judgements of a
variety of forms. Judgements are statements about one or more syntactic objects
of a specified sort, and they must be explained by saying what it is that we
must know in order to have the right to make a judgement of any one of the
various forms.

So far we have introduced, in particular, the notion of set and element of a
set. So it seems mandatory to have judgements that allow us to assert facts
abount these entities. Hence, we consider four forms:

\begin{enumerate}
\item $\setj{A}$ ($A$ is a set);
\item $A = B$ ($A$ and $B$ are equal sets);
\item $\inj{a}{A}$ ($a$ is an element of the set $A$);
\item $\eqelj{a}{b}{A}$ ($a$ and $b$ are equal elements of the set $A$).
\end{enumerate}

\subsection{Explanations of the forms of judgement}

\subsubsection{$A$ set}

To explain the judgement $\setj{A}$, we have to know what a set is, or
equivalently what it is to be known in order to have the right to make such a
judgement. Intuitively we know that a set is defined by prescribing how its
elements are formed. But $2 + 2$ is intuitively an element of $\NN$, even if it
cannot be obtained by the rules. So we need to distinguish between
\emph{canonical} and \emph{noncanonical} elements. Hence \emph{a set $A$ is
  defined by prescribing how a canonical element of $A$ is formed, as well as
  how equal canonical elements of $A$ are formed}.

\subsubsection{$A = B$}

Two sets $A$ and $B$ are equal if

\[
  \begin{prooftree}
    \inj{a}{A}
    \justifies
    \inj{a}{B}
  \end{prooftree},\qquad
  \begin{prooftree}
    \inj{a}{B}
    \justifies
    \inj{a}{A}
  \end{prooftree}
\]

and

\[
\begin{prooftree}
    \inj{a = b}{A}
    \justifies
    \inj{a = b}{B}
  \end{prooftree},\qquad
  \begin{prooftree}
    \inj{a = b}{B}
    \justifies
    \inj{a = b}{A}
  \end{prooftree}
\]

for arbitrary canonical elements $a, b$.

\subsubsection{$\inj{a}{A}$}

Assuming that we know what a set $A$ is, and therefore how to construct its
canonical elements, then \emph{an elements $a$ of a set $A$ is an effective
  method which, when executed, yields a canonical element of $A$ as a result.}

\subsubsection{$a = b \in A$}

Finally \emph{two arbitrary elements $a$ and $b$ of a set $A$ are equal if, when
executed, $a$ and $b$ yield equal canonical elements of $A$ as results.}

This definition makes sense because we assume that we already know that $A$ is a
set, and therefore when two canonical elements are equal.

\subsection{Rules of equality}

TODO...

\subsection{Hypothetical judgements and substitution rules}

We have seen that a family of elements (and similarly a family of sets) can be
expressed in the form $\hinj{b(x)}{B}{x \in A}$. This is just a generalization
of the judgement $\inj{b}{B}$, meaning that $b$ is an element of set $B$, to a
hypothetical judgement saying that $b(x)$ is an element of the set $B$ provided
$x$ is an element of the set $A$. Such a generalization can be carried out for
all four forms of judgement, as follows.

First, we see how terms in the theory are represented in the judgements. In
particular, we have terms/expressions $t$. These terms can have zero or more
free variables in them, hence $t(x_1, \dots, x_n)$. We can think of the term
$t(a_1, \dots, a_n)$ as the term $t$ in which the free variables have been
replaced with the terms $a_1, \dots, a_n$. The syntax and grammar of terms is
left unspecified and open-ended, we just assume a primitive notion of
\emph{variable} and sustitution.

Associated to the notion of (free) variable, there is the notion of abstraction,
which means that a variable in a term acts as a mere place-holder. We denote by
$(x)t$ the abstraction of $x$ in the term $t$, which may of may not be free in
it.

\paragraph{1.}

The first form of hypothetical judgement is the generalization of the set
judgement

\[
  \hsetj{B(x)}{x \in A}
\]

which says that $B(x)$ is a set under the assumption that $x \in A$, of that $B$
is a family of sets indexed by $A$. Moreover, such a family should respect
equality of the index set involved, hence justifying the following
\emph{substitution rules}

\[
  \begin{prooftree}
    \inj{a}{A} \qquad \hsetj{B(x)}{x \in A}
    \justifies
    \setj{B(a)}
  \end{prooftree}, \qquad
  \begin{prooftree}
    \eqelj{a}{c}{A} \qquad \hsetj{B(x)}{x \in A}
    \justifies
    \seteqj{B(a)}{B(c)}
  \end{prooftree}
\]

\paragraph{2.}

We said that equality between families of sets is meant to be extensional. This
is the meaning of the hypothetical judgement

\[
  \hseteqj{B(x)}{D(x)}{x \in A}
\]

which says that $B(x)$ and $D(x)$ are equal families of sets over the set
$A$. Hence the following substitution rule is justified

\[
  \begin{prooftree}
    \inj{a}{A} \qquad \hseteqj{B(x)}{D(x)}{x \in A}
    \justifies
    \seteqj{B(a)}{D(a)}
  \end{prooftree}
\]

We can now derive the rule

\[
  \begin{prooftree}
    \eqelj{a}{c}{A} \qquad \hseteqj{B(x)}{D(x)}{x \in A}
    \justifies
    \seteqj{B(a)}{D(c)}
  \end{prooftree}
\]

TODO...

\paragraph{3.}

A judgement of the form

\[
  \hinj{b(x)}{B(x)}{x \in A}
\]

means that we know $b(a)$ to be an element of the set $B(a)$ assuming we know
$a$ to be an element of the set $A$, that is, $b$ is an operation/family from
elements of the set $A$ to elements of the corresponding set in the family
$B(x)$. Moreover, operations must respect equality, thus justifying the
following substitution rules:

\[
  \begin{prooftree}
    \inj{a}{A} \qquad \hinj{b(x)}{B(x)}{x \in A}
    \justifies
    \inj{b(a)}{B(a)}
  \end{prooftree}, \qquad
  \begin{prooftree}
    \eqelj{a}{c}{A}\qquad \hinj{b(x)}{B(x)}{x \in A}
    \justifies
    \eqelj{b(a)}{b(c)}{B(a)}
  \end{prooftree}
\]

\paragraph{4.}

The judgement

\[
  \heqelj{b(x)}{d(x)}{B(x)}{x \in A}
\]

means that $b(x)$ and $d(x)$ are (extensionally) equal operations/families of
elements indexed by $A$, hence justifying the following substitution rule:

\[
  \begin{prooftree}
    \inj{a}{A} \qquad \heqelj{b(x)}{d(x)}{B(x)}{x \in A}
    \justifies
    \eqelj{b(a)}{d(a)}{B(a)}
  \end{prooftree}
\]

\subsubsection{Hypothetical judgements with more than one assumption}

TODO...

%%% Local Variables:
%%% mode: latex
%%% TeX-master: "../notes"
%%% End:
