\section{Sets}

We have already seen that to define a set is to give an exhaustive method (in
the form of a finite number of rules) to construct its canonical elements,
together with a notion of equality between those elements. We now give a precise
description of the forms of rules that will be used to introduce sets in our
theory. There will be four kinds of rules:

\begin{enumerate}
\item Set formation rules: they describe under which conditions we may infer
  that $A$ is a set, and when two sets $A$ and $B$ are equal;
\item Introduction rules: they describe how to construct the canonical elements
  of a set, and when two canonical elements are to be considered equal;
\item Elimination rules: they show how to \emph{use} the elements of a set, or
  equivalently how to define operations of elements of the set. These give a
  form of induction/recursion, saying that in order to define an operation on
  elements of $A$ it is sufficient to define it only of its canonical elements;
\item Equality rules: they relate introduction and elimination rules by
  describing how the selector computes on the different canonical elements;
\end{enumerate}

Note that elimination rules have a \strong{double meaning}: they show how to
define operations by (primitive) recursion on the canonical elements of a set,
and they equip the inductive type with an induction principle, basically saying
that the set is the least one that is closed under the introduction rules.
Elimination rules come in pairs, where the second shows how the eliminator
symbol that we introduced \strong{respects equality}.

Equality rules give a formal and precise (computational) semantics to the
symbols that are introduced in the elimination rules. In particular, they
explain how these symbols behave by showing how they compute on the canonical
elements. By definition of element of a set, we know this is enough.

\subsection{Finite types}

Finite types are defined by the following meta-rules of formation

\[
  \setj{\nty{n}}, \qquad \seteqj{\nty{n}}{\nty{n}}
\]

and introduction (with the obvious equalities):

\[
  \inj{0_n}{\nty{n}}, \; \inj{1_n}{\nty{n}}, \; \dots, \;
  \inj{(n-1)_n}{\nty{n}}
\]

As an example of how elimination rules equip inductive sets with a principle of
induction, to define an operation on arbitrary elements of $\nty{n}$ is to
define it on all possible elements of the set:

\[
  \begin{prooftree}
    \hsetj{C(c)}{c \in \nty{n}}
    \quad \inj{c}{\nty{n}}
    \quad \inj{d_1}{C(0_n)}
    \quad \inj{d_2}{C(1_n)}
    \quad \dots \quad  \inj{d_n}{C((n-1)_n)}
    \justifies
    \inj{E_n(c, d_1, d_2, \dots, d_n)}{C(c)}
  \end{prooftree}
\]

again with the obvious equalities. The computation rule is also straightforward:

\[
  \begin{prooftree}
    \hsetj{C(c)}{c \in \nty{n}}
    \qquad \inj{d_1}{C(0_n)}
    \qquad \inj{d_2}{C(1_n)}
    \qquad \dots \qquad  \inj{d_n}{C((n-1)_n)}
    \justifies
    \eqelj{E_n((i-1)_n, d_1, d_2, \dots, d_n)}{d_i}{C((i-1)_n)}
  \end{prooftree}
\]

Among the finite types, two will be particularly important, hence we abbreviate
them as such

\[
  \onety \equiv \nty{1},\qquad \twoty \equiv \nty{2}
\]

Note that $\onety$ is what is usually called \emph{unit} in programming
languages, whereas $\twoty$ is the boolean type. We also introduce the following
abbreviations for their elements:

\[
  \star \equiv 0_1, \qquad \ffalse \equiv 0_2, \qquad \ttrue \equiv 1_2
\]

\begin{example}
  We now show that type theory, being a theory of \emph{dependent} types, is
  sufficiently powerful to accept programs such as the following as well-typed:

  \[
    \textsf{if}\; b \; \textsf{then}\; \textsf{true}\; \textsf{else}\; 5
  \]

  where $\inj{b}{\twoty}$ and $\inj{5}{\NN}$. Note that in conventional
  programming languages this program whould be ill-typed, since the two branches
  of the conditional statement have different types. Being type theory a far
  more expressive functional programming language, we can express this by saying
  that the return type of the conditional is a family of types depending on
  the value of the input; in particular, the family will map $\ttrue$ to $\twoty$
  and $\ffalse$ to $\NN$. So suppose we have a family $C$ such that $C(\ttrue) =
  \twoty$ and $C(\ffalse) = \NN$. Then:

  \[
    \begin{prooftree}
      \hsetj{C(b)}{\inj{b}{\twoty}} \qquad
      \inj{b}{\twoty} \quad
      \[
        \inj{\ttrue}{\twoty}\quad
        C(\ttrue) = \twoty
        \justifies
        \inj{\ttrue}{C(\ttrue)}
      \]
      \quad
      \[
        \inj{5}{\NN}\quad C(\ffalse) = \NN
        \justifies
        \inj{5}{C(\ffalse)}
      \]
      \justifies
      \inj{E_2(b, \ttrue, 5)}{C(b)}
    \end{prooftree}
  \]
  
\end{example}

\subsection{Natural numbers}

We now move to the most basic example of infinite set, namely the set of
non-negative integers, natural numbers. Infinite sets can be described by
giving a finite number of rules that describe their elements inductively:

\[
  \setj{\NN}, \qquad \seteqj{\NN}{\NN}
\]

\[
  \inj{0}{\NN},\qquad
  \begin{prooftree}
    \inj{n}{\NN}
    \justifies
    \inj{s(n)}{\NN}
  \end{prooftree}
\]

\[
  \eqelj{0}{0}{\NN},\qquad
  \begin{prooftree}
    \eqelj{n}{m}{\NN}
    \justifies
    \eqelj{s(n)}{s(m)}{\NN}
  \end{prooftree}
\]

The elimination rule is essentially a statement of the principle of induction
for natural numbers, or equivalently a rule to define operations by primitive
recursion (but in a more powerful way since the return type can depend on the
argument).

\[
  \begin{prooftree}
    \hsetj{C(n)}{\inj{n}{\NN}} \quad
    \inj{n}{\NN} \quad
    \inj{d}{C(0)} \quad
    \[(\inj{x}{\NN}, \inj{y}{C(x)})\justifies \inj{e(x,y)}{C(s(x))}\]
    \justifies
    \inj{\natrec{n}{d}{(x,y)e(x,y)}}{C(n)}
  \end{prooftree}
\]

The recursion symbol introduced by the elimination rule has a simple semantical
explanation in terms of its computations: it first evaluates $n$, which is known
to yield either $0$ or $s(n)$ for some $\inj{n}{\NN}$. In the first case, it
evaluates $d$ and returns the result. Otherwise, it takes $n$ and feeds it to
$e$ as its first argument, together with the result of evaluating itself on the
predecessor, namely $n$. The second elimination rule shows that the recursor
respects equality in the obvious way:

\[
  \begin{prooftree}
    \hsetj{C(n)}{\inj{n}{\NN}} \quad
    \eqelj{n}{m}{\NN} \quad
    \eqelj{d}{d'}{C(0)} \quad
    \[(\inj{x}{\NN}, \inj{y}{C(x)})\justifies \eqelj{e(x,y)}{e'(x,y)}{C(s(x))}\]
    \justifies
    \eqelj{\natrec{n}{d}{(x,y)e(x,y)}}
    {\natrec{m}{d'}{(x,y)e'(x,y)}}
    {C(n)}
  \end{prooftree}
\]

The equality rules, which are given for each possible way to construct
canonical forms, just show how recursion on the argument works.

\[
  \begin{prooftree}
    \hsetj{C(n)}{\inj{n}{\NN}} \quad
    \inj{d}{C(0)} \quad
    \[(\inj{x}{\NN}, \inj{y}{C(x)})\justifies \inj{e(x,y)}{C(s(x))}\]
    \justifies
    \eqelj{\natrec{0}{d}{(x,y)e(x,y)}}{d}{C(0)}
  \end{prooftree}
\]

\[
  \begin{prooftree}
    \hsetj{C(n)}{\inj{n}{\NN}} \quad
    \inj{d}{C(0)} \quad
    \[(\inj{x}{\NN}, \inj{y}{C(x)})\justifies \inj{e(x,y)}{C(s(x))}\]
    \justifies
    \eqelj{\natrec{s(n)}{d}{(x,y)e(x,y)}}{e(n,
      \natrec{n}{d}{(x,y)e(x,y)}
    }{C(s(n))}
  \end{prooftree}
\]

\begin{example}
  As an example, we can try to define the summation operation on natural
  numbers. Recall that it is defined as follows by primitive recursion:

  \begin{align*}
    \textsf{sum}(0,m) & = m \\
    \textsf{sum}(s(n),m) & = s(\textsf{sum}(n,m))
  \end{align*}

  In type theory it is no different:

  \[
    \begin{prooftree}
      \hsetj{\NN}{\inj{n}{\NN}} \qquad
      \inj{n}{\NN} \qquad
      \inj{m}{\NN} \qquad
      \[(\inj{x}{\NN}, \inj{y}{\NN})\justifies \inj{s(y)}{\NN}\]
      \justifies
      \inj{\natrec{n}{m}{(x,y)s(y)}}{\NN}
    \end{prooftree}
  \]

  Hence $\textsf{sum}(n,m) \equiv \natrec{n}{m}{(x,y)s(y)}$.

\end{example}

\begin{example}
  Multiplication can be similarly defined, in terms of summation:

  \begin{align*}
    \textsf{mul}(0,m) & = 0 \\
    \textsf{mul}(s(n),m) & = \textsf{sum}(m, \textsf{mul}(n,m))
  \end{align*}

  \[
    \begin{prooftree}
      \hsetj{\NN}{\inj{n}{\NN}} \qquad
      \inj{n}{\NN} \qquad
      \inj{0}{\NN} \qquad
      \[(\inj{x}{\NN}, \inj{y}{\NN}, \inj{m}{\NN})\justifies \inj{\textsf{sum}(m, y)}{\NN}\]
      \justifies
      \inj{\natrec{n}{0}{(x,y)\textsf{sum}(m,y)}}{\NN}
    \end{prooftree}
  \]

  Then, $\textsf{mul}(n,m) \equiv \natrec{n}{0}{(x,y)\textsf{sum}(m,y)}$.

\end{example}

\begin{exercise}
  Define the exponentiation operation.
\end{exercise}

\subsection{Disjoint union of a family of sets}

We have seen that the term $\hsetj{b(x)}{x \in A}$ denotes a family of sets
indexed by the elements of the set $A$. We may want to consider the disjoint
union of the sets the constitute the family:

\[
  \begin{prooftree}
    \setj{A} \qquad \hsetj{B(x)}{x \in A}
    \justifies
    \setj{\sigmaab}
  \end{prooftree}
\]

Two disjoint unions are equal when the index set and the family of which we do
the union are equal:

\[
  \begin{prooftree}
    A = C \qquad \hseteqj{B(x)}{D(x)}{x \in A}
    \justifies
    \seteqj{\sigmaab}{\sigmatype{x \in C}{D(x)}}
  \end{prooftree}
\]

The elements of a disjoint union of sets are, intuitively, all the elements from
all the sets that constitute the family, together with the information that
tells us from which set of the family the element comes from. This information
is easily preserved by remembering the index of the sets from which the element
comes from.

\[
  \begin{prooftree}
    \inj{a}{A} \qquad \inj{b}{B(a)}
    \justifies
    \inj{(a,b)}{\sigmaab}
  \end{prooftree}
\]

Equal canonical elements of this set can be formed when the two components are
equal (possibly non-canonical) elements:

\[
  \begin{prooftree}
    \eqelj{a}{c}{A} \qquad \eqelj{b}{d}{B(a)}
    \justifies
    \eqelj{(a,b)}{(c,d)}{\sigmaab}
  \end{prooftree}
\]

The \strong{elimination rule} introduces the eliminator $E$:

\[
  \begin{prooftree}
    \derives{(\inj{z}{\sigmaab})}{\setj{C(z)}} \qquad
    z \in \sigmaab \qquad
    \derives{(x \in A, \; y \in B(x))}{\inj{d(x,y)}{C(\langle x, y \rangle)}}
    \justifies
    E(z,(x,y)d(x,y)) \in C(z)
  \end{prooftree}
\]

We can justify the elimination rule by explaining the semantics of the
eliminator $E$: it first executes $z$, which by hypothesis will yield a
canonical element of type $\sigmaab$, that is, a pair $\pair{a}{b}$ with
$\inj{a}{A}$ and $\inj{b}{B(a)}$. Then, it substitutes $a$ and $b$ for $x$ and
$y$ in $d$. Executing $d(a,b)$ will return a canonical element $e$ of
$C(\pair{a}{b})$. Since $\pair{a}{b}$ is the normal form of $z$, it follows that
$\eqelj{z}{\pair{a}{b}}{\sigmaab}$, so since type families respect equality, we
have that $\inj{e}{C(z)}$, which is what we wanted.

The second elimination rule tells us how the eliminator symbol respects equality
of the entities involved.

\[
TODO...
\]

The equality rule tells us the semantics of the eliminator symbol when acting
of the canonical elements, that is, the pair.

\[
  \begin{prooftree}
    \derives{(\inj{z}{\sigmaab})}{\setj{C(z)}} \qquad
    \inj{a}{A} \qquad \inj{b}{B(a)} \qquad
    \derives{(x \in A,\; y \in B(x))}{d(x,y) \in C(\langle x, y \rangle)}
    \justifies
    E(\pair{a}{b},(x,y)d(x,y)) = d(a,b) \in C(\pair{a}{b})
  \end{prooftree}
\]

The equality rule above is justified by the very definition that we just gave
for the eliminator $E$.

\begin{example}
  

We now define the first and second projection operations as an example of how to
use the rules introduced so far to build operations. The first projection should
be an operation $\textsf{fst}$ such that

\[
  \hinj{\fst{z}}{A}{z \in \bigsigmaty{x\in A}{B(x)}}
\]

hence we use the elimination rule of the disjoint union type to define it:

\[
  \begin{prooftree}
    \setj{A} \quad \inj{z}{\sigmaab} \quad
    (\inj{x}{A})
    \justifies
    E(z,(x,y).x) \in A
  \end{prooftree}
\]

So we put $\textsf{fst}(z) \equiv E(z,(x,y).x)$. We have to check that the
operation we defined has the intended behaviour. This can be done with the
equality rule:

\[
  \begin{prooftree}
    \setj{A} \quad \inj{x}{A} \quad \inj{y}{B(A)} \quad
    (\inj{x}{A})
    \justifies
    E(\pair{x}{y},(x,y).x) = x \in A
  \end{prooftree}
\]
\end{example}

since $((x,y).x)(x) \equiv x$.

\begin{example}
  We now proceed to define the second projection. Note that this is a
  not-so-trivial projection operation, since the result set of this operation
  depends on the first component of the pair:

  \[
    \hinj{\snd{z}}{B(\fst{z})}{z \in
      \bigsigmaty{x\in A}{B(x)}}
  \]

  We now have the following derivation:


  \[
    \begin{prooftree}
      % set family
      \[
        \hsetj{B(x)}{x \in A} \quad
        \derivesfrom{(\inj{z}{\sigmaab})}{\mathcal{D}_1}{\fst{z} \in A}
        \justifies
        \setj{B(\snd{z})}
      \]
      \quad
      % element
      \inj{z}{\sigmaab}
      \quad
      \derives{\mathcal{D}_2}{\inj{y}{B(\fst{\langle x, y \rangle})}}
      \justifies
      E(z,(x,y).y) \in B(\fst{z})
    \end{prooftree}
  \]

  where the derivation $\mathcal{D}_1$ is just the one above, and
  $\mathcal{D}_2$ is as follows:

\[
  \begin{prooftree}
    (\inj{y}{B(x)}) \quad
    \[
      \hsetj{B(x)}{x \in A} \quad
      \[
        \[
          \[
            (\inj{x}{A}) \quad (\inj{y}{B(x)})
            \justifies
            \inj{\langle x, y \rangle}{\sigmaab}
          \]\quad
          (\inj{x}{A})
          \justifies
          \eqelj{\fst{\langle x, y \rangle}}{x}{A}
        \]
        \justifies
        \eqelj{x}{\fst{\langle x, y \rangle}}{A}
      \]
      \justifies
      \seteqj{B(x)}{B(\fst{\langle x, y \rangle})}
    \]
    \justifies
    \inj{y}{B(\fst{\langle x, y \rangle})}
  \end{prooftree}
\]

The computation using the equality rule is showed similarly to the first
projection.
\end{example}

\subsection{Cartesian product of a family of sets}

In the same way as we constructed the disjoint union of a family of sets, we may
as well want to create the cartesian product of a family of sets. In particular,
given a family of sets $B(x) \mathrm{set} (x \in A)$, we write $\setj{\piab}$
for the cartesian product of that family:

\[
  \begin{prooftree}
  A \; \text{set} \quad \hsetj{B(x)}{x \in A}
  \justifies
  \setj{\piab}
  \end{prooftree}
\]

Intuitively, an element of the set $(\Pi x \in A) B(x)$ is a tuple containing
one element for each set in the family $B(x)$. Alternatively, one can think of
such a tuple as an operation associating every element $a$ of the index set $A$
to an element of the set in the family that corresponds to that element, that is
$B(a)$. We can express such \emph{generalized} operation as

\[
  \hinj{b(x)}{B(x)}{x \in A}
\]

Such operations, following our explanation of the set, constitute
essentially the elements of the cartesian product:

\[
  \begin{prooftree}
  \hinj{b(x)}{B(x)}{x \in A}
  \justifies
  \inj{(\lambda x)b(x)}{\piab}
  \end{prooftree}
\]

It is necessary to understand that $\hinj{b(x)}{B(x)}{x \in A}$ is an operation
to be able to form the canonical element
$\inj{(\lambda x)b(x)}{\pitype{x\in A}{B(x)}}$. Since, in general, there are no
exhaustive rules for generating all operations from one set to another, it
follows that the introduction rule above is justified by the fact that we take
the notion of operation to be primitive.

Notice that, in the special case where $x$ is not free in $B$, the family
$\hinj{b(x)}{B}{x \in A}$ is just the usual operation from a set $A$ to a set
$B$. In such cases, we identify the cartesian product with the type of
operations, $\pitype{x\in A}{B(x)} \equiv A \rightarrow B$, and its elements with
the operations from $A$ to $B$.

\begin{example}
  We can now derive the operation that usually comes with the name
  \emph{currying}

  \[
    \textsf{curry} \in (f \in A \times B \rightarrow C) \rightarrow
    (A \rightarrow B \rightarrow C)
  \]

  as follows

  \[
    \begin{prooftree}
      \[
        \[
          \[
            (f \in A \times B \rightarrow C)
            \qquad
            \[
              (\inj{a}{A}) \qquad (\inj{b}{B})
              \justifies
              \inj{\pair{a}{b}}{A \times B}
            \]
            \justifies
            \ap{f}{\pair{a}{b}} \in C
          \]
          \justifies
          \abs{b}{\ap{f}{\pair{a}{b}}} \in B \rightarrow C
        \]
        \justifies
        \abs{a}{\abs{b}{\ap{f}{\pair{a}{b}}}} \in A \rightarrow B \rightarrow C
      \]
      \justifies
      \abs{f}{\abs{a}{\abs{b}{\ap{f}{\pair{a}{b}}}}} \in (f \in A \times B \rightarrow C) \rightarrow
      (A \rightarrow B \rightarrow C)
    \end{prooftree}
  \]

  Then, simply put $ \textsf{curry} \equiv \abs{f}{\abs{a}{\abs{b}{\ap{f}{\pair{a}{b}}}}}$.
\end{example}

%%% Local Variables:
%%% mode: latex
%%% TeX-master: "../notes"
%%% End:
