\section{Subsets}
\subsection{Basic definitions}

Let $X$ be a set.

Consider a unary property (or predicate), that is, a \emph{proper}\footnote{
Recall that a proposition is said to be proper, or small, when all
the \emph{quantifiers} which appear in it range only on sets.
} proposition \(D(x)\) depending on \emph{one} argument in \(X\).

Then,  \(D\) \emph{is} a \newdefn{subset} of \(X\).

If \(D\) is a subset of \(X\) and \(x \in X\), we say that \(x\) is an
\textbf{element} of \(D\), written \(x \epsilon_X D\), if \(x\)
satisfies the property \(D\).

The subscript \(X\) will be implicit most of the time.

\begin{example}
  Let, for example, be \(S\) the students of the University of Padua
  and \(M(s)\) - a \emph{proper} proposition depending on \emph{one}
  argument in \(S\), where \(M(s)\) if \(s\) is studying for a degree
  in Mathematics.
  
  We will write \(\text{Marco} \epsilon_S M\) to denote that Marco is
  indeed pursuing a degree in Mathematics - and \(M\) also models the
  notion of the subset of mathematics students.
\end{example}

\begin{example}
  Let $C$ be a collection.
  
  Consider $L(s) = 1$ iff $s$ is friends with every element of $C$.

  $L$ is not a subset (it's not proper).
\end{example}


Do take notice how we are using the notation \(\insub\) instead of the
more familiar \(\in\). The reason will be apparent presently.

\paragraph{Equality} Since \(D(x)\), just like any proposition (or, in
this case, propositional family on X) respects equality of \(X\),
membership in \(D\) also necessarily respects it:

\[ \forall (x, y \in X) : (\injsub{x}{D} \wedge x\ =_X\ y) \Rightarrow \injsub{y}{D} \].

Moreover, two subsets \(D, E\) are equal when an element \(x \in X\)
is in \(D\) iff is in \(E\):

\[ \forall (D, E\ \text{subset}\ X) : (\forall x \in X)(
    \injsub{x}{D} \Leftrightarrow\ \injsub{x}{E}) \Leftrightarrow D = E \]

Clearly $\injsub{x}{D} \Leftrightarrow \injsub{x}{E}$ arises from
$D(x) \Leftrightarrow E(x)$.
    
The collection of subsets of \(X\), with this notion
of equality, is denoted by \(\setpow{X}\) and called the power
collection (\strong{not} powerset, for reasons that will soon be
apparent) of \(X\).

\subsection{Properties (and un-properties) of subsets}

We now enumerate some properties of subsets thus defined; we also
highlight some properties which are usual in ZFC and no longer hold
under the definition\footnote{It is to do a disservice to the subject
  to define it in terms of \emph{how it differs from ZFC}.  However,
  the intended audience of the series of lectures that this document
  accompanies is one mostly or exclusively familiar with ZFC and, all
  too often, unaware of it. We try to eliminate misconceptions
  first.}.

?? further expands on the topic and its philosophical motivations.
% TODO: cite sambin/valentini

The first un-property is:

\[ \text{No subset can be \emph{assumed} to exist.} \]
% TODO: use something better than an equation environment...

A subset arises when we can define it by a \emph{property}.  If no
property is given, no subset can be drawn out of thin
air\footnote{Note that in classical (ZFC-based) mathematics we can
  conceive the existence ``of more subsets than propositions'', if we
  assume propositions to be formulas over a finite formula.  This is
  not the case here; one proposition, one subset.}.

The second is:

\[ \text{Subsets are not sets.} \]

Subset and set are different concepts altogether.

This is immediate from the definition: the logical type of a subset
is that of a propositional \emph{family} over a set, which is
different from a set.

In fact we can also write \(\injsub{y}{D}\) as
\(y \in \{x \in X : D(x)\}\); this is why we choose to use \(\insub\)
for subsets.

Besides the fact that we've-defined-them-different-period, there is a
profound reason why we can't allow subsets to be sets in the familiar
way.

In ZFC-based mathematics, subsets are, of course, sets - by the
\strong{axiom of separation}, which states that given any set $A$,
there is a set $B$ such that, given any $x$, is a member of $B$ if and
only if $x$ is a member of $A$ and some property $\varphi$ holds for $x$.

In other words, the axiom of spearation allows to define
$B \subseteq A$ by ``filtering'' elements of $A$ by some property
$\varphi$.

Allowing the axiom of separation in our theory would give rise to
severe inconsistencies.

Consider the property

\[ U(x) = \text{``the Turing machine with index
  $x$ does not stop on input $x$''} \]

It's a well known result of computability theory that there are no
effective rules to generate $\{\inj{x}{N} : U(x) \}$.

Therefore, if we did allow the axiom of separation, we would be faced
with the fact that the subset $U$ of $N$ is also a set, but one the
elements thereof are impossible to enumerate, in open contrast with
the constructive principles we've so far built our theory upon.

Notably, there is no notion of a \strong{complement subset}.

Notice how being able to decide \(\injsub{y}{\complset{D}}\) for
some complement subset \(\complset{D}\) of \(D\) would essentially
entail deciding that \(y \in \{ x \in X : \neg D(x) \}\) on the
basis of \(D(x)\) \emph{not} holding.

Similarly, by admitting \(\complset{\complset{D}}\) we would have 
\(\injsub{y}{\complset{\complset{D}}} = \injsub{y}{\{ x \in X : \neg\neg D(x) \}}\).

This would amount to include the law of excluded middle or double negative
elimination in our theory.

\begin{example}[Not innocent until proven innocent]
  This example, which may or may not be satirical in intent, is due
  to %TODO cite bptools

  Let \(R\) be the set of residents of Italy.
  
  \[ H(x) = x\ \text{is honest} \]
  
  \(H \subseteq R\) is the subset of honest residents.

  \(\inj{r}{R}\) not being exposed as dishonest is not sufficient to say that
  \(r \epsilon H\):
  
  \[ \notinjsub{x}{\complset{H}} \not\Rightarrow \injsub{r}{H} \]
\end{example}

It is now apparent why, whereas in ZFC we have the \strong{power set
  axiom}, by which the power set is itself a set, it would be
inadmissible here.

There is further reason why we are armed with a humble power
\emph{collection}: it can be shown by diagonalization that it is
impossible to generate all subsets of a set by a fixed finite number
of rules.
% TODO elaborate

\subsection{Quantification}
%TODO what should this subsection be called?
Before moving on, we introduce some notation for later.

Let \(\varphi\) be a proposition depending on an argument in \(X\).

We will write \((\exists \injsub{x}{D})\varphi\) in lieu of
\((\exists \inj{x}{X})(\injsub{x}{D} \wedge \varphi)\) and
\((\forall \injsub{x}{D})\varphi\) in lieu of
\((\exists \inj{x}{X})(\injsub{x}{D} \rightarrow \varphi)\).

As we've seen, if \(\varphi\) is unary on \emph{one} argument \(x\)
\emph{and} proper, then \(E \equiv \{\inj{x}{X} : \varphi(x) \}\) is a
subset.

In this case we will also write \(D \between E\) in lieu of
\((\exists x \epsilon D)\varphi\) and \(D \subseteq E\) in lieu of
\((\forall x \epsilon D)\varphi\)


\subsection{Relations and functions}
Until now we've dealt with \emph{unary} propositions.

A \newdefn{relation} $R$ between $X$ and $S$ is a \emph{binary} proper
proposition, i.e. depending on two arguments:

\[ \hpropj{R(x, y)}{x \in X, y \in Y} \]

the order of arguments is relevant, in general, and thus we say that
$R$ is a relation \emph{from} $X$ \emph{to} $S$ and write
$R : X \rightarrow S$.

From equality of propositions, we have that a relation preserves
equality as well:

\[ x =_X x' \wedge a =_S a' \Rightarrow R(x, a) \leftrightarrow R(x' ,
  a' )\]

A relation $R$ from $X$ to $S$ is said to be \newdefn{total} if every
element \(x \in X\) is related to at least one element \(a \in S\),
that is, if \((\forall x \in X)(\exists a \in S)R(x, a)\) holds.

A relation is also said to be \newdefn{single-valued} if every
\(x \in X\) is related to at most one element in $S$.

A relation is called a \newdefn{function} if it is total and single-valued.

\begin{example}
  \[ R : N \rightarrow N \]
  \[ R(x,y) \equiv  x \geq y \] 
  Is total, not single valued, not a function.
\end{example}

\begin{example}
  \[ R : N \rightarrow N \]
  \[ R(x,y) \equiv (x = 2y) \]
  Is single valued but not total; not a function.
\end{example}

\begin{example}
  \[ R : N \rightarrow N \]
  \[ R(x,y) \equiv (x = y) \]
  Is a genuine function.
\end{example}

\begin{example}[Fake example!]
  \[ R : N \rightarrow N \]
  \[ R(x,y)\ \equiv\ \forall s \in P(N) \top \]
  $R$ is not a proper proposition, therefore not a relation!
\end{example}

%\begin{example}
%  \[ R(x,y) : x, y \in N \]
%  \[ R(x,y)\ \text{iff}\ \forall s \in P(N),  \{x, y\} \subseteq s \]
%TODO: this one is a relation, why?
%\end{example}

The \newdefn{collection of relations} from $X$ to $S$ is denoted by
$Rel(X, S)$.  The \newdefn{collection of functions} from \(X\) to
\(S\) is denoted by \(Fun(X, S)\).

Consider a family \(p(i) \in X (i \in I)\).  The subset of X defined
by \((\exists i \in I)(x =_X p(i))\) is called the \newdefn{image} of
\(I\) along \(p\).

As subsets of a set, also relations and functions between two sets
form a collection, and not a set, since there are no rules to
generate all of them.

\subsubsection{Operations and functions}

Every operation \(p\) from a set \(X\) to a set \(S\) determines a
function \(\hat{p} : X \rightarrow S\), called the \newdefn{graph} of
\(p\), s.t. .  $\forall x \in X, a \in S$:

\[ \hat{p}(x,a) \equiv (a =_S p(x))\]

While every operation determines a function, it is \strong{not} the
case that every function determines an operation.

Our functions are total and single-valued relations, and are in this
respect not unlike plain old ZFC functions.

Recall that our ``operations'' are computationally effective, explicit
procedures\footnote{To reinforce this we go as far as choosing the
  letter $p$ to suggest ``program''}.

It is a classic result of computability theory that there are
functions (called uncomputable functions) for computing which there
does not exist an effective operation or algorithm - therefore not all
functions admit an operation.

Note, moreover, that, where one operation exists, it is unique.
Operations are extensionally equal - i.e., they are equal if and only
if they compute the same function.

%%% Local Variables:
%%% mode: latex
%%% TeX-master: "../notes"
%%% End:
