\section{Subsets}
\subsection{Basic definitions}

\begin{definition}[Subset]
  Subsets and their elements are defined as follows:
  \begin{enumerate}
  \item Let $X$ be a set, and $D$ a unary property (or predicate), that is, a
    \emph{proper}\footnote{Recall that a proposition is said to be proper, or
      small, when all the \emph{quantifiers} which appear in it range only on
      sets.} proposition depending on \emph{one} argument in \(X\).
    Then, \(D\) \emph{is} a \newdefn{subset} of \(X\).
  
  \item If \(D\) is a subset of \(X\) and \(\inj{x}{X}\), we say that \(x\) is
    an \textbf{element} of \(D\), written \(\injsubexpl{x}{X}{D}\), if \(x\)
    satisfies the property \(D\).
  \end{enumerate}
\end{definition}

The subscript in a subset membership judgement will be implicit most of the
time. Do take notice how we are using the notation \(\insub\) instead of the
more familiar \(\in\). The reason will be apparent presently.

\begin{example}
  Let, for example, be \(S\) the students of the University of Padua
  and \(M(s)\) - a \emph{proper} proposition depending on \emph{one}
  argument in \(S\), where \(M(s)\) if \(s\) is studying for a degree
  in Mathematics.
  
  We will write \(\injsubexpl{\text{Marco}}{S}{M}\) to denote that Marco is
  indeed pursuing a degree in Mathematics - and \(M\) also models the
  notion of the subset of mathematics students.
\end{example}

\begin{example}
  Let $C$ be a collection.
  Consider $L(s) \equiv$ ``$s$ is friends with every element of $C$''.
  $L$ is not a subset (it's not proper).
\end{example}

Since \(D(x)\), just like any proposition (or, in this case, propositional
family on X) respects equality of \(X\), membership in \(D\) also necessarily
respects it:

\[
  \forallconn{x,y}{X}{(\implconn{\injsub{x}{D} \wedge x\ =_X\ y}{\injsub{y}{D}})}
\].

This leads to a natural formulation of equality between subsets.

\begin{definition}[Subset equality]
  Two subsets \(D, E\) of a set $X$ are equal when an element \(\inj{x}{X}\) is
  in \(D\) iff is in \(E\):

  \[
    \forall D, E \subseteq X,
    \implconn{\forallconn{x}{X}(\injsub{x}{D} \iff \injsub{x}{E})}{D = E}
  \]  
\end{definition}

The collection of subsets of \(X\), with this notion of equality, is denoted by
\(\setpow{X}\) and called the power collection of $X$ (\strong{not} powerset,
for reasons that will soon be apparent) of \(X\).

Before moving on, we introduce some notation for later.

\begin{notation}
  Let \(\varphi\) be a proposition depending on an argument in a set \(X\).

  \begin{enumerate}
  \item We will write \((\exists \injsub{x}{D})\varphi(x)\) in lieu of
    \((\exists \inj{x}{X})(\injsub{x}{D} \wedge \varphi(x))\) and
    \((\forall \injsub{x}{D})\varphi(x)\) in lieu of
    \((\exists \inj{x}{X})(\injsub{x}{D} \rightarrow \varphi(x))\);
  \item If \(\varphi\) is unary \emph{and} proper, then
    \(\{\inj{x}{X} : \varphi(x) \}\) is a subset;
  \item We write \(D \between E\) in lieu of \((\exists \injsub{x}{D})E(x)\)
    and \(D \subseteq E\) in lieu of \((\forall \injsub{x}{D})E(x)\).
  \end{enumerate}
\end{notation}

\subsection{Properties (and un-properties) of subsets}

We now enumerate some properties of subsets thus defined; we also
highlight some properties which are usual in ZFC and no longer hold
under the definition\footnote{It is to do a disservice to the subject
  to define it in terms of \emph{how it differs from ZFC}.  However,
  the intended audience of the series of lectures that this document
  accompanies is one mostly or exclusively familiar with ZFC and, all
  too often, unaware of it. We try to eliminate misconceptions
  first.}.
% TODO: further expand on the topic and its philosophical motivations.
% TODO: cite sambin/valentini
The first un-property is:

\[ \text{No subset can be \emph{assumed} to exist.} \]
% TODO: use something better than an equation environment...

A subset arises when we can define it by a \emph{property}. If no property is
given, no subset can be drawn out of thin air
\footnote{Note that in classical (ZFC-based) mathematics we can conceive the
existence ``of more subsets than propositions'', if we assume propositions to be
formulas over an alphabet with countably many symbols. This is not the case
here; one proposition, one subset.}.
Moreover, subsets of a set do not form themselves a set, but only a collection,
since it can be shown by diagonalization that it is impossible to
generate all subsets of a set by a fixed finite number of rules.
This is in contrast with the \strong{powerset axiom} of classical set theory,
which postulates the existence of a set of all subsets of any given set.

The second is:

\[ \text{Subsets are not sets.} \]

Subset and set are different concepts altogether.
This is immediate from the definition: the logical type of a subset
is that of a propositional \emph{family} over a set, which is
different from a set.
In fact we can also write \(\injsub{y}{D}\) as
\(y \in \{x \in X : D(x)\}\); this is why we choose to use \(\insub\)
for subsets.

Besides the fact that sets and subsets are different by definition, there is a
profound reason why we can't allow subsets to be sets in the familiar way.
In ZFC-based mathematics, subsets are, of course, sets - by the
\strong{axiom of separation}, which states that given any set $A$,
there is a set $B$ such that, given any $x$, is a member of $B$ if and
only if $x$ is a member of $A$ and some property $\varphi$ holds for $x$.
In other words, the axiom of spearation allows to define
$B \subseteq A$ by ``filtering'' elements of $A$ by some property
$\varphi$.

Allowing the axiom of separation in our theory would give rise to
severe inconsistencies.
Consider the property

\[ U(x) = \text{``the Turing machine with index
  $x$ does not stop on input $x$''} \]

It's a well known result of computability theory that there are no effective
rules to generate the elements of $\{\inj{x}{N} : U(x) \}$.  Therefore, if we
did allow the axiom of separation, we would be faced with the fact that there
are sets for which we do not have rules to effectively construct its canonical
elements, in open contrast with the constructive principles we've so far built
our theory upon.

\[ \text{There is no notion of a}\ \strong{complement subset}.\]

Or at least, the usual notion of complement in classical set theory does not
find any correspondence in this setting. In fact, \(\complset{\complset{D}}\) is
just \(\{ x \in X : \neg\neg D(x) \}\). Proving $D = \complset{\complset{D}}$
would require the classical law of double negation, or equivalent principles.

\begin{example}
  This example, which may or may not be satirical in intent, is due to
  \cite{basicpicture}.

  Let \(R\) be the set of residents of Italy.
  
  \[ H(x) \equiv \text{``}x\ \text{is honest''} \]
  
  \(H \subseteq R\) is the subset of honest residents.
  \(\inj{r}{R}\) not being exposed as dishonest is not sufficient to say that
  \(\injsub{r}{H}\):
  
  \[ \notinjsub{x}{\complset{H}} \not\Rightarrow \injsub{r}{H} \]
\end{example}

\subsection{Relations and functions}
Until now we've dealt with \emph{unary} propositions.

\begin{definition}[Relation]
  A \newdefn{relation} $R$ between $X$ and $S$ is a \emph{binary} proper
  proposition, i.e. depending on two arguments:

  \[ \hpropj{R(x, y)}{x \in X, y \in Y} \]
\end{definition}

the order of arguments is relevant, in general, and thus we say that
$R$ is a relation \emph{from} $X$ \emph{to} $S$ and write
$R : X \rightarrow S$.

From the fact that propositions preserve equality, we have that a relation
preserves equality as well:

\[
  \implconn{
    \andconn{x =_X x'}{a =_S a'}
  }{
    (R(x, a) \iff R(x' , a'))
  }
\]

\begin{definition}[Functional relation]
  Let $X,S$ be sets, and $R$ a relation from $X$ to $S$.
  
  \begin{enumerate}
  \item $R$ is \newdefn{total} if every element \(x \in X\) is related to at
    least one element \(a \in S\), that is, if
    \((\forall x \in X)(\exists a \in S)R(x, a)\) holds.

  \item $R$ is \newdefn{single-valued} if every \(x \in X\) is related to
    at most one element in $S$, that is, if

    \[
      \forallconn{x}{A}{
        \forallconn{y,z}{A}{
          (\implconn{
            \andconn{R(x,y)}{R(x,z)}
          }{
            y = z
          })
        }
      }
    \]

    holds.

  \item $R$ is called a \newdefn{function} if it is total and single-valued.
  \end{enumerate}
\end{definition}

\begin{example}
  Consider relations $R$ over the set of natural numbers.

  \begin{enumerate}
  \item $R(x,y) \equiv  x \geq y$ is total, not single valued, not a function;
  \item $R(x,y) \equiv (x = 2y)$ is single valued but not total; not a function;
  \item $R(x,y) \equiv (x = y)$ is total and single valued, hence a function;
  \item $R(x,y)\ \equiv\ \forall s \in \mathcal{P}(\NN) \top$ is not a proper proposition,
    therefore not a relation!
  \end{enumerate}
\end{example}

\begin{example}
  Notice that there is an obvious correspondence between subsets of the
  cartiesian product $A \times B$ between two sets $A,B$ (which is itself a set)
  and relations from $A$ to $B$. In particular

  \begin{enumerate}
  \item From a subset $\hpropj{D(z)}{\inj{z}{A\times B}}$, we can define the
    corresponding relation $R(x,y) \equiv D(\pair{x}{y})$;
  \item From a relation $\hpropj{R(x,y)}{\inj{x}{A},\inj{y}{B}}$, we can define
    the corresponding subset $D(z) \equiv R(\fst{z},\snd{z})$.
  \end{enumerate}

  Moreover, from the induction principle of the cartesian product, it is
  actually possible to prove
  $\forallconn{x}{A}{\forallconn{y}{B}{(R(x,y) \iff D(\pair{x}{y}))}}$.
\end{example}

%\begin{example}
%  \[ R(x,y) : x, y \in N \]
%  \[ R(x,y)\ \text{iff}\ \forall s \in P(N),  \{x, y\} \subseteq s \]
%TODO: this one is a relation, why?
%\end{example}

The \newdefn{collection of relations} from $X$ to $S$ is denoted by
$Rel(X, S)$.  The \newdefn{collection of functions} from \(X\) to
\(S\) is denoted by \(Fun(X, S)\).

Consider an operation $\hinj{p(i)}{X}{\inj{i}{I}}$. The subset of $X$ defined
by $\existsconn{i}{I}{(x =_X p(i))}$ is called the \newdefn{image} of \(I\)
along \(p\).

As subsets of a set, also relations and functions between two sets
form a collection, and not a set, since there are no rules to
generate all of them.

\subsubsection{Operations and functions}

Every operation \(p\) from a set \(X\) to a set \(S\) determines a
function \(\hat{p} : X \rightarrow S\), called the \newdefn{graph} of
\(p\), s.t. .  $\forall x \in X, a \in S$:

\[ \hat{p}(x,a) \equiv (a =_S p(x))\]

While every operation determines a function, it is \strong{not} the case that
every function determines an operation, since defining an operation $p$ requires
to be able to give instructions to yield the output $p(x)$ from a given argument
$x$
\footnote{To reinforce this we go as far as choosing the letter $p$ to suggest
  ``program''}.
It is a classic result of computability theory that there are
functions (called uncomputable functions) for computing which there
does not exist an effective operation or algorithm. Therefore not all
functions admit an operation.

Note, moreover, that as the theory we are describing is extensional, also
equality between operations is extensional:

\[
  p = q \iff \forallconn{x,y}{X}{p(x) = q(x)}
\]

\subsection{Operations on subsets}

We've introduced operations as an effective sequence of instructions.
We now discuss operations on subsets and their properties.

\begin{definition}[Injective operation] We call an operation
  \newdefn{injective} or \emph{one-one} iff \(O\ p =_Q O\ p'\) implies \(p
  =_P p'\).
\end{definition}

This is the familiar definition of injectivity from classical
mathematics.

\begin{definition}[Surjective operation] We call an operation
  \newdefn{surjective} when \((\forall q : Q)(\exists p : P) (Op =_Q
  q)\)
\end{definition}

% If \(P\) is a set we say that \(Q\) is indexed by a set \(P\).
First, recall that operations on collections are \emph{compositional}:
% TODO "recall", but has it been introduced before?

for a pair of operations

\[ O\ p: Q(p:P) \]
\[ O'\ q: Q'(q:Q) \]

we can define their composition as

\[ \compos{O'}{O' p} \equiv O'(O(p)) : Q'(p : P) \]

We say an operation $O$ is ``\newdefn{strongly onto}'' when there
exists an operation \(O'\) such that \(\compos{O'}{O} = \ident_Q\).

We keep the two separate because we don't assume the axiom of choice.

\begin{definition}[Bijective operation] Finally, we call an operation
  \newdefn{bijective} when it's injective and strongly onto.
\end{definition}

% Todo explain how surjective is not enough

\footnote{ One may wonder whether there is such a notion as ``strongly
  1:1''.

  % TODO: something? something what?
  
  In category theory something that satisfies \(\compos{O'}{O} =
  \ident_P\) is called \strong{split mono} and \strong{split epi} something that
  that satisfies \(\compos{O}{O'} = \ident_Q\).

  We avoid this choice of naming, as we don't want to suggest we are
  using cat theory.

  However, our notion of ``strongly onto'' is akin to the notion of
  split epi in cat theory.

  Therefore, the notion of \newdefn{strongly 1:1} could be the
  property of satisfying \(\compos{O}{O'} = \ident_Q\),
  \(\compos{O'}{O} = \ident_P\) }


It is worth noting how being a bijection implies being 1:1 and onto,
but not the other way around.

We call "bijection" something for which we have an \emph{effective}
inverse operation.

This is not the case with the ``classical'' definition of
surjectivity, which assumes there \emph{is} a way back by AC!

% TODO: explain \(\exists x ( D = \{x \})\) thing

\subsubsection{Operations and relations} % TODO merge with previous
subsubsection about operations and relations?

An important example of a bijection is the one that holds between the
collection of relations from $X$ to $S$ and the collection of
operations from $X$ to $\setpow{S}$:

\[ \Rel{X}{S} \sim \Op{X}{\setpow{S}} \]

We show how it can be constructed in each direction.

\begin{enumerate}
\item First we see how an element of $\Rel{X}{S}$ can be constructed
  from an element of $\Op{X}{\setpow{S}}$.

  An operation from $X$ to $\setpow{S}$ is family of subsets - for
  each $x \in X$ there is a unary proposition \(U_x\) which yields a
  subset:

  \[ U_x \subseteq S (x \in X) \]

  Recall that a relation from $X$ to $S$ is a \emph{binary}
  proposition with arguments $x$ and $a$.

  We can always construct $R(x,a)$ in the following way:

  \[ R(x,a) \equiv U_x(a) \equiv \injsub{a}{U_x} \]
  \[ R(x,a) \leftrightarrow \injsub{a}{\{\inj{b}{S} : R(x,b)\}} \]
  Note how since \(U_x\) only depends on \(a\), for all \(U_x \subseteq
  S(x \in X)\) we have a different operation.

  Because for every element of the set the operation yields a
  different subset of \(S\) i.e. a family of subset indexed by \(x\)

  Thus can always find a relation between set \(x\) and set \(S\).

\item To go the opposite way and construct $U_x$ from \(R(x,a)\) we
  can define

  \[ U_x \equiv \{ a \in S : R(x,a) \} \]

  For some fixed \(x\) we can now costruct an effective procedure to
  decide whether $a$ belongs to $U_x$.  % TODO: was this it?
\end{enumerate}

Notice how this gives a bijection, because both compositions of the
definitions given above give identity.

We will therefore write $x\ r\ a$ instead of $R(x,a)$.

Wi will also write $r\ x$ in lieu of $\{a \in S : x\ r\ a\}$.

\begin{definition}[Inverse] We define the \newdefn{inverse} relation
  of $r$ as $\invr{r}$ su that
  \[ a\ \invr{r}\ x \equiv x\ r\ a \]
\end{definition}

\begin{definition}[Finer relation] We say that a relation $r$ is
  \newdefn{contained} or \newdefn{finer} than $r'$ and write \(r
  \subseteq r'\) if and only if
  \[ x\ r\ a \rightarrow x\ r'\ a \]

\end{definition}

% TODO what is this?
% \(rx \subseteq r'x\)
% \((r^-) \equiv r \Longrightarrow Rel(x,s) \equiv Rel(s,x)\)

\paragraph{Currying} From the computer scientist's perspective it's
worth noting how currying\footnote{ Recall that the currying of a
  function
  \[ f : A \times B \rightarrow C \] results in:
  \[ curry(f): A \rightarrow (B \rightarrow C) \] } can be used to
show

\[ Rel(X,S) : X \rightarrow \setpow{S} \]

Firstly,

\[ \Rel{X}{S} \sim X \times S \rightarrow \hprop \]

by definition of relation.  By currying this is an isomorphism of

\[ X \rightarrow S \rightarrow \hprop \]

In turn, since subsets of $S$ have type $Prop \rightarrow S$,

\[ X \rightarrow S \rightarrow \hprop \sim X \rightarrow \setpow{P}{S}
  \sim \Op{X}{\setpow{S}} \]

\begin{definition}[Composition of relations] For two relations \(r : X
  \rightarrow Y\) and \(r : Y \rightarrow Z\) we can define composition
  as

  \[ x\; (\compos{r'}{r})\; z \equiv (\exists y \in P)(x\; r\; y \wedge y\;
    r'\; z) \]
\end{definition}

% Do note that we can't extend this definition to collections because?
% TODO


% Recall that, if \(D,E \subseteq X\) are subsets, equality is defined
%as \((\forall x \in X) D(x) %\Leftrightarrow E(x) \equiv D=E\).
% 
% \begin{definition}[Equivalence of relations]
%   TODO
%   \(\exists y
%(\injsub{y}{rx} \wedge \injsub{y}{r'^- z}\) \(x (r' \cdot r) z \equiv
%r x \between r'^-z\)
% \end{definition}

\subsection{Union, intersection and overlap of subsets}

\paragraph{Empty and total subset} The empty $\emptyset$ and total set
$X \subseteq X$ can be defined as, respectively,

\[ \emptyset \equiv \{ x \in X : \bot \}= \{ x \in X : \neg(x = x) \}
  = \{ x \in X : 0 \neq 1 \} = \ldots \]

and

\[ X \equiv \{ x \in X : \top \} = \{ x \in X: x = x \} = \{ x \in X :
  1= 0 \} = \ldots \]

We also define:

\begin{definition}[Union of relations]
  \[ D \cup E \equiv \{ x \in X : D(x) \vee E(x) \} \]
\end{definition}

\begin{definition}[Intersection of subsets]
  \[D \cap E \equiv \{ x \in X : D(x) \wedge E(x) \}\]
\end{definition}

\begin{definition}[Overlapping subsets]
  \[ D \between E \equiv (\exists x \in X)(D(x) \wedge E(x)) \]
\end{definition}

A few remarks are necessary wrt subset overlap.

\begin{enumerate}
\item Note that defining overlap as \(\neg(\forall x \in X)\neg(D(x)
  \wedge E(x))\) would not make sense.  It is not a constructive
  definition - it doesn't produce a witness.  \(\neg(\forall x \in
  X)\neg(D(x) \wedge E(x))\) essentially means \(D \cap E \neq
  \emptyset\); from an intuitionistic perspective \(D \cap E \neq
  \emptyset\) and \(D \between E\) are two different things (as usual,
  having equality between the two would boil down to have the rule of
  double negation).
\item By replacing \(\exists\) with \(\forall\) and \(\wedge\) in the
  definition of overlap we have the definition of inclusion ( \((\forall
  x \in X) (D(x) \Rightarrow E(x))\)) once more.
\end{enumerate}


\begin{definition}[Union and intersection of families of subsets]

  Let \(D \subseteq X (i \in I)\); the generalized union and
  intersection are the operations s.t.:

  \[ \injsub{x}{\bigcup_{i\in I} D_i} \Leftrightarrow\injsub{x}{D_i}
    \exists i \in I \]
  \[ \injsub{x}{\bigcap_{i\in I} D_i} \Leftrightarrow \injsub{x}{D_i}
    \forall i \in I \]

  Their definition is as follows:

  \[ \bigcup_{i \in I} \equiv \{ x \in X : (\exists i \in
    I)(\injsub{x}{D_i}) \} \]

  \[ \bigcap_{i \in I} D_i \equiv \{ x \in X (\forall i \in
    I)(\injsub{x}{D_i}) \} \]
\end{definition}

Two important properties arise:

\begin{enumerate}
\item \(\bigcup_{i\in I} D_i \subseteq E\) iff for all \(i \in I\)
  \(D_i\subseteq E\)
\item \(E \subseteq \bigcap_{i\in I} D_i\) iff for all \(i\in I\) \(E
  \subseteq D_i\)
\end{enumerate}

Moreover, the property of distributivity follows:

\[ (\cup_{i\in I} D_i) \cap E = \bigcup_{i\in I} (D_i \cap E)\]
% \paragraph{Union over families of subsets}
% 
% \[ \cup_{\injsub{j}{J}} D_j = \{ x \in X : (\exists
% \injsub{j}{J})(\injsub{x}{D_j}) \} \]
% 
% TODO \((\cup_{\injsub{j}{J}} D_j === ????\)
% TODO i missed something here

\subsection{Singletons}
\begin{definition}[Singletons] For a set $X$ for which equality $=_X$
  is defined, we define singletons as subsets with exactly one element,
  and write

  \(\{ x\}\equiv \{ y \in X : y = x \}\)

\end{definition}

% TODO: i missed something here % They enjoy the following properties:

% \begin{itemize}
% \item \(\exists! x (\injsub{x}{D})\)
% \item \(\exists x (\injsub{x}{D})\)\item \(\forall x,y \varepsilon
% D)(x = y)\)
% \item \(y \in \{ x \}\) sse \(y = x\) sse \(x \in \{ y \}\) sse
% \(\{x\}\subseteq \{y\}\) sse \(\{x\} \between \{y\}\) sse \(\{ x \} =
% \{ y \}\)
% \end{itemize}


%%% Local Variables:
%%% mode: latex
%%% TeX-master: "../notes"
%%% End:
