\section{Propositions}

The notion of proposition is open. In particular, it does not coincide with the
traditional notion of formula in a fixed formal language. A proposition is any
expression (in any language) which can be asserted to hold, or to be true, and
for which we understand what is required to make it true. A proposition may
depend on a number of variables, or arguments, each variable ranging in its
specific domain, that can be either a set or a collection. In the latter case
actually we should use the term \textit{propositional family} since a
proposition is a \textit{proposition} only when each argument is specified in
the actual domain.

Of course, once one has fixed a domain, a proposition $\psi(x)$ must respect
equality, as stated by the following substitution rule:

\[
  \begin{prooftree}
    \psi(x)\quad
    x =_{X} y
    \justifies
    \psi(y)
  \end{prooftree}
\]

Actually, this equality constraint is a very good instance of a propositional
family. Here, for every set $X$, \textit{x $=_{X}$ y} is a propositional family
with two arguments. Fixing $x$ and $y$ leads to a proposition.

As usual, we may compose different propositions to obtain new ones. That is made
possible by using the usual \textit{logical connectives} and
\textit{quantifiers}, i.e. the following:

\begin{itemize}
\item \textbf{Connectives}:
  \begin{itemize}
  \item $\wedge$, or \textbf{and, conjunction}
  \item $\vee$, or \textbf{or, disjunction}
  \item $\supset$, or \textbf{implies, implication}
  \item $\neg$, or \textbf{not, negation}
  \end{itemize}
\item \textbf{Quantifiers}:
  \begin{itemize}
  \item $\exists$, or \textbf{there exists, existential}
  \item $\forall$, or \textbf{for all, universal}
  \end{itemize}
\end{itemize}

And we may eventually even use induction, as specified by this formation rule:

\[
  \begin{prooftree}
    a \in S \quad
    U : \mathcal{P}S
    \justifies
    R(a,U)
  \end{prooftree}
\]

Where S is supposed to be a set and \textit{U : $\mathcal{P}$S} means U is in
the collection $\mathcal{P}$S obtained from the set S.

Classical logic is based on the notion of truth. Everything is either
\textit{true} or \textit{false} (i.e. \textit{not true}), and the truth of a
statement is ''absolute'', in the meaning it is independent of any
reasoning. Contrariwise, in \textbf{intuitionistic logic} judgments about
statements are no longer based on any predefined value of that statement, but on
the existence of a proof or ''construction'' of that statement. That is, a
propositions ''holds'' if exists a proof, constructed by us, of that
proposition. Under this completely different point of view the following rules
explain the informal constructive semantics of propositional connectives and
quantifiers:

\begin{itemize}
\item A construction of $\psi\wedge\varphi$ consists of a construction of $\psi$
  and a construction of $\varphi$;
\item A construction of $\psi_{1} \vee \psi_{2}$ consists of a number
  $i \in {1, 2}$ and a construction of $\psi_{i}$;
\item A construction of $\psi_{1} \supset \psi_{2}$ is a method transforming
  every construction of $\psi_{1}$ into a construction of $\psi_{2}$;
\item There is no possible construction of $\perp$ (where $\perp$ denotes
  falsity);
\item From $\perp$ one may infer everything;
\item A construction of $\neg\psi$ is a method that turns every construction of
  $\psi$ into a non-existent object. That is, actually $\neg\psi$ is just an
  abbreviation of an implication $\psi \supset \perp$;
\item One may infer $\psi$ from $\chi\wedge\varphi$ iff one may infer
  $\varphi \supset \psi$ from $\chi$;
\item A construction of $(\varphi \vee \psi) \supset \chi$ is a construction of
  $\varphi \supset \chi$ and $\psi \supset \chi$;
\item For any set $X$, a construction of $(\exists x \in X)\psi(x)$ exists if
  and only if there exists (in the sense that one can produce) an element
  $d \in X$ such that $\psi(d)$ holds;
\item For any set $X$, a construction of $(\forall x \in X)\psi(x)$ exists if
  and only if $\psi(d)$ holds for every $d \in X$;
\item For any collection $\mathcal{P}$, a construction of
  $(\exists p : P)\psi(p)$ exists if and only if there exists an object
  $q : \mathcal{P}$ such that $\psi(q)$ holds;
\item For any collection $\mathcal{P}$, a construction of
  $(\forall p \in \mathcal{P})\psi(p)$ exists if and only if $\psi(q)$ holds for
  every $q : \mathcal{P}$.
\end{itemize}

As we shall see, this informal explanation may be expressed in terms of rules,
and those rules may be interpreted (and hence used) by a computer.

Note that the equivalence between $\neg\psi$ and $\psi\supset\perp$ holds also
in classical logic. But note also that intuitionistic statement $\neg\psi$
is much stronger than just ''there is no construction for $\psi$''.

\begin{example}
  Consider the following two propositions:
  \begin{enumerate}
  \item $\psi\supset\neg\neg\psi$
  \item $\neg\neg\psi\supset\psi$
  \end{enumerate}
  They are classical tautologies, and in classical logic there is a symmetry
  between 1 and 2. The former, which should be written as
  $\psi\supset((\psi\supset\perp)\supset\perp)$ has also an interpretation in
  intuitionistic logic, as follows:
  \begin{center}
    Given a proof of $\psi$, here is a proof of
    $(\psi\supset\perp)\supset\perp$: Take a proof of $\psi\supset\perp$. It is
    a method to translate proofs of $\psi$ into proofs of $\perp$. Since we have
    a proof of $\psi$, we can use this method to obtain a proof of $\perp$.
  \end{center}
  On the other hand, for the latter there isn't such a construction. Hence, it
  is no more true in intuitionistic logic.
\end{example}

A proposition is said to be \textit{proper} when all the quantifiers which
appear in it range only on sets. So a proposition is proper also if it contains
some arguments in a collection, but they are not quantified. The distinction
between arbitrary and proper propositions is important constructively, because
only the former admits a computational interpretation on a finite number of
rules. Further, a proper proposition which is true today will remain true
forever. On the other hand, a not proper proposition might be true today and
false tomorrow, since a collection doesn't have a fixed number of rules. For
instance, $(\forall D : \mathcal{P}X)\varphi(D)$ might be true today and may become
false when another subset, for which $\varphi$ is false, is added.

Propositions, and proper propositions as well, form a collection; two
propositions $\psi$ and $\varphi$, are considered to be equal when they are
\textit{logically equivalent}, that is, $\psi$ holds if and only if $\varphi$
holds, or equivalently when both $\psi$ and $\varphi$ hold. This is written as
$\psi \iff \varphi$.


%%% Local Variables:
%%% mode: latex
%%% TeX-master: "../notes"
%%% End:
