\documentclass{article}

\usepackage{amsmath}
\usepackage{amssymb}
\usepackage{dsfont}
\usepackage{amsthm}
\message{<Paul Taylor's Proof Trees, 2 August 1996>}
%% Build proof tree for Natural Deduction, Sequent Calculus, etc.
%% WITH SHORTENING OF PROOF RULES!
%% Paul Taylor, begun 10 Oct 1989
%% *** THIS IS ONLY A PRELIMINARY VERSION AND THINGS MAY CHANGE! ***
%%
%% 2 Aug 1996: fixed \mscount and \proofdotnumber
%%
%%      \prooftree
%%              hyp1            produces:
%%              hyp2
%%              hyp3            hyp1    hyp2    hyp3
%%      \justifies              -------------------- rulename
%%              concl                   concl
%%      \thickness=0.08em
%%      \shiftright 2em
%%      \using
%%              rulename
%%      \endprooftree
%%
%% where the hypotheses may be similar structures or just formulae.
%%
%% To get a vertical string of dots instead of the proof rule, do
%%
%%      \prooftree                      which produces:
%%              [hyp]
%%      \using                                  [hyp]
%%              name                              .
%%      \proofdotseparation=1.2ex                 .name
%%      \proofdotnumber=4                         .
%%      \leadsto                                  .
%%              concl                           concl
%%      \endprooftree
%%
%% Within a prooftree, \[ and \] may be used instead of \prooftree and
%% \endprooftree; this is not permitted at the outer level because it
%% conflicts with LaTeX. Also,
%%      \Justifies
%% produces a double line. In LaTeX you can use \begin{prooftree} and
%% \end{prootree} at the outer level (however this will not work for the inner
%% levels, but in any case why would you want to be so verbose?).
%%
%% All of of the keywords except \prooftree and \endprooftree are optional
%% and may appear in any order. They may also be combined in \newcommand's
%% eg "\def\Cut{\using\sf cut\thickness.08em\justifies}" with the abbreviation
%% "\prooftree hyp1 hyp2 \Cut \concl \endprooftree". This is recommended and
%% some standard abbreviations will be found at the end of this file.
%%
%% \thickness specifies the breadth of the rule in any units, although
%% font-relative units such as "ex" or "em" are preferable.
%% It may optionally be followed by "=".
%% \proofrulebreadth=.08em or \setlength\proofrulebreadth{.08em} may also be
%% used either in place of \thickness or globally; the default is 0.04em.
%% \proofdotseparation and \proofdotnumber control the size of the
%% string of dots
%%
%% If proof trees and formulae are mixed, some explicit spacing is needed,
%% but don't put anything to the left of the left-most (or the right of
%% the right-most) hypothesis, or put it in braces, because this will cause
%% the indentation to be lost.
%%
%% By default the conclusion is centered wrt the left-most and right-most
%% immediate hypotheses (not their proofs); \shiftright or \shiftleft moves
%% it relative to this position. (Not sure about this specification or how
%% it should affect spreading of proof tree.)
%
% global assignments to dimensions seem to have the effect of stretching
% diagrams horizontally.
%
%%==========================================================================

\def\introrule{{\cal I}}\def\elimrule{{\cal E}}%%
\def\andintro{\using{\land}\introrule\justifies}%%
\def\impelim{\using{\Rightarrow}\elimrule\justifies}%%
\def\allintro{\using{\forall}\introrule\justifies}%%
\def\allelim{\using{\forall}\elimrule\justifies}%%
\def\falseelim{\using{\bot}\elimrule\justifies}%%
\def\existsintro{\using{\exists}\introrule\justifies}%%

%% #1 is meant to be 1 or 2 for the first or second formula
\def\andelim#1{\using{\land}#1\elimrule\justifies}%%
\def\orintro#1{\using{\lor}#1\introrule\justifies}%%

%% #1 is meant to be a label corresponding to the discharged hypothesis/es
\def\impintro#1{\using{\Rightarrow}\introrule_{#1}\justifies}%%
\def\orelim#1{\using{\lor}\elimrule_{#1}\justifies}%%
\def\existselim#1{\using{\exists}\elimrule_{#1}\justifies}

%%==========================================================================

\newdimen\proofrulebreadth \proofrulebreadth=.05em
\newdimen\proofdotseparation \proofdotseparation=1.25ex
\newdimen\proofrulebaseline \proofrulebaseline=2ex
\newcount\proofdotnumber \proofdotnumber=3
\let\then\relax
\def\hfi{\hskip0pt plus.0001fil}
\mathchardef\squigto="3A3B
%
% flag where we are
\newif\ifinsideprooftree\insideprooftreefalse
\newif\ifonleftofproofrule\onleftofproofrulefalse
\newif\ifproofdots\proofdotsfalse
\newif\ifdoubleproof\doubleprooffalse
\let\wereinproofbit\relax
%
% dimensions and boxes of bits
\newdimen\shortenproofleft
\newdimen\shortenproofright
\newdimen\proofbelowshift
\newbox\proofabove
\newbox\proofbelow
\newbox\proofrulename
%
% miscellaneous commands for setting values
\def\shiftproofbelow{\let\next\relax\afterassignment\setshiftproofbelow\dimen0 }
\def\shiftproofbelowneg{\def\next{\multiply\dimen0 by-1 }%
\afterassignment\setshiftproofbelow\dimen0 }
\def\setshiftproofbelow{\next\proofbelowshift=\dimen0 }
\def\setproofrulebreadth{\proofrulebreadth}

%=============================================================================
\def\prooftree{% NESTED ZERO (\ifonleftofproofrule)
%
% first find out whether we're at the left-hand end of a proof rule
\ifnum  \lastpenalty=1
\then   \unpenalty
\else   \onleftofproofrulefalse
\fi
%
% some space on left (except if we're on left, and no infinity for outermost)
\ifonleftofproofrule
\else   \ifinsideprooftree
        \then   \hskip.5em plus1fil
        \fi
\fi
%
% begin our proof tree environment
\bgroup% NESTED ONE (\proofbelow, \proofrulename, \proofabove,
%               \shortenproofleft, \shortenproofright, \proofrulebreadth)
\setbox\proofbelow=\hbox{}\setbox\proofrulename=\hbox{}%
\let\justifies\proofover\let\leadsto\proofoverdots\let\Justifies\proofoverdbl
\let\using\proofusing\let\[\prooftree
\ifinsideprooftree\let\]\endprooftree\fi
\proofdotsfalse\doubleprooffalse
\let\thickness\setproofrulebreadth
\let\shiftright\shiftproofbelow \let\shift\shiftproofbelow
\let\shiftleft\shiftproofbelowneg
\let\ifwasinsideprooftree\ifinsideprooftree
\insideprooftreetrue
%
% now begin to set the top of the rule (definitions local to it)
\setbox\proofabove=\hbox\bgroup$\displaystyle % NESTED TWO
\let\wereinproofbit\prooftree
%
% these local variables will be copied out:
\shortenproofleft=0pt \shortenproofright=0pt \proofbelowshift=0pt
%
% flags to enable inner proof tree to detect if on left:
\onleftofproofruletrue\penalty1
}

%=============================================================================
% end whatever box and copy crucial values out of it
\def\eproofbit{% NESTED TWO
%
% various hacks applicable to hypothesis list 
\ifx    \wereinproofbit\prooftree
\then   \ifcase \lastpenalty
        \then   \shortenproofright=0pt  % 0: some other object, no indentation
        \or     \unpenalty\hfil         % 1: empty hypotheses, just glue
        \or     \unpenalty\unskip       % 2: just had a tree, remove glue
        \else   \shortenproofright=0pt  % eh?
        \fi
\fi
%
% pass out crucial values from scope
\global\dimen0=\shortenproofleft
\global\dimen1=\shortenproofright
\global\dimen2=\proofrulebreadth
\global\dimen3=\proofbelowshift
\global\dimen4=\proofdotseparation
\global\count255=\proofdotnumber
%
% end the box
$\egroup  % NESTED ONE
%
% restore the values
\shortenproofleft=\dimen0
\shortenproofright=\dimen1
\proofrulebreadth=\dimen2
\proofbelowshift=\dimen3
\proofdotseparation=\dimen4
\proofdotnumber=\count255
}

%=============================================================================
\def\proofover{% NESTED TWO
\eproofbit % NESTED ONE
\setbox\proofbelow=\hbox\bgroup % NESTED TWO
\let\wereinproofbit\proofover
$\displaystyle
}%
%
%=============================================================================
\def\proofoverdbl{% NESTED TWO
\eproofbit % NESTED ONE
\doubleprooftrue
\setbox\proofbelow=\hbox\bgroup % NESTED TWO
\let\wereinproofbit\proofoverdbl
$\displaystyle
}%
%
%=============================================================================
\def\proofoverdots{% NESTED TWO
\eproofbit % NESTED ONE
\proofdotstrue
\setbox\proofbelow=\hbox\bgroup % NESTED TWO
\let\wereinproofbit\proofoverdots
$\displaystyle
}%
%
%=============================================================================
\def\proofusing{% NESTED TWO
\eproofbit % NESTED ONE
\setbox\proofrulename=\hbox\bgroup % NESTED TWO
\let\wereinproofbit\proofusing
\kern0.3em$
}

%=============================================================================
\def\endprooftree{% NESTED TWO
\eproofbit % NESTED ONE
% \dimen0 =     length of proof rule
% \dimen1 =     indentation of conclusion wrt rule
% \dimen2 =     new \shortenproofleft, ie indentation of conclusion
% \dimen3 =     new \shortenproofright, ie
%                space on right of conclusion to end of tree
% \dimen4 =     space on right of conclusion below rule
  \dimen5 =0pt% spread of hypotheses
% \dimen6, \dimen7 = height & depth of rule
%
% length of rule needed by proof above
\dimen0=\wd\proofabove \advance\dimen0-\shortenproofleft
\advance\dimen0-\shortenproofright
%
% amount of spare space below
\dimen1=.5\dimen0 \advance\dimen1-.5\wd\proofbelow
\dimen4=\dimen1
\advance\dimen1\proofbelowshift \advance\dimen4-\proofbelowshift
%
% conclusion sticks out to left of immediate hypotheses
\ifdim  \dimen1<0pt
\then   \advance\shortenproofleft\dimen1
        \advance\dimen0-\dimen1
        \dimen1=0pt
%       now it sticks out to left of tree!
        \ifdim  \shortenproofleft<0pt
        \then   \setbox\proofabove=\hbox{%
                        \kern-\shortenproofleft\unhbox\proofabove}%
                \shortenproofleft=0pt
        \fi
\fi
%
% and to the right
\ifdim  \dimen4<0pt
\then   \advance\shortenproofright\dimen4
        \advance\dimen0-\dimen4
        \dimen4=0pt
\fi
%
% make sure enough space for label
\ifdim  \shortenproofright<\wd\proofrulename
\then   \shortenproofright=\wd\proofrulename
\fi
%
% calculate new indentations
\dimen2=\shortenproofleft \advance\dimen2 by\dimen1
\dimen3=\shortenproofright\advance\dimen3 by\dimen4
%
% make the rule or dots, with name attached
\ifproofdots
\then
        \dimen6=\shortenproofleft \advance\dimen6 .5\dimen0
        \setbox1=\vbox to\proofdotseparation{\vss\hbox{$\cdot$}\vss}%
        \setbox0=\hbox{%
                \advance\dimen6-.5\wd1
                \kern\dimen6
                $\vcenter to\proofdotnumber\proofdotseparation
                        {\leaders\box1\vfill}$%
                \unhbox\proofrulename}%
\else   \dimen6=\fontdimen22\the\textfont2 % height of maths axis
        \dimen7=\dimen6
        \advance\dimen6by.5\proofrulebreadth
        \advance\dimen7by-.5\proofrulebreadth
        \setbox0=\hbox{%
                \kern\shortenproofleft
                \ifdoubleproof
                \then   \hbox to\dimen0{%
                        $\mathsurround0pt\mathord=\mkern-6mu%
                        \cleaders\hbox{$\mkern-2mu=\mkern-2mu$}\hfill
                        \mkern-6mu\mathord=$}%
                \else   \vrule height\dimen6 depth-\dimen7 width\dimen0
                \fi
                \unhbox\proofrulename}%
        \ht0=\dimen6 \dp0=-\dimen7
\fi
%
% set up to centre outermost tree only
\let\doll\relax
\ifwasinsideprooftree
\then   \let\VBOX\vbox
\else   \ifmmode\else$\let\doll=$\fi
        \let\VBOX\vcenter
\fi
% this \vbox or \vcenter is the actual output:
\VBOX   {\baselineskip\proofrulebaseline \lineskip.2ex
        \expandafter\lineskiplimit\ifproofdots0ex\else-0.6ex\fi
        \hbox   spread\dimen5   {\hfi\unhbox\proofabove\hfi}%
        \hbox{\box0}%
        \hbox   {\kern\dimen2 \box\proofbelow}}\doll%
%
% pass new indentations out of scope
\global\dimen2=\dimen2
\global\dimen3=\dimen3
\egroup % NESTED ZERO
\ifonleftofproofrule
\then   \shortenproofleft=\dimen2
\fi
\shortenproofright=\dimen3
%
% some space on right and flag we've just made a tree
\onleftofproofrulefalse
\ifinsideprooftree
\then   \hskip.5em plus 1fil \penalty2
\fi
}

\endinput

%==========================================================================
% IDEAS
% 1.    Specification of \shiftright and how to spread trees.
% 2.    Spacing command \m which causes 1em+1fil spacing, over-riding
%       exisiting space on sides of trees and not affecting the
%       detection of being on the left or right.
% 3.    Hack using \@currenvir to detect LaTeX environment; have to
%       use \aftergroup to pass \shortenproofleft/right out.
% 4.    (Pie in the sky) detect how much trees can be "tucked in"
% 5.    Discharged hypotheses (diagonal lines).

Date: Tue, 19 May 1998 16:45:32 +0100
From: Simon Gay <simon@dcs.rhbnc.ac.uk>

I've got another problem when combining
your packages with elsart.cls. The code

\documentclass{elsart}
\message{<Paul Taylor's Proof Trees, 2 August 1996>}
%% Build proof tree for Natural Deduction, Sequent Calculus, etc.
%% WITH SHORTENING OF PROOF RULES!
%% Paul Taylor, begun 10 Oct 1989
%% *** THIS IS ONLY A PRELIMINARY VERSION AND THINGS MAY CHANGE! ***
%%
%% 2 Aug 1996: fixed \mscount and \proofdotnumber
%%
%%      \prooftree
%%              hyp1            produces:
%%              hyp2
%%              hyp3            hyp1    hyp2    hyp3
%%      \justifies              -------------------- rulename
%%              concl                   concl
%%      \thickness=0.08em
%%      \shiftright 2em
%%      \using
%%              rulename
%%      \endprooftree
%%
%% where the hypotheses may be similar structures or just formulae.
%%
%% To get a vertical string of dots instead of the proof rule, do
%%
%%      \prooftree                      which produces:
%%              [hyp]
%%      \using                                  [hyp]
%%              name                              .
%%      \proofdotseparation=1.2ex                 .name
%%      \proofdotnumber=4                         .
%%      \leadsto                                  .
%%              concl                           concl
%%      \endprooftree
%%
%% Within a prooftree, \[ and \] may be used instead of \prooftree and
%% \endprooftree; this is not permitted at the outer level because it
%% conflicts with LaTeX. Also,
%%      \Justifies
%% produces a double line. In LaTeX you can use \begin{prooftree} and
%% \end{prootree} at the outer level (however this will not work for the inner
%% levels, but in any case why would you want to be so verbose?).
%%
%% All of of the keywords except \prooftree and \endprooftree are optional
%% and may appear in any order. They may also be combined in \newcommand's
%% eg "\def\Cut{\using\sf cut\thickness.08em\justifies}" with the abbreviation
%% "\prooftree hyp1 hyp2 \Cut \concl \endprooftree". This is recommended and
%% some standard abbreviations will be found at the end of this file.
%%
%% \thickness specifies the breadth of the rule in any units, although
%% font-relative units such as "ex" or "em" are preferable.
%% It may optionally be followed by "=".
%% \proofrulebreadth=.08em or \setlength\proofrulebreadth{.08em} may also be
%% used either in place of \thickness or globally; the default is 0.04em.
%% \proofdotseparation and \proofdotnumber control the size of the
%% string of dots
%%
%% If proof trees and formulae are mixed, some explicit spacing is needed,
%% but don't put anything to the left of the left-most (or the right of
%% the right-most) hypothesis, or put it in braces, because this will cause
%% the indentation to be lost.
%%
%% By default the conclusion is centered wrt the left-most and right-most
%% immediate hypotheses (not their proofs); \shiftright or \shiftleft moves
%% it relative to this position. (Not sure about this specification or how
%% it should affect spreading of proof tree.)
%
% global assignments to dimensions seem to have the effect of stretching
% diagrams horizontally.
%
%%==========================================================================

\def\introrule{{\cal I}}\def\elimrule{{\cal E}}%%
\def\andintro{\using{\land}\introrule\justifies}%%
\def\impelim{\using{\Rightarrow}\elimrule\justifies}%%
\def\allintro{\using{\forall}\introrule\justifies}%%
\def\allelim{\using{\forall}\elimrule\justifies}%%
\def\falseelim{\using{\bot}\elimrule\justifies}%%
\def\existsintro{\using{\exists}\introrule\justifies}%%

%% #1 is meant to be 1 or 2 for the first or second formula
\def\andelim#1{\using{\land}#1\elimrule\justifies}%%
\def\orintro#1{\using{\lor}#1\introrule\justifies}%%

%% #1 is meant to be a label corresponding to the discharged hypothesis/es
\def\impintro#1{\using{\Rightarrow}\introrule_{#1}\justifies}%%
\def\orelim#1{\using{\lor}\elimrule_{#1}\justifies}%%
\def\existselim#1{\using{\exists}\elimrule_{#1}\justifies}

%%==========================================================================

\newdimen\proofrulebreadth \proofrulebreadth=.05em
\newdimen\proofdotseparation \proofdotseparation=1.25ex
\newdimen\proofrulebaseline \proofrulebaseline=2ex
\newcount\proofdotnumber \proofdotnumber=3
\let\then\relax
\def\hfi{\hskip0pt plus.0001fil}
\mathchardef\squigto="3A3B
%
% flag where we are
\newif\ifinsideprooftree\insideprooftreefalse
\newif\ifonleftofproofrule\onleftofproofrulefalse
\newif\ifproofdots\proofdotsfalse
\newif\ifdoubleproof\doubleprooffalse
\let\wereinproofbit\relax
%
% dimensions and boxes of bits
\newdimen\shortenproofleft
\newdimen\shortenproofright
\newdimen\proofbelowshift
\newbox\proofabove
\newbox\proofbelow
\newbox\proofrulename
%
% miscellaneous commands for setting values
\def\shiftproofbelow{\let\next\relax\afterassignment\setshiftproofbelow\dimen0 }
\def\shiftproofbelowneg{\def\next{\multiply\dimen0 by-1 }%
\afterassignment\setshiftproofbelow\dimen0 }
\def\setshiftproofbelow{\next\proofbelowshift=\dimen0 }
\def\setproofrulebreadth{\proofrulebreadth}

%=============================================================================
\def\prooftree{% NESTED ZERO (\ifonleftofproofrule)
%
% first find out whether we're at the left-hand end of a proof rule
\ifnum  \lastpenalty=1
\then   \unpenalty
\else   \onleftofproofrulefalse
\fi
%
% some space on left (except if we're on left, and no infinity for outermost)
\ifonleftofproofrule
\else   \ifinsideprooftree
        \then   \hskip.5em plus1fil
        \fi
\fi
%
% begin our proof tree environment
\bgroup% NESTED ONE (\proofbelow, \proofrulename, \proofabove,
%               \shortenproofleft, \shortenproofright, \proofrulebreadth)
\setbox\proofbelow=\hbox{}\setbox\proofrulename=\hbox{}%
\let\justifies\proofover\let\leadsto\proofoverdots\let\Justifies\proofoverdbl
\let\using\proofusing\let\[\prooftree
\ifinsideprooftree\let\]\endprooftree\fi
\proofdotsfalse\doubleprooffalse
\let\thickness\setproofrulebreadth
\let\shiftright\shiftproofbelow \let\shift\shiftproofbelow
\let\shiftleft\shiftproofbelowneg
\let\ifwasinsideprooftree\ifinsideprooftree
\insideprooftreetrue
%
% now begin to set the top of the rule (definitions local to it)
\setbox\proofabove=\hbox\bgroup$\displaystyle % NESTED TWO
\let\wereinproofbit\prooftree
%
% these local variables will be copied out:
\shortenproofleft=0pt \shortenproofright=0pt \proofbelowshift=0pt
%
% flags to enable inner proof tree to detect if on left:
\onleftofproofruletrue\penalty1
}

%=============================================================================
% end whatever box and copy crucial values out of it
\def\eproofbit{% NESTED TWO
%
% various hacks applicable to hypothesis list 
\ifx    \wereinproofbit\prooftree
\then   \ifcase \lastpenalty
        \then   \shortenproofright=0pt  % 0: some other object, no indentation
        \or     \unpenalty\hfil         % 1: empty hypotheses, just glue
        \or     \unpenalty\unskip       % 2: just had a tree, remove glue
        \else   \shortenproofright=0pt  % eh?
        \fi
\fi
%
% pass out crucial values from scope
\global\dimen0=\shortenproofleft
\global\dimen1=\shortenproofright
\global\dimen2=\proofrulebreadth
\global\dimen3=\proofbelowshift
\global\dimen4=\proofdotseparation
\global\count255=\proofdotnumber
%
% end the box
$\egroup  % NESTED ONE
%
% restore the values
\shortenproofleft=\dimen0
\shortenproofright=\dimen1
\proofrulebreadth=\dimen2
\proofbelowshift=\dimen3
\proofdotseparation=\dimen4
\proofdotnumber=\count255
}

%=============================================================================
\def\proofover{% NESTED TWO
\eproofbit % NESTED ONE
\setbox\proofbelow=\hbox\bgroup % NESTED TWO
\let\wereinproofbit\proofover
$\displaystyle
}%
%
%=============================================================================
\def\proofoverdbl{% NESTED TWO
\eproofbit % NESTED ONE
\doubleprooftrue
\setbox\proofbelow=\hbox\bgroup % NESTED TWO
\let\wereinproofbit\proofoverdbl
$\displaystyle
}%
%
%=============================================================================
\def\proofoverdots{% NESTED TWO
\eproofbit % NESTED ONE
\proofdotstrue
\setbox\proofbelow=\hbox\bgroup % NESTED TWO
\let\wereinproofbit\proofoverdots
$\displaystyle
}%
%
%=============================================================================
\def\proofusing{% NESTED TWO
\eproofbit % NESTED ONE
\setbox\proofrulename=\hbox\bgroup % NESTED TWO
\let\wereinproofbit\proofusing
\kern0.3em$
}

%=============================================================================
\def\endprooftree{% NESTED TWO
\eproofbit % NESTED ONE
% \dimen0 =     length of proof rule
% \dimen1 =     indentation of conclusion wrt rule
% \dimen2 =     new \shortenproofleft, ie indentation of conclusion
% \dimen3 =     new \shortenproofright, ie
%                space on right of conclusion to end of tree
% \dimen4 =     space on right of conclusion below rule
  \dimen5 =0pt% spread of hypotheses
% \dimen6, \dimen7 = height & depth of rule
%
% length of rule needed by proof above
\dimen0=\wd\proofabove \advance\dimen0-\shortenproofleft
\advance\dimen0-\shortenproofright
%
% amount of spare space below
\dimen1=.5\dimen0 \advance\dimen1-.5\wd\proofbelow
\dimen4=\dimen1
\advance\dimen1\proofbelowshift \advance\dimen4-\proofbelowshift
%
% conclusion sticks out to left of immediate hypotheses
\ifdim  \dimen1<0pt
\then   \advance\shortenproofleft\dimen1
        \advance\dimen0-\dimen1
        \dimen1=0pt
%       now it sticks out to left of tree!
        \ifdim  \shortenproofleft<0pt
        \then   \setbox\proofabove=\hbox{%
                        \kern-\shortenproofleft\unhbox\proofabove}%
                \shortenproofleft=0pt
        \fi
\fi
%
% and to the right
\ifdim  \dimen4<0pt
\then   \advance\shortenproofright\dimen4
        \advance\dimen0-\dimen4
        \dimen4=0pt
\fi
%
% make sure enough space for label
\ifdim  \shortenproofright<\wd\proofrulename
\then   \shortenproofright=\wd\proofrulename
\fi
%
% calculate new indentations
\dimen2=\shortenproofleft \advance\dimen2 by\dimen1
\dimen3=\shortenproofright\advance\dimen3 by\dimen4
%
% make the rule or dots, with name attached
\ifproofdots
\then
        \dimen6=\shortenproofleft \advance\dimen6 .5\dimen0
        \setbox1=\vbox to\proofdotseparation{\vss\hbox{$\cdot$}\vss}%
        \setbox0=\hbox{%
                \advance\dimen6-.5\wd1
                \kern\dimen6
                $\vcenter to\proofdotnumber\proofdotseparation
                        {\leaders\box1\vfill}$%
                \unhbox\proofrulename}%
\else   \dimen6=\fontdimen22\the\textfont2 % height of maths axis
        \dimen7=\dimen6
        \advance\dimen6by.5\proofrulebreadth
        \advance\dimen7by-.5\proofrulebreadth
        \setbox0=\hbox{%
                \kern\shortenproofleft
                \ifdoubleproof
                \then   \hbox to\dimen0{%
                        $\mathsurround0pt\mathord=\mkern-6mu%
                        \cleaders\hbox{$\mkern-2mu=\mkern-2mu$}\hfill
                        \mkern-6mu\mathord=$}%
                \else   \vrule height\dimen6 depth-\dimen7 width\dimen0
                \fi
                \unhbox\proofrulename}%
        \ht0=\dimen6 \dp0=-\dimen7
\fi
%
% set up to centre outermost tree only
\let\doll\relax
\ifwasinsideprooftree
\then   \let\VBOX\vbox
\else   \ifmmode\else$\let\doll=$\fi
        \let\VBOX\vcenter
\fi
% this \vbox or \vcenter is the actual output:
\VBOX   {\baselineskip\proofrulebaseline \lineskip.2ex
        \expandafter\lineskiplimit\ifproofdots0ex\else-0.6ex\fi
        \hbox   spread\dimen5   {\hfi\unhbox\proofabove\hfi}%
        \hbox{\box0}%
        \hbox   {\kern\dimen2 \box\proofbelow}}\doll%
%
% pass new indentations out of scope
\global\dimen2=\dimen2
\global\dimen3=\dimen3
\egroup % NESTED ZERO
\ifonleftofproofrule
\then   \shortenproofleft=\dimen2
\fi
\shortenproofright=\dimen3
%
% some space on right and flag we've just made a tree
\onleftofproofrulefalse
\ifinsideprooftree
\then   \hskip.5em plus 1fil \penalty2
\fi
}

\endinput

%==========================================================================
% IDEAS
% 1.    Specification of \shiftright and how to spread trees.
% 2.    Spacing command \m which causes 1em+1fil spacing, over-riding
%       exisiting space on sides of trees and not affecting the
%       detection of being on the left or right.
% 3.    Hack using \@currenvir to detect LaTeX environment; have to
%       use \aftergroup to pass \shortenproofleft/right out.
% 4.    (Pie in the sky) detect how much trees can be "tucked in"
% 5.    Discharged hypotheses (diagonal lines).

Date: Tue, 19 May 1998 16:45:32 +0100
From: Simon Gay <simon@dcs.rhbnc.ac.uk>

I've got another problem when combining
your packages with elsart.cls. The code

\documentclass{elsart}
\message{<Paul Taylor's Proof Trees, 2 August 1996>}
%% Build proof tree for Natural Deduction, Sequent Calculus, etc.
%% WITH SHORTENING OF PROOF RULES!
%% Paul Taylor, begun 10 Oct 1989
%% *** THIS IS ONLY A PRELIMINARY VERSION AND THINGS MAY CHANGE! ***
%%
%% 2 Aug 1996: fixed \mscount and \proofdotnumber
%%
%%      \prooftree
%%              hyp1            produces:
%%              hyp2
%%              hyp3            hyp1    hyp2    hyp3
%%      \justifies              -------------------- rulename
%%              concl                   concl
%%      \thickness=0.08em
%%      \shiftright 2em
%%      \using
%%              rulename
%%      \endprooftree
%%
%% where the hypotheses may be similar structures or just formulae.
%%
%% To get a vertical string of dots instead of the proof rule, do
%%
%%      \prooftree                      which produces:
%%              [hyp]
%%      \using                                  [hyp]
%%              name                              .
%%      \proofdotseparation=1.2ex                 .name
%%      \proofdotnumber=4                         .
%%      \leadsto                                  .
%%              concl                           concl
%%      \endprooftree
%%
%% Within a prooftree, \[ and \] may be used instead of \prooftree and
%% \endprooftree; this is not permitted at the outer level because it
%% conflicts with LaTeX. Also,
%%      \Justifies
%% produces a double line. In LaTeX you can use \begin{prooftree} and
%% \end{prootree} at the outer level (however this will not work for the inner
%% levels, but in any case why would you want to be so verbose?).
%%
%% All of of the keywords except \prooftree and \endprooftree are optional
%% and may appear in any order. They may also be combined in \newcommand's
%% eg "\def\Cut{\using\sf cut\thickness.08em\justifies}" with the abbreviation
%% "\prooftree hyp1 hyp2 \Cut \concl \endprooftree". This is recommended and
%% some standard abbreviations will be found at the end of this file.
%%
%% \thickness specifies the breadth of the rule in any units, although
%% font-relative units such as "ex" or "em" are preferable.
%% It may optionally be followed by "=".
%% \proofrulebreadth=.08em or \setlength\proofrulebreadth{.08em} may also be
%% used either in place of \thickness or globally; the default is 0.04em.
%% \proofdotseparation and \proofdotnumber control the size of the
%% string of dots
%%
%% If proof trees and formulae are mixed, some explicit spacing is needed,
%% but don't put anything to the left of the left-most (or the right of
%% the right-most) hypothesis, or put it in braces, because this will cause
%% the indentation to be lost.
%%
%% By default the conclusion is centered wrt the left-most and right-most
%% immediate hypotheses (not their proofs); \shiftright or \shiftleft moves
%% it relative to this position. (Not sure about this specification or how
%% it should affect spreading of proof tree.)
%
% global assignments to dimensions seem to have the effect of stretching
% diagrams horizontally.
%
%%==========================================================================

\def\introrule{{\cal I}}\def\elimrule{{\cal E}}%%
\def\andintro{\using{\land}\introrule\justifies}%%
\def\impelim{\using{\Rightarrow}\elimrule\justifies}%%
\def\allintro{\using{\forall}\introrule\justifies}%%
\def\allelim{\using{\forall}\elimrule\justifies}%%
\def\falseelim{\using{\bot}\elimrule\justifies}%%
\def\existsintro{\using{\exists}\introrule\justifies}%%

%% #1 is meant to be 1 or 2 for the first or second formula
\def\andelim#1{\using{\land}#1\elimrule\justifies}%%
\def\orintro#1{\using{\lor}#1\introrule\justifies}%%

%% #1 is meant to be a label corresponding to the discharged hypothesis/es
\def\impintro#1{\using{\Rightarrow}\introrule_{#1}\justifies}%%
\def\orelim#1{\using{\lor}\elimrule_{#1}\justifies}%%
\def\existselim#1{\using{\exists}\elimrule_{#1}\justifies}

%%==========================================================================

\newdimen\proofrulebreadth \proofrulebreadth=.05em
\newdimen\proofdotseparation \proofdotseparation=1.25ex
\newdimen\proofrulebaseline \proofrulebaseline=2ex
\newcount\proofdotnumber \proofdotnumber=3
\let\then\relax
\def\hfi{\hskip0pt plus.0001fil}
\mathchardef\squigto="3A3B
%
% flag where we are
\newif\ifinsideprooftree\insideprooftreefalse
\newif\ifonleftofproofrule\onleftofproofrulefalse
\newif\ifproofdots\proofdotsfalse
\newif\ifdoubleproof\doubleprooffalse
\let\wereinproofbit\relax
%
% dimensions and boxes of bits
\newdimen\shortenproofleft
\newdimen\shortenproofright
\newdimen\proofbelowshift
\newbox\proofabove
\newbox\proofbelow
\newbox\proofrulename
%
% miscellaneous commands for setting values
\def\shiftproofbelow{\let\next\relax\afterassignment\setshiftproofbelow\dimen0 }
\def\shiftproofbelowneg{\def\next{\multiply\dimen0 by-1 }%
\afterassignment\setshiftproofbelow\dimen0 }
\def\setshiftproofbelow{\next\proofbelowshift=\dimen0 }
\def\setproofrulebreadth{\proofrulebreadth}

%=============================================================================
\def\prooftree{% NESTED ZERO (\ifonleftofproofrule)
%
% first find out whether we're at the left-hand end of a proof rule
\ifnum  \lastpenalty=1
\then   \unpenalty
\else   \onleftofproofrulefalse
\fi
%
% some space on left (except if we're on left, and no infinity for outermost)
\ifonleftofproofrule
\else   \ifinsideprooftree
        \then   \hskip.5em plus1fil
        \fi
\fi
%
% begin our proof tree environment
\bgroup% NESTED ONE (\proofbelow, \proofrulename, \proofabove,
%               \shortenproofleft, \shortenproofright, \proofrulebreadth)
\setbox\proofbelow=\hbox{}\setbox\proofrulename=\hbox{}%
\let\justifies\proofover\let\leadsto\proofoverdots\let\Justifies\proofoverdbl
\let\using\proofusing\let\[\prooftree
\ifinsideprooftree\let\]\endprooftree\fi
\proofdotsfalse\doubleprooffalse
\let\thickness\setproofrulebreadth
\let\shiftright\shiftproofbelow \let\shift\shiftproofbelow
\let\shiftleft\shiftproofbelowneg
\let\ifwasinsideprooftree\ifinsideprooftree
\insideprooftreetrue
%
% now begin to set the top of the rule (definitions local to it)
\setbox\proofabove=\hbox\bgroup$\displaystyle % NESTED TWO
\let\wereinproofbit\prooftree
%
% these local variables will be copied out:
\shortenproofleft=0pt \shortenproofright=0pt \proofbelowshift=0pt
%
% flags to enable inner proof tree to detect if on left:
\onleftofproofruletrue\penalty1
}

%=============================================================================
% end whatever box and copy crucial values out of it
\def\eproofbit{% NESTED TWO
%
% various hacks applicable to hypothesis list 
\ifx    \wereinproofbit\prooftree
\then   \ifcase \lastpenalty
        \then   \shortenproofright=0pt  % 0: some other object, no indentation
        \or     \unpenalty\hfil         % 1: empty hypotheses, just glue
        \or     \unpenalty\unskip       % 2: just had a tree, remove glue
        \else   \shortenproofright=0pt  % eh?
        \fi
\fi
%
% pass out crucial values from scope
\global\dimen0=\shortenproofleft
\global\dimen1=\shortenproofright
\global\dimen2=\proofrulebreadth
\global\dimen3=\proofbelowshift
\global\dimen4=\proofdotseparation
\global\count255=\proofdotnumber
%
% end the box
$\egroup  % NESTED ONE
%
% restore the values
\shortenproofleft=\dimen0
\shortenproofright=\dimen1
\proofrulebreadth=\dimen2
\proofbelowshift=\dimen3
\proofdotseparation=\dimen4
\proofdotnumber=\count255
}

%=============================================================================
\def\proofover{% NESTED TWO
\eproofbit % NESTED ONE
\setbox\proofbelow=\hbox\bgroup % NESTED TWO
\let\wereinproofbit\proofover
$\displaystyle
}%
%
%=============================================================================
\def\proofoverdbl{% NESTED TWO
\eproofbit % NESTED ONE
\doubleprooftrue
\setbox\proofbelow=\hbox\bgroup % NESTED TWO
\let\wereinproofbit\proofoverdbl
$\displaystyle
}%
%
%=============================================================================
\def\proofoverdots{% NESTED TWO
\eproofbit % NESTED ONE
\proofdotstrue
\setbox\proofbelow=\hbox\bgroup % NESTED TWO
\let\wereinproofbit\proofoverdots
$\displaystyle
}%
%
%=============================================================================
\def\proofusing{% NESTED TWO
\eproofbit % NESTED ONE
\setbox\proofrulename=\hbox\bgroup % NESTED TWO
\let\wereinproofbit\proofusing
\kern0.3em$
}

%=============================================================================
\def\endprooftree{% NESTED TWO
\eproofbit % NESTED ONE
% \dimen0 =     length of proof rule
% \dimen1 =     indentation of conclusion wrt rule
% \dimen2 =     new \shortenproofleft, ie indentation of conclusion
% \dimen3 =     new \shortenproofright, ie
%                space on right of conclusion to end of tree
% \dimen4 =     space on right of conclusion below rule
  \dimen5 =0pt% spread of hypotheses
% \dimen6, \dimen7 = height & depth of rule
%
% length of rule needed by proof above
\dimen0=\wd\proofabove \advance\dimen0-\shortenproofleft
\advance\dimen0-\shortenproofright
%
% amount of spare space below
\dimen1=.5\dimen0 \advance\dimen1-.5\wd\proofbelow
\dimen4=\dimen1
\advance\dimen1\proofbelowshift \advance\dimen4-\proofbelowshift
%
% conclusion sticks out to left of immediate hypotheses
\ifdim  \dimen1<0pt
\then   \advance\shortenproofleft\dimen1
        \advance\dimen0-\dimen1
        \dimen1=0pt
%       now it sticks out to left of tree!
        \ifdim  \shortenproofleft<0pt
        \then   \setbox\proofabove=\hbox{%
                        \kern-\shortenproofleft\unhbox\proofabove}%
                \shortenproofleft=0pt
        \fi
\fi
%
% and to the right
\ifdim  \dimen4<0pt
\then   \advance\shortenproofright\dimen4
        \advance\dimen0-\dimen4
        \dimen4=0pt
\fi
%
% make sure enough space for label
\ifdim  \shortenproofright<\wd\proofrulename
\then   \shortenproofright=\wd\proofrulename
\fi
%
% calculate new indentations
\dimen2=\shortenproofleft \advance\dimen2 by\dimen1
\dimen3=\shortenproofright\advance\dimen3 by\dimen4
%
% make the rule or dots, with name attached
\ifproofdots
\then
        \dimen6=\shortenproofleft \advance\dimen6 .5\dimen0
        \setbox1=\vbox to\proofdotseparation{\vss\hbox{$\cdot$}\vss}%
        \setbox0=\hbox{%
                \advance\dimen6-.5\wd1
                \kern\dimen6
                $\vcenter to\proofdotnumber\proofdotseparation
                        {\leaders\box1\vfill}$%
                \unhbox\proofrulename}%
\else   \dimen6=\fontdimen22\the\textfont2 % height of maths axis
        \dimen7=\dimen6
        \advance\dimen6by.5\proofrulebreadth
        \advance\dimen7by-.5\proofrulebreadth
        \setbox0=\hbox{%
                \kern\shortenproofleft
                \ifdoubleproof
                \then   \hbox to\dimen0{%
                        $\mathsurround0pt\mathord=\mkern-6mu%
                        \cleaders\hbox{$\mkern-2mu=\mkern-2mu$}\hfill
                        \mkern-6mu\mathord=$}%
                \else   \vrule height\dimen6 depth-\dimen7 width\dimen0
                \fi
                \unhbox\proofrulename}%
        \ht0=\dimen6 \dp0=-\dimen7
\fi
%
% set up to centre outermost tree only
\let\doll\relax
\ifwasinsideprooftree
\then   \let\VBOX\vbox
\else   \ifmmode\else$\let\doll=$\fi
        \let\VBOX\vcenter
\fi
% this \vbox or \vcenter is the actual output:
\VBOX   {\baselineskip\proofrulebaseline \lineskip.2ex
        \expandafter\lineskiplimit\ifproofdots0ex\else-0.6ex\fi
        \hbox   spread\dimen5   {\hfi\unhbox\proofabove\hfi}%
        \hbox{\box0}%
        \hbox   {\kern\dimen2 \box\proofbelow}}\doll%
%
% pass new indentations out of scope
\global\dimen2=\dimen2
\global\dimen3=\dimen3
\egroup % NESTED ZERO
\ifonleftofproofrule
\then   \shortenproofleft=\dimen2
\fi
\shortenproofright=\dimen3
%
% some space on right and flag we've just made a tree
\onleftofproofrulefalse
\ifinsideprooftree
\then   \hskip.5em plus 1fil \penalty2
\fi
}

\endinput

%==========================================================================
% IDEAS
% 1.    Specification of \shiftright and how to spread trees.
% 2.    Spacing command \m which causes 1em+1fil spacing, over-riding
%       exisiting space on sides of trees and not affecting the
%       detection of being on the left or right.
% 3.    Hack using \@currenvir to detect LaTeX environment; have to
%       use \aftergroup to pass \shortenproofleft/right out.
% 4.    (Pie in the sky) detect how much trees can be "tucked in"
% 5.    Discharged hypotheses (diagonal lines).

Date: Tue, 19 May 1998 16:45:32 +0100
From: Simon Gay <simon@dcs.rhbnc.ac.uk>

I've got another problem when combining
your packages with elsart.cls. The code

\documentclass{elsart}
\input{prooftree}
\input{diagrams}
\begin{document}

\[
\begin{prooftree}
A \rTo^{f} B
\justifies
p \rTo^{(a,c)} q
\end{prooftree}
\]

\end{document}

doesn't leave enough space below the line in the proof tree, so that
the (a,c) label on the lower arrow runs into the line. It's fine with
article.cls. 

\input{diagrams}
\begin{document}

\[
\begin{prooftree}
A \rTo^{f} B
\justifies
p \rTo^{(a,c)} q
\end{prooftree}
\]

\end{document}

doesn't leave enough space below the line in the proof tree, so that
the (a,c) label on the lower arrow runs into the line. It's fine with
article.cls. 

\input{diagrams}
\begin{document}

\[
\begin{prooftree}
A \rTo^{f} B
\justifies
p \rTo^{(a,c)} q
\end{prooftree}
\]

\end{document}

doesn't leave enough space below the line in the proof tree, so that
the (a,c) label on the lower arrow runs into the line. It's fine with
article.cls. 

\usepackage{mathpazo}

\theoremstyle{definition}
\newtheorem{example}{Example}[section]
\newtheorem{exercise}{Exercise}[section]

\newcommand{\strong}[1]{\textbf{#1}}
\newcommand{\hsetj}[2]{#1 \; \text{set} \; (#2)}
\newcommand{\setj}[1]{#1 \; \text{set}}
\newcommand{\hinj}[3]{#1 \in #2 \; (#3)}
\newcommand{\eqelj}[3]{#1 = #2 \in #3}
\newcommand{\heqelj}[4]{#1 = #2 \in #3 \; (#4)}
\newcommand{\seteqj}[2]{#1 = #2}
\newcommand{\hseteqj}[3]{#1 = #2 \; (#3)}
\newcommand{\inj}[2]{#1 \in #2}
\newcommand{\NN}{\mathbb{N}}
\newcommand{\natrec}[3]{\textsf{R}(#1, #2, #3)}
\newcommand{\ap}[2]{\textsf{Ap}(#1, #2)}
\newcommand{\abs}[2]{(\lambda #1)(#2)}

\newcommand{\derives}[2]{\[#1 \proofdotnumber=0 \leadsto #2\]}
\newcommand{\derivesfrom}[3]{\[#1 \using #2 \proofdotseparation=1.2ex
    \proofdotnumber=4 \leadsto #3\]}

\newcommand{\pitype}[2]{\prod_{#1} #2}
\newcommand{\bigpity}[2]{\displaystyle \prod_{#1} #2}
\newcommand{\smallpity}[2]{\textstyle{\prod}_{#1} #2}
\newcommand{\piab}{\smallpity{x\in A}{B(x)}}

\newcommand{\sigmatype}[2]{\sum_{#1} #2}
\newcommand{\sigmaab}{\smallsigmaty{x\in A}{B(x)}}
\newcommand{\smallsigmaty}[2]{\textstyle{\sum}_{#1} #2}
\newcommand{\bigsigmaty}[2]{\displaystyle \sum_{#1} #2}

\newcommand{\pair}[2]{\langle #1, #2 \rangle}
\newcommand{\fst}[1]{\textsf{fst}(#1)}
\newcommand{\snd}[1]{\textsf{snd}(#1)}

\newcommand{\onety}{\mathds{1}}
\newcommand{\twoty}{\textsf{2}}
\newcommand{\nty}[1]{\mathds{N}_{#1}}

\newcommand{\ttrue}{\textsf{true}}
\newcommand{\ffalse}{\textsf{false}}

\title{Type Theory lecture notes}

\begin{document}

\maketitle{}
\tableofcontents{}

\newpage

\section{Introduction}

Modern day mathematics is built on top of set theory, where the existence of a
universe of sets closed under standard operations is postulated. This approach
has two main weak points

\begin{itemize}
\item intrinsic difficulty, due to G\"odel's incompleteness theorem, in showing
  that such universe can exist at all;
\item assumption, lacking a reasonable philosophical justification, that all
  mathematical concepts can be reduced to a single notion, that of set, and a
  single relation, that of membership.
\end{itemize}

Under this approach, sets are implicitly understood to exist in a static
supernatural world, whereas mathematics, as all intellectual products, has a
\strong{human nature} with historical and \strong{dynamic} origin.

Here we consider the constructivist point of view, that has the motivation of
providing a \strong{finer grid} to look at reality, \strong{preserving} pieces
of information and conceptual \strong{distinctions} that would be lost of not
considered in the classical approach. Thus constructivism should firstly be seen
as a \strong{widening} of the scope of classical mathematics, and only
consequently as a restriction of its foundational principles.

Historically, there have been many approaches to constructivism. Our intention
is to preserve all notions and conceptual distinctions as much as possible, thus
leading to an arsenal of foundational principles that is reduced to a
\strong{minimum}. In particular, we wish to

\begin{itemize}
\item preserve an effective notion of set all of whose elements are produced by
  a finite number of rules, and to block impredicative definitions. Hence we
  avoid the powerset axiom;
\item preserve a positive notion of existence. Hence the underlying logic must
  be intuitionistic;
\item preserve the distinction between operation (algorithm, explicit procedure)
  and function (total and single-valued relation). Hence we avoid the axiom of
  unique choice AC! (and therefore AC), that would identify the two concepts.
\end{itemize}

An advantage of this approach is that every mathematician can recognize such
theory as one that can be consistently extended to match his own.

\subsection{Basic notions}

A central concept in the philosophy of constructive mathematics is that all
entites consist of constructions. Therefore, we wish to know and control the
concrete content of every entity that we manipulate and statement that we
assert. Hence, the preservation of information is of central importance.

\paragraph{Sets}

Here, sets have an effective nature, as opposed to axiomatic set theory:

\begin{itemize}
\item each set is specified by a finite number of rules to build all of its
  elements;
\item no universe of all sets is assumed to be given a priori. We say that
  ``set'' is an \emph{open} concept.
\end{itemize}

There is a human nature in these constructions: it is we that we call some
aggregate a set when we know how to build its elements. So to know that $X$ is a
set is to know how to build all its elements.  There is no need to specify the
language in which the rules are given, as soon as it can be understood by a
reasonable mathematician. This requirement is fullfilled as soon as the set is
described using

\begin{itemize}
\item a language that is unambiguous;
\item a finite amount of information (such as a finite number of rules, or an
  infinite number of them as described by a finite number of meta-level rules,
  etc...)
\end{itemize}

Every set automatically comes with a notion of \strong{element}, which is an
entity that can be obtained by repeated use of the rules associated with that
set. Every set $X$ must also come equipped with a specific notion of
\strong{equality} $=_X$, which must be an equivalence relation. Every
mathematical entity involving elements of $X$ is expected to respect this
equality. So in order to define a set, we need to give

\begin{itemize}
\item a finite number of rules that allow to construct all elements of the set;
\item a notion of equality between elements of the set.
\end{itemize}

\paragraph{Quotient sets}

It follows that sets are closed under \strong{quotients}: given a set $X$ and an
equivalence relation $\sim$ over $X$, we can construct a set $(X,\sim)$ simply by
changing $=_X$ with $\sim$, leaving the element-forming rules untouched.
Consider for example the set $X^*$ of finite lists of elements of the set
$X$. We can construct the set of finite subsets of $X$ by simply quotienting
$X^*$ with an equivalence relation $\sim$ that identifies lists with the same
elements regardless of the order of appearence or their quantity.

\paragraph{Collections}

We can consider objects of a certain logical type even if we have grasped what
it means to be an object of that type, without having rules to generate all of
them. We call these aggregates \strong{collections}, and we write $p :
\mathcal{P}$ for an object $p$ in a collection $\mathcal{P}$. Every collection
comes with a notion of equality preserved by all mathematical entities that
involve objects in that collection. So to know that some aggregate is a
collection is

\begin{itemize}
\item to have grasped what it means to be an object of that collection, and
\item to have given a notion of equality for objects of that collection
\end{itemize}

An example of a collection is the collection $\mathcal{P}(X)$ of subsets of a
set $X$. A simple diagonalization shows that it is impossible to generate all
possible subsets of a set by means of a finite number of rules, so
$\mathcal{P}(X)$ is certainly not a set. However, it is completely clear to us
what constitues a subset of a set, and when two subsets are equal.

The distinction between sets and collections has some analogy with the concepts
of set/class in classical set theory, but different motivations, that have to do
with

\begin{itemize}
\item the quality of information, distinguishing between effective, computable
  domains (sets) and non-computable ones (collections);
\item the flow of time: once a set is defined, it remains the same
  forever. Conversely, a collection may change its contents with the time. For
  example, the collection $\mathcal{P}(X)$ grows as we define new subsets of
  $X$.
\end{itemize}

Hence the need for collections as an entity that is distinct from sets comes
from the need, among others, to recognize and respect the dynamic nature of our
mathematical world.

\paragraph{Family of sets}

Having described sets and collections, we now need a primitve concept for
associating instances of these two concepts. Suppose we know that $T(x)
\text{set} (a \in I)$, that is, that $T(x)$ yields a description of a set
whenever $x$ is an explicit description of an element of the set $I$. We call
this a \strong{family of sets} indexed by $I$.

\paragraph{Operations}

In the same way as a family of sets maps elements of an index set to sets, we
may want to map element of a set to elements of another set. An
\strong{operation} between a set $I$ and a set $X$, written
$p(i) \in X (i \in I)$ is the same as a family of elements of $X$ indexed by
$I$, that is an explicit description $p(i)$ of an element of $X$ for each
element $i$ of $I$. Operations must preserve equality, that is, $p(i) =_X p(i')$
whenever $i =_I i'$.
Equality between operations is extensional, that is, we conclude $p = p'$
whenever $p(i) =_X p'(i)$ for all $i \in I$.
We denote by $\mathrm{Op}(I,X)$ the collection of operations between the sets
$I$ and $X$.

The difference between functions (total single-valued relations) and operations
is clearified by a simple diagonalization argument. We have an operation $p$
from $A$ to $B(x)$ whenever we are able to derive the judgement
$\hinj{p(x)}{B(x)}{x \in A}$, meaning that we are able to effectively enumerate
all operations that can be derived in our system. These are moreover total, and
it is a well known result of computability theory that such enumeration cannot
exhaustively enumerate all total effective functions. We therefore must conclude
that there are strictly more functions than operations.

An operation $\mathcal{O}$ from a collection $\mathcal{P}$ into a collection
$\mathcal{Q}$ is the same as a family of objects of $\mathcal{Q}$ indexed on
$\mathcal{P}$, i.e., an explicit description of an object $\mathcal{O}(p)$ in
$\mathcal{Q}$ for an arbitrary $p : \mathcal{P}$. Note that $\mathcal{O}$ can be
well-defined even if we cannot produce all its arguments. In fact, correct
effective operations may be based only of the evidence that the argument object
actually belongs to the indended collection. An example of this is the identity
operation on collections.

We will also consider families of objects in a collection indexed by a set, that
is, operations $p(i) : \mathcal{P} (i \in I)$ from a set $I$ to a collection
$\mathcal{P}$. An example of this is the operation with logical type $I
\rightarrow \mathcal{P}(X)$ that associates every index to a subset of $X$.

%%% Local Variables:
%%% mode: latex
%%% TeX-master: "../notes"
%%% End:


\newpage
\section{Judgements}

In usual mathematical practice, we specify formulas and rules of inference, and
then \emph{a posteriori} give them a semantical interpretation, usually
interpreting them in set theory. Here we will avoid keeping form and meaning
(content) apart, and instead we will at the same time display certain forms of a
judgement and inference and explain them semantically.

Our formal system will consist of a set of rules for deriving judgements of a
variety of forms. Judgements are statements about one or more syntactic objects
of a specified sort, and they must be explained by saying what it is that we
must know in order to have the right to make a judgement of any one of the
various forms.

So far we have introduced, in particular, the notion of set and element of a
set. So it seems mandatory to have judgements that allow us to assert facts
abount these entities. Hence, we consider four forms:

\begin{enumerate}
\item $\setj{A}$ ($A$ is a set);
\item $A = B$ ($A$ and $B$ are equal sets);
\item $\inj{a}{A}$ ($a$ is an element of the set $A$);
\item $\eqelj{a}{b}{A}$ ($a$ and $b$ are equal elements of the set $A$).
\end{enumerate}

\subsection{Explanations of the forms of judgement}

\subsubsection{$A$ set}

To explain the judgement $\setj{A}$, we have to know what a set is, or
equivalently what it is to be known in order to have the right to make such a
judgement. Intuitively we know that a set is defined by prescribing how its
elements are formed. But $2 + 2$ is intuitively an element of $\NN$, even if it
cannot be obtained by the rules. So we need to distinguish between
\emph{canonical} and \emph{noncanonical} elements. Hence \emph{a set $A$ is
  defined by prescribing how a canonical element of $A$ is formed, as well as
  how equal canonical elements of $A$ are formed}.

\subsubsection{$A = B$}

Two sets $A$ and $B$ are equal if

\[
  \begin{prooftree}
    \inj{a}{A}
    \justifies
    \inj{a}{B}
  \end{prooftree},\qquad
  \begin{prooftree}
    \inj{a}{B}
    \justifies
    \inj{a}{A}
  \end{prooftree}
\]

and

\[
\begin{prooftree}
    \inj{a = b}{A}
    \justifies
    \inj{a = b}{B}
  \end{prooftree},\qquad
  \begin{prooftree}
    \inj{a = b}{B}
    \justifies
    \inj{a = b}{A}
  \end{prooftree}
\]

for arbitrary canonical elements $a, b$.

\subsubsection{$\inj{a}{A}$}

Assuming that we know what a set $A$ is, and therefore how to construct its
canonical elements, then \emph{an elements $a$ of a set $A$ is an effective
  method which, when executed, yields a canonical element of $A$ as a result.}

\subsubsection{$a = b \in A$}

Finally \emph{two arbitrary elements $a$ and $b$ of a set $A$ are equal if, when
executed, $a$ and $b$ yield equal canonical elements of $A$ as results.}

This definition makes sense because we assume that we already know that $A$ is a
set, and therefore when two canonical elements are equal.

\subsection{Rules of equality}

TODO...

\subsection{Hypothetical judgements and substitution rules}

We have seen that a family of elements (and similarly a family of sets) can be
expressed in the form $\hinj{b(x)}{B}{x \in A}$. This is just a generalization
of the judgement $\inj{b}{B}$, meaning that $b$ is an element of set $B$, to a
hypothetical judgement saying that $b(x)$ is an element of the set $B$ provided
$x$ is an element of the set $A$. Such a generalization can be carried out for
all four forms of judgement, as follows.

First, we see how terms in the theory are represented in the judgements. In
particular, we have terms/expressions $t$. These terms can have zero or more
free variables in them, hence $t(x_1, \dots, x_n)$. We can think of the term
$t(a_1, \dots, a_n)$ as the term $t$ in which the free variables have been
replaced with the terms $a_1, \dots, a_n$. The syntax and grammar of terms is
left unspecified and open-ended, we just assume a primitive notion of
\emph{variable} and sustitution.

Associated to the notion of (free) variable, there is the notion of abstraction,
which means that a variable in a term acts as a mere place-holder. We denote by
$(x)t$ the abstraction of $x$ in the term $t$, which may of may not be free in
it.

\paragraph{1.}

The first form of hypothetical judgement is the generalization of the set
judgement

\[
  \hsetj{B(x)}{x \in A}
\]

which says that $B(x)$ is a set under the assumption that $x \in A$, of that $B$
is a family of sets indexed by $A$. Moreover, such a family should respect
equality of the index set involved, hence justifying the following
\emph{substitution rules}

\[
  \begin{prooftree}
    \inj{a}{A} \qquad \hsetj{B(x)}{x \in A}
    \justifies
    \setj{B(a)}
  \end{prooftree}, \qquad
  \begin{prooftree}
    \eqelj{a}{c}{A} \qquad \hsetj{B(x)}{x \in A}
    \justifies
    \seteqj{B(a)}{B(c)}
  \end{prooftree}
\]

\paragraph{2.}

We said that equality between families of sets is meant to be extensional. This
is the meaning of the hypothetical judgement

\[
  \hseteqj{B(x)}{D(x)}{x \in A}
\]

which says that $B(x)$ and $D(x)$ are equal families of sets over the set
$A$. Hence the following substitution rule is justified

\[
  \begin{prooftree}
    \inj{a}{A} \qquad \hseteqj{B(x)}{D(x)}{x \in A}
    \justifies
    \seteqj{B(a)}{D(a)}
  \end{prooftree}
\]

We can now derive the rule

\[
  \begin{prooftree}
    \eqelj{a}{c}{A} \qquad \hseteqj{B(x)}{D(x)}{x \in A}
    \justifies
    \seteqj{B(a)}{D(c)}
  \end{prooftree}
\]

TODO...

\paragraph{3.}

A judgement of the form

\[
  \hinj{b(x)}{B(x)}{x \in A}
\]

means that we know $b(a)$ to be an element of the set $B(a)$ assuming we know
$a$ to be an element of the set $A$, that is, $b$ is an operation/family from
elements of the set $A$ to elements of the corresponding set in the family
$B(x)$. Moreover, operations must respect equality, thus justifying the
following substitution rules:

\[
  \begin{prooftree}
    \inj{a}{A} \qquad \hinj{b(x)}{B(x)}{x \in A}
    \justifies
    \inj{b(a)}{B(a)}
  \end{prooftree}, \qquad
  \begin{prooftree}
    \eqelj{a}{c}{A}\qquad \hinj{b(x)}{B(x)}{x \in A}
    \justifies
    \eqelj{b(a)}{b(c)}{B(a)}
  \end{prooftree}
\]

\paragraph{4.}

The judgement

\[
  \heqelj{b(x)}{d(x)}{B(x)}{x \in A}
\]

means that $b(x)$ and $d(x)$ are (extensionally) equal operations/families of
elements indexed by $A$, hence justifying the following substitution rule:

\[
  \begin{prooftree}
    \inj{a}{A} \qquad \heqelj{b(x)}{d(x)}{B(x)}{x \in A}
    \justifies
    \eqelj{b(a)}{d(a)}{B(a)}
  \end{prooftree}
\]

\subsubsection{Hypothetical judgements with more than one assumption}

TODO...

%%% Local Variables:
%%% mode: latex
%%% TeX-master: "../notes"
%%% End:


\newpage
\section{Sets}

We have already seen that to define a set is to give an exhaustive method (in
the form of a finite number of rules) to construct its canonical elements,
together with a notion of equality between those elements. We now give a precise
description of the forms of rules that will be used to introduce sets in our
theory. There will be four kinds of rules:

\begin{enumerate}
\item Set formation rules: they describe under which conditions we may infer
  that $A$ is a set, and when two sets $A$ and $B$ are equal;
\item Introduction rules: they describe how to construct the canonical elements
  of a set, and when two canonical elements are to be considered equal;
\item Elimination rules: they show how to \emph{use} the elements of a set, or
  equivalently how to define operations of elements of the set. These give a
  form of induction/recursion, saying that in order to define an operation on
  elements of $A$ it is sufficient to define it only of its canonical elements;
\item Equality rules: they relate introduction and elimination rules by
  describing how the selector computes on the different canonical elements;
\end{enumerate}

Note that elimination rules have a \strong{double meaning}: they show how to
define operations by (primitive) recursion on the canonical elements of a set,
and they equip the inductive type with an induction principle, basically saying
that the set is the least one that is closed under the introduction rules.
Elimination rules come in pairs, where the second shows how the eliminator
symbol that we introduced \strong{respects equality}.

Equality rules give a formal and precise (computational) semantics to the
symbols that are introduced in the elimination rules. In particular, they
explain how these symbols behave by showing how they compute on the canonical
elements. By definition of element of a set, we know this is enough.

\subsection{Finite types}

Finite types are defined by the following meta-rules of formation

\[
  \setj{\nty{n}}, \qquad \seteqj{\nty{n}}{\nty{n}}
\]

and introduction (with the obvious equalities):

\[
  \inj{0_n}{\nty{n}}, \; \inj{1_n}{\nty{n}}, \; \dots, \;
  \inj{(n-1)_n}{\nty{n}}
\]

As an example of how elimination rules equip inductive sets with a principle of
induction, to define an operation on arbitrary elements of $\nty{n}$ is to
define it on all possible elements of the set:

\[
  \begin{prooftree}
    \hsetj{C(c)}{c \in \nty{n}}
    \quad \inj{c}{\nty{n}}
    \quad \inj{d_1}{C(0_n)}
    \quad \inj{d_2}{C(1_n)}
    \quad \dots \quad  \inj{d_n}{C((n-1)_n)}
    \justifies
    \inj{E_n(c, d_1, d_2, \dots, d_n)}{C(c)}
  \end{prooftree}
\]

again with the obvious equalities. The computation rule is also straightforward:

\[
  \begin{prooftree}
    \hsetj{C(c)}{c \in \nty{n}}
    \qquad \inj{d_1}{C(0_n)}
    \qquad \inj{d_2}{C(1_n)}
    \qquad \dots \qquad  \inj{d_n}{C((n-1)_n)}
    \justifies
    \eqelj{E_n((i-1)_n, d_1, d_2, \dots, d_n)}{d_i}{C((i-1)_n)}
  \end{prooftree}
\]

Among the finite types, two will be particularly important, hence we abbreviate
them as such

\[
  \onety \equiv \nty{1},\qquad \twoty \equiv \nty{2}
\]

Note that $\onety$ is what is usually called \emph{unit} in programming
languages, whereas $\twoty$ is the boolean type. We also introduce the following
abbreviations for their elements:

\[
  \star \equiv 0_1, \qquad \ffalse \equiv 0_2, \qquad \ttrue \equiv 1_2
\]

\begin{example}
  We now show that type theory, being a theory of \emph{dependent} types, is
  sufficiently powerful to accept programs such as the following as well-typed:

  \[
    \textsf{if}\; b \; \textsf{then}\; \textsf{true}\; \textsf{else}\; 5
  \]

  where $\inj{b}{\twoty}$ and $\inj{5}{\NN}$. Note that in conventional
  programming languages this program whould be ill-typed, since the two branches
  of the conditional statement have different types. Being type theory a far
  more expressive functional programming language, we can express this by saying
  that the return type of the conditional is a family of types depending on
  the value of the input; in particular, the family will map $\ttrue$ to $\twoty$
  and $\ffalse$ to $\NN$. So suppose we have a family $C$ such that $C(\ttrue) =
  \twoty$ and $C(\ffalse) = \NN$. Then:

  \[
    \begin{prooftree}
      \hsetj{C(b)}{\inj{b}{\twoty}} \qquad
      \inj{b}{\twoty} \quad
      \[
        \inj{\ttrue}{\twoty}\quad
        C(\ttrue) = \twoty
        \justifies
        \inj{\ttrue}{C(\ttrue)}
      \]
      \quad
      \[
        \inj{5}{\NN}\quad C(\ffalse) = \NN
        \justifies
        \inj{5}{C(\ffalse)}
      \]
      \justifies
      \inj{E_2(b, \ttrue, 5)}{C(b)}
    \end{prooftree}
  \]
  
\end{example}

\subsection{Natural numbers}

We now move to the most basic example of infinite set, namely the set of
non-negative integers, natural numbers. Infinite sets can be described by
giving a finite number of rules that describe their elements inductively:

\[
  \setj{\NN}, \qquad \seteqj{\NN}{\NN}
\]

\[
  \inj{0}{\NN},\qquad
  \begin{prooftree}
    \inj{n}{\NN}
    \justifies
    \inj{s(n)}{\NN}
  \end{prooftree}
\]

\[
  \eqelj{0}{0}{\NN},\qquad
  \begin{prooftree}
    \eqelj{n}{m}{\NN}
    \justifies
    \eqelj{s(n)}{s(m)}{\NN}
  \end{prooftree}
\]

The elimination rule is essentially a statement of the principle of induction
for natural numbers, or equivalently a rule to define operations by primitive
recursion (but in a more powerful way since the return type can depend on the
argument).

\[
  \begin{prooftree}
    \hsetj{C(n)}{\inj{n}{\NN}} \quad
    \inj{n}{\NN} \quad
    \inj{d}{C(0)} \quad
    \[(\inj{x}{\NN}, \inj{y}{C(x)})\justifies \inj{e(x,y)}{C(s(x))}\]
    \justifies
    \inj{\natrec{n}{d}{(x,y)e(x,y)}}{C(n)}
  \end{prooftree}
\]

The recursion symbol introduced by the elimination rule has a simple semantical
explanation in terms of its computations: it first evaluates $n$, which is known
to yield either $0$ or $s(n)$ for some $\inj{n}{\NN}$. In the first case, it
evaluates $d$ and returns the result. Otherwise, it takes $n$ and feeds it to
$e$ as its first argument, together with the result of evaluating itself on the
predecessor, namely $n$. The second elimination rule shows that the recursor
respects equality in the obvious way:

\[
  \begin{prooftree}
    \hsetj{C(n)}{\inj{n}{\NN}} \quad
    \eqelj{n}{m}{\NN} \quad
    \eqelj{d}{d'}{C(0)} \quad
    \[(\inj{x}{\NN}, \inj{y}{C(x)})\justifies \eqelj{e(x,y)}{e'(x,y)}{C(s(x))}\]
    \justifies
    \eqelj{\natrec{n}{d}{(x,y)e(x,y)}}
    {\natrec{m}{d'}{(x,y)e'(x,y)}}
    {C(n)}
  \end{prooftree}
\]

The equality rules, which are given for each possible way to construct
canonical forms, just show how recursion on the argument works.

\[
  \begin{prooftree}
    \hsetj{C(n)}{\inj{n}{\NN}} \quad
    \inj{d}{C(0)} \quad
    \[(\inj{x}{\NN}, \inj{y}{C(x)})\justifies \inj{e(x,y)}{C(s(x))}\]
    \justifies
    \eqelj{\natrec{0}{d}{(x,y)e(x,y)}}{d}{C(0)}
  \end{prooftree}
\]

\[
  \begin{prooftree}
    \hsetj{C(n)}{\inj{n}{\NN}} \quad
    \inj{d}{C(0)} \quad
    \[(\inj{x}{\NN}, \inj{y}{C(x)})\justifies \inj{e(x,y)}{C(s(x))}\]
    \justifies
    \eqelj{\natrec{s(n)}{d}{(x,y)e(x,y)}}{e(n,
      \natrec{n}{d}{(x,y)e(x,y)}
    }{C(s(n))}
  \end{prooftree}
\]

\begin{example}
  As an example, we can try to define the summation operation on natural
  numbers. Recall that it is defined as follows by primitive recursion:

  \begin{align*}
    \textsf{sum}(0,m) & = m \\
    \textsf{sum}(s(n),m) & = s(\textsf{sum}(n,m))
  \end{align*}

  In type theory it is no different:

  \[
    \begin{prooftree}
      \hsetj{\NN}{\inj{n}{\NN}} \qquad
      \inj{n}{\NN} \qquad
      \inj{m}{\NN} \qquad
      \[(\inj{x}{\NN}, \inj{y}{\NN})\justifies \inj{s(y)}{\NN}\]
      \justifies
      \inj{\natrec{n}{m}{(x,y)s(y)}}{\NN}
    \end{prooftree}
  \]

  Hence $\textsf{sum}(n,m) \equiv \natrec{n}{m}{(x,y)s(y)}$.

\end{example}

\begin{example}
  Multiplication can be similarly defined, in terms of summation:

  \begin{align*}
    \textsf{mul}(0,m) & = 0 \\
    \textsf{mul}(s(n),m) & = \textsf{sum}(m, \textsf{mul}(n,m))
  \end{align*}

  \[
    \begin{prooftree}
      \hsetj{\NN}{\inj{n}{\NN}} \qquad
      \inj{n}{\NN} \qquad
      \inj{0}{\NN} \qquad
      \[(\inj{x}{\NN}, \inj{y}{\NN}, \inj{m}{\NN})\justifies \inj{\textsf{sum}(m, y)}{\NN}\]
      \justifies
      \inj{\natrec{n}{0}{(x,y)\textsf{sum}(m,y)}}{\NN}
    \end{prooftree}
  \]

  Then, $\textsf{mul}(n,m) \equiv \natrec{n}{0}{(x,y)\textsf{sum}(m,y)}$.

\end{example}

\begin{exercise}
  Define the exponentiation operation.
\end{exercise}

\subsection{Disjoint union of a family of sets}

We have seen that the term $\hsetj{b(x)}{x \in A}$ denotes a family of sets
indexed by the elements of the set $A$. We may want to consider the disjoint
union of the sets the constitute the family:

\[
  \begin{prooftree}
    \setj{A} \qquad \hsetj{B(x)}{x \in A}
    \justifies
    \setj{\sigmaab}
  \end{prooftree}
\]

Two disjoint unions are equal when the index set and the family of which we do
the union are equal:

\[
  \begin{prooftree}
    A = C \qquad \hseteqj{B(x)}{D(x)}{x \in A}
    \justifies
    \seteqj{\sigmaab}{\sigmatype{x \in C}{D(x)}}
  \end{prooftree}
\]

The elements of a disjoint union of sets are, intuitively, all the elements from
all the sets that constitute the family, together with the information that
tells us from which set of the family the element comes from. This information
is easily preserved by remembering the index of the sets from which the element
comes from.

\[
  \begin{prooftree}
    \inj{a}{A} \qquad \inj{b}{B(a)}
    \justifies
    \inj{(a,b)}{\sigmaab}
  \end{prooftree}
\]

Equal canonical elements of this set can be formed when the two components are
equal (possibly non-canonical) elements:

\[
  \begin{prooftree}
    \eqelj{a}{c}{A} \qquad \eqelj{b}{d}{B(a)}
    \justifies
    \eqelj{(a,b)}{(c,d)}{\sigmaab}
  \end{prooftree}
\]

The \strong{elimination rule} introduces the eliminator $E$:

\[
  \begin{prooftree}
    \derives{(\inj{z}{\sigmaab})}{\setj{C(z)}} \qquad
    z \in \sigmaab \qquad
    \derives{(x \in A, \; y \in B(x))}{\inj{d(x,y)}{C(\langle x, y \rangle)}}
    \justifies
    E(z,(x,y)d(x,y)) \in C(z)
  \end{prooftree}
\]

We can justify the elimination rule by explaining the semantics of the
eliminator $E$: it first executes $z$, which by hypothesis will yield a
canonical element of type $\sigmaab$, that is, a pair $\pair{a}{b}$ with
$\inj{a}{A}$ and $\inj{b}{B(a)}$. Then, it substitutes $a$ and $b$ for $x$ and
$y$ in $d$. Executing $d(a,b)$ will return a canonical element $e$ of
$C(\pair{a}{b})$. Since $\pair{a}{b}$ is the normal form of $z$, it follows that
$\eqelj{z}{\pair{a}{b}}{\sigmaab}$, so since type families respect equality, we
have that $\inj{e}{C(z)}$, which is what we wanted.

The second elimination rule tells us how the eliminator symbol respects equality
of the entities involved.

\[
TODO...
\]

The equality rule tells us the semantics of the eliminator symbol when acting
of the canonical elements, that is, the pair.

\[
  \begin{prooftree}
    \inj{a}{A} \qquad \inj{b}{B(a)} \qquad
    d(x,y) \in C(\langle x, y \rangle) (x \in A, y \in B(x))
    \justifies
    E(\pair{a}{b},(x,y)d(x,y)) = d(a,b) \in C(\pair{a}{b})
  \end{prooftree}
\]

The equality rule above is justified by the very definition that we just gave
for the eliminator $E$.

\begin{example}
  

We now define the first and second projection operations as an example of how to
use the rules introduced so far to build operations. The first projection should
be an operation $\textsf{fst}$ such that

\[
  \hinj{\fst{z}}{A}{z \in \bigsigmaty{x\in A}{B(x)}}
\]

hence we use the elimination rule of the disjoint union type to define it:

\[
  \begin{prooftree}
    \hsetj{A}{z \in \sigmaab} \quad \inj{z}{\sigmaab} \quad
    \begin{prooftree}
      (x \in A, y \in B(x))
      \justifies
      x \in A
    \end{prooftree}
    \justifies
    E(z,(x,y).x) \in A
  \end{prooftree}
\]

So we put $\textsf{fst}(z) \equiv E(z,(x,y).x)$. We have to check that the
operation we defined has the intended behaviour. This can be done with the
equality rule:

\[
  \begin{prooftree}
    \hsetj{A}{\inj{z}{\sigmaab}} \quad \inj{z}{\sigmaab} \quad
    \begin{prooftree}
      (x \in A, y \in B(x))
      \justifies
      x \in A
    \end{prooftree}
    \justifies
    E(z,(x,y).x) = x \in A
  \end{prooftree}
\]
\end{example}

\begin{example}
  We now proceed to define the second projection. Note that this is a
  not-so-trivial projection operation, since the result set of this operation
  depends on the first component of the pair:

  \[
    \hinj{\snd{z}}{B(\fst{z})}{z \in
      \bigsigmaty{x\in A}{B(x)}}
  \]

  We now have the following derivation:


  \[
    \begin{prooftree}
      % set family
      \[
        \hsetj{B(x)}{x \in A} \quad
        \[
          (\inj{z}{\sigmaab})
          \justifies
          \fst{z} \in A
        \]
        \justifies
        \setj{B(\snd{z})}
      \]
      \quad
      % element
      \inj{z}{\sigmaab}
      \quad
      \inj{y}{B(\fst{\langle x, y \rangle})}
      \justifies
      E(z,(x,y).y) \in B(\fst{z})
    \end{prooftree}
  \]

where the last judgement can be derived as follows:

\[
  \begin{prooftree}
    (\inj{y}{B(x)}) \quad
    \[
      \hsetj{B(x)}{x \in A} \quad
      \[
        \[
          \[
            (\inj{x}{A}) \quad (\inj{y}{B(x)})
            \justifies
            \inj{\langle x, y \rangle}{\sigmaab}
          \]\quad
          (\inj{x}{A})
          \justifies
          \eqelj{\fst{\langle x, y \rangle}}{x}{A}
        \]
        \justifies
        \eqelj{x}{\fst{\langle x, y \rangle}}{A}
      \]
      \justifies
      \seteqj{B(x)}{B(\fst{\langle x, y \rangle})}
    \]
    \justifies
    \inj{y}{B(\fst{\langle x, y \rangle})}
  \end{prooftree}
\]

The computation using the equality rule is showed similarly to the first
projection.
\end{example}

\subsection{Cartesian product of a family of sets}

In the same way as we constructed the disjoint union of a family of sets, we may
as well want to create the cartesian product of a family of sets. In particular,
given a family of sets $B(x) \mathrm{set} (x \in A)$, we write $\setj{\piab}$
for the cartesian product of that family:

\[
  \begin{prooftree}
  A \; \text{set} \quad \hsetj{B(x)}{x \in A}
  \justifies
  \setj{\piab}
  \end{prooftree}
\]

Intuitively, an element of the set $(\Pi x \in A) B(x)$ is a tuple containing
one element for each set in the family $B(x)$. Alternatively, one can think of
such a tuple as an operation associating every element $a$ of the index set $A$
to an element of the set in the family that corresponds to that element, that is
$B(a)$. We can express such \emph{generalized} operation as

\[
  \hinj{b(x)}{B(x)}{x \in A}
\]

Such operations, following our explanation of the set, constitute
essentially the elements of the cartesian product:

\[
  \begin{prooftree}
  \hinj{b(x)}{B(x)}{x \in A}
  \justifies
  \inj{(\lambda x)b(x)}{\piab}
  \end{prooftree}
\]

It is necessary to understand that $\hinj{b(x)}{B(x)}{x \in A}$ is an operation
to be able to form the canonical element
$\inj{(\lambda x)b(x)}{\pitype{x\in A}{B(x)}}$. Since, in general, there are no
exhaustive rules for generating all operations from one set to another, it
follows that the introduction rule above is justified by the fact that we take
the notion of operation to be primitive.

Notice that, in the special case where $x$ is not free in $B$, the family
$\hinj{b(x)}{B}{x \in A}$ is just the usual operation from a set $A$ to a set
$B$. In such cases, we identify the cartesian product with the type of
operations, $\pitype{x\in A}{B(x)} \equiv A \rightarrow B$, and its elements with
the operations from $A$ to $B$.

\begin{example}
  We can now derive the operation that usually comes with the name
  \emph{currying}

  \[
    \textsf{curry} \in (f \in A \times B \rightarrow C) \rightarrow
    (A \rightarrow B \rightarrow C)
  \]

  as follows

  \[
    \begin{prooftree}
      \[
        \[
          \[
            (f \in A \times B \rightarrow C)
            \qquad
            \[
              (\inj{a}{A}) \qquad (\inj{b}{B})
              \justifies
              \inj{\pair{a}{b}}{A \times B}
            \]
            \justifies
            \ap{f}{\pair{a}{b}} \in C
          \]
          \justifies
          \abs{b}{\ap{f}{\pair{a}{b}}} \in B \rightarrow C
        \]
        \justifies
        \abs{a}{\abs{b}{\ap{f}{\pair{a}{b}}}} \in A \rightarrow B \rightarrow C
      \]
      \justifies
      \abs{f}{\abs{a}{\abs{b}{\ap{f}{\pair{a}{b}}}}} \in (f \in A \times B \rightarrow C) \rightarrow
      (A \rightarrow B \rightarrow C)
    \end{prooftree}
  \]

  Then, simply put $ \textsf{curry} \equiv \abs{f}{\abs{a}{\abs{b}{\ap{f}{\pair{a}{b}}}}}$.
\end{example}

%%% Local Variables:
%%% mode: latex
%%% TeX-master: "../notes"
%%% End:


\newpage
\section{Intuitionistic logic}


%%% Local Variables:
%%% mode: latex
%%% TeX-master: "../notes"
%%% End:


\end{document}
